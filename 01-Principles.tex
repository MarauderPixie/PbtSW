\chapter{Principles Of Play} \label{principles}

\epigraph{Mr. Johnson does not play dice with his Runners; He plays an ineffable game of His own devising, which might be compared, from the perspective of any of the other players (i.e. everybody), to being involved in an obscure and complex variant of poker in a pitch-dark room, with blank cards, for infinite stakes, with a Dealer who won't tell you the rules, and \textit{who smiles all the time}.}{\textit{adapted from Terry Pratchett}}

\textbf{Fiction First:} everything that happens in a session of Sixth World starts with the fiction, proceeds to rules (if necessary), and ends with the fiction.

\textbf{Moves are Not Powers:} most of the game's rules are encapsulated in small packages called moves. A move provides the rules to resolve particular situations that arise in the fiction (for instance, how to shoot someone, or seduce someone). Try not to think of moves as powers you must activate or ``use'', but as the rules that come into play when your character gets into a situation.

\textbf{Never Say Your Move:} because the game starts with and ends with the game fiction, you won’t say ``I use \skill{Rock \& Roll} on that guy!'' Instead, determine from what you are doing in the game world (running, shooting, jumping, dying, etc.) what move would apply. When the rolling is done, you conclude with some more fiction (or perhaps the GM does, depending on the outcome). This is the story, not whether you used \skill{Rock \& Roll} or \skill{Stay Frosty}.

\textbf{Fiction Forces:} if you do something in the game world that would trigger a move, then you \textbf{must} make that move. You can’t say ``I’m diving into the closet to avoid being spotted'' and then not make the \skill{Stay Frosty} move. Conversely, you can’t make a move unless the situation actually demands it. If you’re not fighting someone who’s fighting back, then you don’t get to make the \skill{Rock \& Roll} move. The game fiction dictates what happens.

\textbf{Speak to the Characters:} since the fiction anchors the game, remember that if you want to speak to or ask something of Valentin, the character being played by Keith, don’t say ``Hey Keith, do you have a spare frag grenade?'' Instead, speak to the character: ``Hey, Valentin, do you have a spare frag?'' (remember, though, you don’t have to act in first person! it’s okay to speak about your character, not as your character, if you prefer).


\section{Stats} \label{stats}

Most of the rules of Sixth World rely on the value of a player character’s Stats. You’ll hear more about these later on (especially when you get to the \nameref{Dossiers}), but every player character is described by 5 basic stats.

Each stat could fill in the blank in the following sentence: How \_\_\_\_\_\_\_\_\_\_ is my character?

\begin{easylist}
    # \textbf{Sharp:} \textit{alert, quick, perceptive, and instinctive}
    # \textbf{Hard:} \textit{ruthless, cold, and willing to do harm}
    # \textbf{Steady:} \textit{focused, cool, and mentally and physically tough}
    # \textbf{Skilled:} \textit{educated, trained, skillful, and intelligent}
    # \textbf{Smooth:} \textit{stylish, appealing, and charismatic}
\end{easylist}

And two pools of points that fluctuate in the course of play:

\begin{easylist}
# \textbf{Essence:} your life force and (meta)humanity, this also fuels the powers of magical archetypes (Adept, Mage, and Shaman)
# \textbf{Edge:} a pool of points used to boost you when you need it, or bail you out in a tight spot. Your Edge starts out at zero, but gain in through Experience, if you live long enough. You gain one Edge every third Advance, instead of new move or stat boost.
\end{easylist}

% \subsection{Sag's ihnen ins Gesicht}



\section{Rolling The Dice}

In this game, the dice rolling revolves around the concept of the Move. When you are instructed to roll dice for a move, your responsibility is simple: roll 2d6, and add the value of a stat (or sometimes some other value) to the result. When a roll is needed, it is usually phrased as \skill{roll+Stat}, where ``Stat'' is the value of your characters stat to add to the roll.

\textbox{Example}{If you are told to \skill{roll+Steady}, you would roll 2d6, sum the total, and add the value of your Steady stat to the result.}

The total of the roll indicates the outcome of the action taken by the character:

\begin{easylist}
    # On a 10+, you achieve a strong success: you've achieved your aim without complication, and to the fullest extent possible.
    # On 7-9, you have achieved a weak success: your achieve your aim, but with a cost. You will usually be presented with a list of complications to choose from, although sometimes instead the GM will tell you what complication occurs.
    # On a total of 6 or less, you have failed: you don’t get what you want. In fact, things are probably going to get worse.
\end{easylist}

Note that if a move just says ``roll'', then you don’t add anything — just roll 2d6. In addition to the common 2d6 roll, Sixth World uses the other common polyhedral dice: d4, d6, d8, d10, and d12. Twenty-sided dice are not used for mechanics, but can be used for some of the random generators at the end of this document.


\subsection{Roll Modifiers}
While the basic move roll is 2d6+(something), there are a few modifiers and tricks that may apply to a roll. The rules will always indicate when to use one of these modifiers.

\textbf{boosted:} whenever you are boosted, your result is never lower than 7 (even if you roll 6 or less). So, when boosted, you cannot fail, though success may still come at a cost (not least of which is the fact that while boosted, you can’t receive Karma while boosted).

\textbf{glitched:} glitched rolls are the opposite of boosted rolls. Whenever you are glitched, your result is never higher than 9, even if you rolled a 10+. You can succeed while glitched, but it will always come with a cost.

\textbf{hold \textit{n}:} when you are told to Hold n, or that you gain n Hold, this means you have a small pool of points that can be spent at some future moment of your choosing. You will be told on what, specifically, you may spend the Hold. Note that if you can spend Hold on a dice roll, you can do so after you see the results of the roll!

\textbf{take +n forward/-n forward:} this means take a bonus (the +) or a penalty (the -) equal to n to your next Move.

\textbf{take +n ongoing/-n ongoing:} this means to take a bonus or penalty equal to n to all of your future rolls, until whatever circumstances caused the ongoing modifier have changed.

\textbf{b:} this means ``take the best of'' - you roll multiple dice, but keep only one of them to determine the final total. For instance, if you are instructed to roll 2d6b, you would roll 2d6, and keep the highest die. When written by itself (without a dice expression) it will be written as [b].

\textbf{w:} this means ``take the worst of'' - if you are instructed to roll 2d6w, then you would roll 2d6 and keep the lowest die. When written by itself (without a dice expression), it will be written as [w].


\section{Essence}

Every character in Sixth World has a stat called essence, representing their humanity, life force, and mystical connection with the world.

Characters start with 6 essence, although this may be reduced through the installation of cyberware. Essence can also be lost to some creatures and to certain injuries, depending on what optional rules you have in effect.

Essence is an important characteristic in three ways:

\begin{easylist}
    # It fuels the Adept’s powers, the Mage’s spells, and the Shaman’s spirits.
    # It acts as a limit on the amount of cyberware any character can carry.
    # It may be the thing that saves your life when the chips are down. See the \skill{Last Chance} move.
\end{easylist}


\section{Edge}

Each character has a pool of points called Edge. Edge is an in-game currency representing a number of real-world (or at least, game-world) concepts, from luck to experience to their ability to turn a bad situation into a survivable one.

\subsection{Spending Edge}
The main way to spend Edge is to gain bonuses to rolls. When a player wishes it, they can spend Edge as follows:

\begin{easylist}
    # To improve damage: for every point of Edge spent, they can add +1d6 damage to their most recent attack.
    # To boost a roll: a character can spend one point of Edge to be boosted on their next roll.    
\end{easylist}

Edge refreshes at a rate of one point per day (assuming a good night’s rest).

\subsection{Earning Edge}
When you start play, your Edge value is zero. You gain one Edge every third Advance, instead of normal advancement.


\section{Karma}

Characters earn Karma (typically called ``Receiving Karma'') as they navigate the shadows, get in fights, and survive their adventures in the Sixth World. Characters can receive Karma in the following circumstances:

\begin{easylist}
# when they finish a run, or a significant portion of a major run
# when they resolve one of the debts or favors they have with another character
# when they are manipulated (see page 4) by another character
\end{easylist}

Once a character received 5 Karma, they may use the \skill{Advance} move to ``spend'' that Karma to improve their character. If you already have 5 Karma marked, you don’t receive any more Karma until you advance. You're not Buddha and can only learn so much at a time.


\section{Debts \& Favors}

Nobody goes it alone in the shadows for long. Sooner or later, you need to get help from somebody. Sometimes, you can buy that help with money. Other times, legal tender won’t cover it and that’s when debts and favors come into play.

The total number of Debts \& Favors you have with another character on your team equals your Bond with that character. For example, if you have 2 debts to another character, your Bond with them is 2. If, at the end of a session, you have resolved one of these bonds, you erase the debt or favor, and you and the other runner receive Karma.

\subsection{Debts}
A debt is something you owe a fellow runner. Maybe they yanked your ass out of a bad situation down in Aztlan, or helped spring you from jail, or just lent you some of their own hard-won experience that saved your bacon.

\subsection{Favors}
A favor, conversely, is something owed to you by a fellow runner. Maybe you were the one doing the hot-LZ extraction in Aztlan, or you took the rap for them on a particular smash ‘n grab job.

Debts and favors are not necessarily reciprocal! A character might perceive a debt to another that is entirely self-imposed. Conversely, a character might feel like one of their teammates owes them something, while that teammate might be completely unaware of that feeling. So, when establishing debts and favors, don’t assume that a debt on one sheet has to correspond to a favor on another.
