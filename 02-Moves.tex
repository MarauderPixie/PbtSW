\chapter{Moves} \label{moves}

In Sixth World, the place where rules and fiction intersect are the character’s Moves. Moves are the mechanical structure used when the fictional actions of a character require some resolution, and where the outcome of such actions is sufficiently interesting - or in doubt - as to be worth taking a risk to achieve.

It is tempting to think of moves as a character’s ``powers'' or ``abilities'', but remember: you should not be looking for a move to make. Instead, you should describe fictional actions that fit the circumstances, and when those actions trigger a move, then you engage the game mechanics to determine the outcome.

There are four general categories of moves in Sixth World: Core, Secondary, Archetype, and Metatype.

Core moves are the most commonly used moves, and provide mechanics for frequent activities like fighting, hiding, looking around, and interacting.

Secondary moves are less frequently used, and are usually situational.

Archetype moves are moves unique to one of the character archetypes, and reflect their particular abilities.

Metatype moves are moves that reflect the differing traits of the five human metatypes in the game. Core, secondary, and metatype moves are detailed on the following pages. Archetype moves can be found in the dossier for each archetype.

\section{CORE MOVES}

\paragraph{CHECK THE SITUATION}
When you check out a charged situation, roll+Sharp. On a hit, you can ask the GM questions. Whenever you act on one of the GM’s answers, take +1.

On a 10+, ask 3. On a 7–9, ask 1:

\begin{easylist}
    # Where’s my best escape route / way in / way past?
    # Which enemy is most vulnerable to me?
    # Which enemy is the biggest threat?
    # What should I be on the lookout for?
    # What’s my enemy’s true position?
    # Who’s really in control here?
    # Ask the GM a question of your own. If they will answer it, it stands; otherwise, retract it and ask another.
\end{easylist}


\paragraph{READ A PERSON}
When you read a person in a charged interaction, roll+Sharp. On a 10+, hold 3. On a 7–9, hold 1. While you’re interacting with them, spend your hold to ask their player questions, 1 for 1:

\begin{easylist}
    # Is your character telling the truth?
    # What’s your character really feeling?
    # What does your character intend to do?
    # What does your character wish I’d do?
    # How could I get your character to \_\_\_\_\_?
    # Ask their player a question of your own. If their player will answer it, it stands; otherwise, retract it and ask another.
\end{easylist}

The player must answer truthfully.


\paragraph{RECALL}
When you consult your knowledge of a specific topic or determine facts about your environment, roll+Skilled. On 10+, the GM will tell you give you a useful, specific detail about the situation. On 7-9, the GM will give you a general impression.

\paragraph{FUCK IT UP / MAKE IT RAIN}
When you aid or interfere with someone you have Bond with, roll+your Bond with them. On 10+, they are boosted or glitched, your choice. On 7-9, they’re still boosted or glitched, but you are exposed to danger or retribution.

\paragraph{SEDUCE / MANIPULATE}
When you try to seduce or manipulate someone, tell them what you want and roll+Smooth.
For NPCs: on a hit, they ask you to promise something first, and do it if you promise. On a 10+, whether you keep your promise is up to you, later. On a 7–9, they need some concrete assurance right now.
For PCs: on a 10+, both. On a 7–9, choose 1:

\begin{easylist}
    # if they do it, they receive Karma
    # if they refuse, they have to Stay Frosty 
\end{easylist}

What they do then is up to them.


\paragraph{MAKE ‘EM SWEAT}
When you impose your will on someone through violence or threat thereof, roll+Hard. On a 7+, they choose one:

\begin{easylist}
    # do what you say
    # get the hell out
    # attack you
\end{easylist}

On a 10+, you also take +1 forward against them. On a miss, they do what they want (or, if it’s an NPC, the GM makes their move), and you take -1 forward against them.


\paragraph{ROCK \& ROLL}
When you attack an enemy, roll+Hard. Determine the outcome based on the range at which you attack:

\begin{easylist}
    # Melee Combat: on 10+, you hit and deal your damage. On 7-9, you hit and deal damage, but your target attacks you as well.
    # Ranged Combat: on 10+, you hit and deal your damage. On 7-9, you deal damage, but (choose 1):
    ## you must expose yourself to danger or attack
    ## you burn up ammunition; mark off 1 ammo
    ## you only graze the target (-2 damage)
\end{easylist}


\paragraph{STAY FROSTY}
When you try to stay frosty in the face of pain, danger, urgency, impatience, or emotion roll+Steady. On 10+, you succeed. On 7-9, you succeed, but the GM will present you with a worse outcome, hard bargain, or ugly choice.



\section{SECONDARY MOVES}

\paragraph{ADVANCE}
When you have downtime and have marked 5 Karma, you can spend time reflecting on your experiences and honing your skills. When you Advance, choose one of the following:
\begin{easylist}
    # increase a stat (each stat may be advanced only once; check the small box in the stat area to indicate a stat that has already been advanced)
    # gain a new move from your dossier
    # gain a move from another Archetype’s dossier (up to three times)
\end{easylist}
You may only choose one benefit each time you advance. However, you can choose a benefit multiple times, subject to the limits specified above. Once you have advanced, clear your Karma track.

Your every third Advance, you get a point of Edge instead.

\paragraph{AWAKEN}
You become attuned to the deeper mysteries of the magical world. You can now spend Essence to power abilities and spells. This move, obviously, can only be taken once.

\paragraph{CITATION NEEDED}
When you research something, roll+Skilled. On 10+, you spend 1 day searching, and locate a useful detail about the topic of the research. On 7-9, you locate a useful detail, but (choose 1):

\begin{easylist}
    # you end up in a rabbit warren of information; spend 1 additional day digging through it
    # your search raises a flag in someone else’s systems (the GM determines whose)
    # the information is in hardcopy, and you need to go to it; spend 1 additional day on the search
\end{easylist}

\paragraph{FIRST AID}
When you try to keep a teammate from dying from their wounds, roll+Skilled. On 10+, you stabilize your teammate. On 7-9, you stabilize them, but (choose 1):
\begin{easylist}
    # you can’t move them to cover
    # you expose yourself to danger (take 2 damage)
    # their wounds force you to Stay Frosty
\end{easylist}
On a failure, your teammate cannot be saved.

\paragraph{GUT CHECK}
When you check off your 8th wound box, roll+Steady. On 10+, you stay on your feet, and if the damage you just received would take you beyond 8 boxes, ignore any excess. On 7-9, as above, but (choose 2):
\begin{easylist}
    # you take -2 ongoing to all moves
    # you’ll pass out in a few moments (you’ll have time for 1 or 2 moves, tops)
    # you’re making it worse; First Aid moves to help you take -1
\end{easylist}
On a failure, you collapse unconscious. If you were taken down by a weapon dealing stun damage, you are merely unconscious. Otherwise, you require first aid to stabilize you.

\paragraph{HIT THE BOOKS}
When you spend time training, practicing, or studying your abilities, you gain Prep. You gain 1 Prep for every 2 days spent in training or practice. When that training and preparation pays off, you can spend 1 Prep to get +1 to any roll. You can only spend 1 Prep per roll.

\paragraph{LAST CHANCE}
When you teeter on the brink of death and have no options left, permanently sacrifice at least one point of Edge, and roll+the amount sacrificed. On 10+, you’ll pull through somehow—you just won’t let go of life that easy. On 7-9, you will survive, but the GM will privately discuss with you what terrible bargain you agreed to in order to live. On a 6 or less, nothing can save you.

If you survive, your maximum Edge is reduced by 1 point (this can reduce your Edge to 0). Edge may be regained by Advancing as normal.

\paragraph{OVERWATCH}
When you’re providing cover for an ally and a threat appears, roll+Sharp. On 10+, your ally gets the drop on the threat. On 7-9, they’re alerted, and take +1 forward to their next move. On a miss, the threat gets the drop on your ally.

\paragraph{PULL STRINGS}
When you hit up a contact for info, items, or assistance, roll+Smooth. On 10+, the contact provides useful information (related to their own knowledge) or assistance. On 7-9, the contact provides information or assistance, but (choose 1):
\begin{easylist}
    # has to get back to you; wait 1 day
    # isn’t happy about it; take -1 forward to the next time you Pull Strings with this contact
    # requires a favor in return
\end{easylist}
If you fail, your contact doesn’t want to see you for a while, and will not return calls or meet with you for 1d6+1 days. Repeated failures of this move can permanently sever your relationship.

\paragraph{POP PILLS}
When you indulge in a drug, roll+Steady. On a 10+, you experience the effects as normal. On 7-9, you experience the effects but you got a weak batch, so the effects last half as long.

If you roll snake eyes when you pop pills, you become addicted to the drug. If you go 3 sessions without a hit, roll 2d6w. If you roll a 4 or higher, you are no longer addicted; otherwise, you’re still hooked. If you are an addict and roll snake eyes while popping pills, you overdose and take 8 Stun.


\paragraph{SUPPRESSIVE FIRE}
When you suppress an area to pin the enemy down down, roll+Hard and mark off 2 Ammo. On 10+, the targets are suppressed and cannot move or return fire. On 7-9, the only most of the targets are suppressed or they are only mostly suppressed.

\paragraph{TAKE A BULLET}
When you stand in defense of another, roll+Steady. On 10+, the attack hits you intead. On 7-9, the attack partly hits you (you take half damage, or half the attack’s effect, if non-damaging).



\section{Metatype Moves}
There are five primary metahuman types (or ``metatypes'') in the Sixth World: Human, Dwarf, Elf, Ork and Troll, each with their own unique moves. When you choose your metatype, you get a passive trait and choose one of it's moves.

\subsection{Human}
Humans can choose from the following moves:
\paragraph{Just Lucky:} you start with an extra point of Edge.
\paragraph{Privilege:} when interacting with humans, take +1 when you roll+Smooth.

\subsection{Dwarf}
All dwarves have natural low-light vision and can choose from the following moves:
\paragraph{Never Sick:} you are immune to disease and poisons.
\paragraph{Tonight We Drink:} if you’re drinking with someone, you may manipulate someone using Steady instead of Smooth.

\subsection{Elf}
All elves have natural low-light vision and can choose from the following moves:
\paragraph{Ethereal:} take +1 forward to persuade or seduce someone.
\paragraph{Uncanny Grace:} once per fight, when you take damage, roll+Sharp. On 10+, reduce damage by half. On 7-9, reduce damage, but take -2 forward.

\subsection{Ork}
All orks have natural low-light vision and can choose from the following moves:

\paragraph{Hard bastard:} take +1 forward to gut checks.
\paragraph{Fearless:} take +1 forward to stay frosty in the face of fear.
\paragraph{Streetfighter:} the first time you attack an enemy with a non-lethal weapon (fists, feet, batons, etc), you are boosted.

\subsection{Troll}
All trolls have natural thermographic vision and can choose from the following moves:

% Thermographic Vision: when you Check the Situation, you may ask one additional question from the list.
\paragraph{Dermal Bone Plating:} you have +1 armor. (idk, give this by default?)
\paragraph{You’ll Just Make It Angry:} you gain 1 additional wound box.
\paragraph{Juggernaut:} your fists should be licensed weapons. You deal 1d6 lethal damage in unarmed combat.


\section{Multiclassing}

You can choose moves freely from other archetypes, subject to the following two restrictions:

\begin{easylist}
    # You may choose no more than 3 moves from another archetype.
    # If your character is a non-magical archetype, they may not select moves that require Essence to be spent. They may select moves in which Essence expenditure is optional, however (although those usually don’t have much benefit without it).
\end{easylist}
