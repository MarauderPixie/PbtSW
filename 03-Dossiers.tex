\chapter{Dossiers} \label{dossiers}

Prinzipiell das Abenteuer \textit{Das Geheimnis der Hohepriesterin} aus dem Abenteuerband \textit{Im Bann der Karten}: Ein Johnson hat Probleme mit seinem Radiowecker und will, dass die Runner sich drum kümmern. Dabei kommen die Runner mit dem Tarot der sechsten Welt in Kontakt, wenn auch zunächst ohne (großen) Magiebezug. Im Verlauf des Runs sollte ihnen der unheimliche Realitätsbezug der Abbildungen auf den Karten dennoch mehr als verdächtig vorkommen.

\section{Interlude}

Wie immer erstmal eine Art Übergangsrunde: Nach dem letzten Run und ihrer (erneuten) Rückkehr aus Vancouver gibt es für die Runner einiges an Hausmeisterei zu erledigen. Mit dem unerwarteten Geldregen der letzten Wochen ist ja auch so einiges an Upgrades drin! Kontakte wollen gepflegt, Probleme gelöst und Erfolge erreicht werden.

\subsection{Rude}

Durch den kritischen Glitch im Zug sind Rudes Augen, gelinde gesagt, im Arsch. Beste Gelegenheit, ein wenig aufzurüsten! Auch: Brand-Wechsel von Shiawase zu Evo.

\textbox{}{
    {\rowcolors{1}{gray!30!white}{}
        \begin{tabular}{m{0.7\textwidth} m{0.15\textwidth}}
            {\fontspec{Njord.otf}{Feature}} & \lcr{r}{\fontspec{Njord.otf}{Kosten}} \\
            Cyberaugen, Stufe 4 & \lcr{r}{10.000} \\
            Troll Tax & \lcr{r}{x1.1} \\
            \hline
            Smartlink & \lcr{r}{4.000} \\
            Sichtverbesserung & \lcr{r}{4.000} \\
            Blitzkompensation & \lcr{r}{1.000} \\
            Infrarotsicht & \lcr{r}{2.000} \\
            Restlichtverstärkung & \lcr{r}{1.500} \\
            Ultraschallsensor & \lcr{r}{3.600} \\
            \hline
            Alphaware & \lcr{r}{x1.2} \\
            \textbf{Gesamt:} & \lcr{r}{33.600}
        \end{tabular}
    }
}

Ok, da ist ein Rechenfehler drin, aber der Deal ist eh wie folgt: Das Superangebot vom Evo-Shop bietet eine schamanistische Rundumversorgung für glate 34.000\nuyen an, anstatt für einen 10\%igen Aufpreis.

Sobald Rude dann im Center ankommt, wird er in einen ``Behandlungsraum'' geführt und von einem Mitarbeiter begrüßt.
    
\textbox{Sag's Ihnen ins Gesicht}{``Hey Rude. Darf ich dich Rude nennen? Cool, ich bin Jesse und Bär ist mein Totem. Also. Das ist nicht dein erstes Rodeo, wie ich sehe. Manches von dem, was wir dir sagen, kennst du also schon, aber wir sind trotzdem verpflichtet, es dir zu sagen. Du weißt schon, Paragraphenreiter und Versicherungsansprüche und der ganze Spaß. Ich werde dir gleich meine Hand auf die Schulter legen - das ist notwendig, damit mein Freund Bär dich stärken kann. Oder "um einen Kanal präventiv-regenerativen Manas auf deine Essenz zu richten", wie die Eierköpfe in ihren Elfenbeintürmen es sagen würden. Wenn Bär und ich einen guten Tag haben, dürftest du exakt nichts vom Eingriff mitbekommen, aber selbst an nicht so guten kaum mehr als ein leichtes Kribbeln spüren. Also gut, hast du noch irgendwelche Fragen oder wollen wir anfangen?''}


\subsection{Frostburn}

Frosty besorgt sich eine AR-fähige Brille (nicht gerade unauffällig $\rightarrow$ der Russe wird sie drauf ansprechen) und geht mal wieder Harrison besuchen. Für Harrison klaue ich ganz stumpf bei Mr. Mercer's Pumat Sol; als Frosty ankommt, sind 3 Harrisons im Laden und ein vierter (``Harrison Prime'') kommt erst dann dazu. Neben einigem Schnack und ein bisschen Trödelei - Frosty kauft sich zwei Schriftrollen; \textit{Gerät analysieren} und \textit{Stille} - vermittelt er sie auch an den nächsten Johnson der Gruppe: \textbf{George Blackburn}, ein freier Berater, der hauptsächlich für Renraku tätig ist und die Runner in einem Teppan-Yaki Restaurant in der Arkologie treffen möchte.


\subsection{Tusk}

    - Meister Arvid stellt einen Gegner vor: 
    Wapeka "Skillful" Becerra,
    Ini 16, 
    NK 13, 
    VT 11 (11), 
    Stab (4B, 10 AT), 
    Athletik 12, 
    K12
    
    "So zu tun, als verstünde man Lethani, ist unmöglich. Lethani ist wie Schwimmen: wenn du es kannst, ist es offensichtlich für jeden der zuschaut. Ein Weg, aber kein Pfad. Es ist sowohl der Weg, als auch das Wissen darüber. Und auch wenn es das Wissen ist, so kann es doch nicht verstanden werden."


\subsection{Yu}
    AAAHHHH!!!
    
    
    
    
\section{Kleinvieh ist doch Mist...}

    Johnson:
    - Teppan-Yaki-Restaurant \textit{Goshi} - bestellt wird per AR Interface (?)
    - George Blackburn
    - spricht japanisch: "Keine Sorge wegen der Bediensteten hier; ich habe sichergestellt, dass diese ausschließlich japanisch, das heißt, kein Englisch sprechen." Tatsächlich gehen die Bediensteten einfach ihrer Arbeit nach, ohne euch überhaupt zu beachten. Er sagt etwas auf japanisch zum Koch und erst jetzt verraten die Geischter der Leute so etwas wie Beachtung, dass überhaupt andere Leute da sind. Während der Johnson zu ihm spricht, schaut der Koch ihn an, blickt dann kurz jeden von euch an, während er unbeirrt weiterarbeitet und nickt dann, als Johnson aufhört zu sprechen. Johnson selbst scheint zufrieden: "Nun denn, kommen wir zum Geschäftlichen."
    
% Ich sollte einbauen, dass die Bilder vom Alpha, die Yu geleaked hat, nicht so bombastisch einschlagen, wie man denken könnte; siehe Kommentar in "Loose-Ends"

\subsection{Wrap up}


\section{Szene 2: Kropotklub}

\textbf{Praktisch 1:1 aus Szene 2 (Die Trotzkiste) in ``Im Bann der Karten'' entnommen.}

Pablo ``die Vogelscheuche'' Santiago
George Blackburn

% Name: The Iron Dome? Trotzkiste? Iron Trotzki?


- beim Betreten des Clubs: entweder \skill{Menschenkenntnis} oder \skill{Wahrnehmung} - was immer höher ist - um einen Eindruck vom Klientel zu bekommen: 
- (1) erstaunlich viele Konzerner
- (2) viele \textit{echte} Barrens-Bewohner
- (3) bla.

\probe{Erinnerung/Gedächtnisprobe: Gangwissen}{2} (?) Vory-Tags in Touristville und die auf dem Hof und in der näheren Umgebung zu sehenden Kerle in Anzügen machen klar, \textbf{wen} man sich hier zum Feind machen kann


\section{Szene 3: Spookeys}

Ebenfalls quasi 1:1 übernommen; die "Textbausteine" von Kira und Kollegen wurden durch \textit{Pulp Fiction} Zitate ersetzt.

\vspace{1em}
\textbf{Wrapup}
\vspace{0.5em}

Die Runner haben Frosty zu Neuron geschickt, um den Wecker untersuchen zu lassen - dabei kommt \textit{etwas} raus, aber nicht viel. Außerdem soll sie (am nächsten Tag) bei Harrison Infos zu (possibly magischem) Tarot einholen.

Der Strohmann, Pablo, wurde von Kira erschossen (und verblutet gerade). Kira haben die Runner zerlegt, Tusk ``Mad Mickey'' bewusstlos geschlagen, aber der Fahrer Jimmy liegt schwer verletzt, aber bei Bewusstsein, auf dem Parkplatz vor dem \textit{Spookeys}. Von ihm können die Runner ggf. erfahren, was es sonst von Kira gegeben hätte. Abgesehen davon hat Rude vmtl. einiges abbekommen.

Die Kiste haben die Runner sichergestellt und mit den Karten in Verbindung gebracht, außerdem hat Yu ein Foto von Kira gemacht und die Video-Feeds des Eingangsbereichs (sowie angrenzender Räume) von der letzten Woche im \textit{Kropotklub} von Zolotoy gekauft.

Derzeit stehen folgende Fragen im Raum:

\begin{easylist}
    # Wird ``der Clubbetreiber'' (also zolotoy) erpresst? Was hat es sonst mit den Karten im Club auf sich? Yu glaubt, Zolotoy sei sich ihrer nicht bewusst.
    # Was ist mit den Münzen? Gemeint ist das (wiederkehrende) Motiv auf den Karten, u.a. auf der "Sieben der Münzen" und "Neun der Münzen"
    # Ist eine der beteiligten Frauen die Affäre vom Johnson? Christophs Verdacht war Kira, den ich quasi ungewollt zerstreut habe, aber evtl. kommt der Gedanke bei Nathalie Brook wieder auf.
\end{easylist}

\section{Szene 4: Let's talk}

Wir beginnen auf dem Parkplatz.

Die Kiste hat ein Tastaturmagschloss der Stufe 3 (\probe{Mechanik}{6, 1 Min.})


Treffen im \textit{Sos Caife} im \textit{Emerald City Center}, Downtown, 

ReuterCore im Glas-Donut (eigtl. \textit{Blue Maple}), Bellevue.

\subsection{Sos Caife}

\textbf{Wiedereinstieg für Sitzung 4.} Nathalie fängt an mit:

\gesicht{
    Also gut, diese KI macht mich durchaus neugierig, aber ich will, dass Sie folgendes verstehen: Ich möchte schon seit langem in einem kreativen Bereich arbeiten, und auch ich bisher zwar für die Matrixsicherheit bei ReuterCore zuständig bin, ist dieser Job die beste und vielleicht sogar einzige Chance dazu. Ich kann euch also mit allerhand Informationen zu den Büros versorgen, aber ich werde nichts unternehmen, was mich diesen Job kosten könnte oder die Matrixabteilung unseres Mutterkonzerns einschaltet - bevor die Konsolen-Sherrifs von S-K anfangen, meine Arbeit zu durchleuchten, bin ich dann doch lieber arbbeitslos.
}

Während die Runner im Cafe sitzen, bekommt Rude eine Nachricht von Hez: \textit{``Schlechte Nachrichten, mein Großer. Wir müssen reden.''}. Wenn Rude nachhakt, kommt nur sowas wie: \textit{``Erzähl ich dir dann, du findest mich im HQ.''}

\subsubsection{Security Briefing}

Nathalie erklärt die drei Zugänge, die Art der Personen, die ein- und ausgehen, die Matrixsicherheit und macht Vorschläge zum Vorgehen. \textbf{FRAGE: Bleiben die anderen, wo sie sind?} Je nachdem, werden Proben auf Wissensfertigkeiten erst nach dem Gespräch fällig und Nachfragen sind nicht mehr Möglich.

\begin{easylist}
    # Bellevue, daher: \probe{Sicherheitssysteme}{2} - Response-Times in \textit{Bellevue}, nämlich 2-7 Minuten
    # ReuterCore befindet sich im 12. Stockwerk
    # \textit{``Ich bezweifle, dass den Wachleuten jeder Mitarbeiter bekannt ist - es sind mehr als ein Dutzend verschiedene Firmen dort ansässig. Wenn Ihr euch nicht gerade wie absolute Klischee-Einbrecher verhaltet, dürftet ihr also kaum auffallen.''}
    # MAD Scanner, Stufe 4: WP von 4 $\rightarrow$ 1 Erfolg reicht, um versteckte (Metall-)Waffen zu entdecken (und zu melden)
    # 4 Wachleute nachts, 5 tagsüber - davon immer 2 auf Patroullie im Gebäude.
\end{easylist}

\subsection{ReuterCore}

Teamworkproben \skill{Einfluss} vs. \skill{Wahrnehmung} des Kamerawachmanns - bei Misserfolg wird die Patroullie geschickt. -4 Würfel bei unpassender Kleidung oder auffälligem Verhalten.

Sollte eine erste \textbf{Streife} gerufen werden, braucht diese 9 Minuten, um anzukommen. Ein \textbf{HTR Team} braucht 5 Minuten.

Unbemerkt an Bewegungssensoren vorbeizukommen, erfordert \probe{Heimlichkeit}{3}. Gelingt das, muss der Wachmann \probe{Wahrnehmung}{vs. Hemlichkeit} mit -2 Würfeln ablegen, gelingt es nicht und die Runner wollen unentdeckt sein, dann mit +2 Würfeln.

\subsubsection{What happened?}


\subsection{Wrapu}


\clearpage
\section{Hintergrund}

\subsection{Protagonisten}

Falls da Namensänderungen nötig sind, zB:

\begin{easylist}
    # Herrmann Moltke - \textbf{George Blackburn}
    # Piet ``Die Vogelscheuche'' Saarbock - \textbf{Pablo ``Strawman'' Santiago}
    # \textbf{Matwej ``Zolotoy'' Nowikow}
    # \textbf{Nathalie Brook}
\end{easylist}



\subsection{Der Wecker}

\subparagraph{pre-Club:} AR-Programm zum Anzeigen dieser Karten. \probe{Elektronik(Software)}{2} Der Code wurde erweitert und das Programm kann die WiFi-Funktion des Weckers nutzen, um entsprechende Dateien zu empfangen.

\subparagraph{pre-Piet:} Die Karten bestehen aus übertrieben komplexerem Code als es einfache Bilddateien sollten; es ist abolut unklar, was dieser Code tut, aber es befinden sich auf jeden Fall viele Redundanzen darin.

\subparagraph{pre-Nathalie:} neben den komplexen Code-Teilen befindet sich außerdem ein Programm namens \textit{``Bring Her Home''} auf dem Wecker - was vorher aber sicher nicht da war. Was es tut, bleibt zunächst unklar.

\subparagraph{pre-ReuterCore:} Das BHH-Programm fügt alle Code Fragmente zusammen (Anm. 20.12.: Details noch nachgucken)