\chapter{Character Creation}

\epigraph{\textit{You wanna run the shadows? Then listen, chummer, learn everything you can, cuz ignorance will kill you faster than a fireball.}}{-- Ess El El, Snake Shaman}

Creating a character is a multi-step process (don’t worry, though, it’s pretty easy). The overall process is described here; more detail is provided in each Archetype’s dossier. You’ll record the details you create on the dossier page or the supplemental “extra info” page.

\begin{enumerate}
    \item \textbf{Choose your archetype} \\
    There are 10 Archetypes to choose from: Adept, Face, Ex-Cop, Decker, Mage, Mercenary, Rigger, Shaman, Street Doc, and Street Samurai. You can learn more about them in the Dossiers that follow.
    
    \item \textbf{Choose your Metatype and Moves} \\
    There are 5 metatypes: Human, Dwarf, Elf, Ork, and Troll. Each metatype offers a choice of Metatype Moves.
    
    \item \textbf{Choose your Looks} \\
    Each character archetype will present options for look; you are free to make up your own as well.
    
    \item \textbf{Choose your Name and Street Name} \\
    Pick a real name and street name. You may use the lists provided in the GM Resources, or create your own.
    
    \item \textbf{Assign your Stats} \\
    You have an array of points to distribute among your stats: +2, +1, +1, 0, -1. Assign each of these freely to one of your \refname{stats}: \textit{Sharp, Hard, Steady, Smooth, Skilled}
    
    \item \textbf{Choose Equipment} \\
    Each archetype will present various weapon, spell, cyberware, and equipment options. Choose one item from each list (unless the list indicates that you may choose more than one item). Some choices, particularly cyberware, are optional.
    
    \item \textbf{Determine your Essence and Edge} \\
    Your starting Essence is equal to 6 – the Essence cost of any cyberware you have installed or choose to install. Your starting Edge is one.
    
    \item \textbf{Choose Contacts} \\
    Everybody knows somebody. You will be presented with a list of potential contacts your character might know as a result of their experiences both before and after they became shadowrunners.
    
    \item \textbf{Establish Debts and Favors} \\
    It's dangerous in the shadows and nobody survives there for long completely on their own. Luckily, chummers help each other out - albeit not out of pure altruism, of course. A runner might live a life of freedom when compared to your average wage slave but that doesn't mean there ain't no strings attached. Take a look at the next section for more details.
    
    \item \textbf{Starting Moves} \\
    Your character knows all the Core and Secondary Moves. You character also knows one or more of his or her Archetype moves. If you are given an option to choose additional moves, check off the box next to them on the character sheet.
    
    \item \textbf{Advancement} \\
    Each time you fail a roll - that is, you roll a 6 or less - you receive Karma. When you mark 5 Karma and you have downtime, you can make the Advance move.
\end{enumerate}


% \subsection{Debts \& Favors}
In your life before and after becoming a shadowrunner, you’ve worked with a lot of people, and ended up owing, or being owed, by them. These relationships include at least one of your fellow shadowrunners, and are expresse d in the form of debts and favors. When you are instructed to create your debts and favors with fellow runners, you’ll see a list of sample statements to help you create them. You don’t have to use these; they’re simply suggestions.
    
To create a debt or favor, place the name of one of the other characters in the blank space in one of the statements presented. You can place the same name more than once (that is, in more than one sentence), but you must establish at least one debt or favor to start with. Collectively, debts and favors are known as Bond. Later, during play, you may end up resolving a bond with someone. If you do, both of you receive Karma.



\section{Dossiers} \label{Dossiers}

Each character archetype in Sixth World is described by a Dossier. You might think of these as what in other games would be their class: each archetype has unique moves and starts with different equipment. 


\clearpage
\subsection{The Adept}
\textbf{The Adept} is a magic-user whose power is focused inward, unlocking their full physical potential. Realized in the form of performance, speed, and endurance at or exceeding the peak of human capability, mastery of martial combat, and total control of self, the adept’s inner calm and perfected body are the envy of many.

\subsubsection{CREATING YOUR ADEPT}
\begin{enumerate}
    \item Follow through general steps 2 to 6. Some suggestions for Adept looks: \textit{wise eyes, wary eyes, glowing eyes; no hair, cropped hair, long braid; clean skin, tattooed skin, hard skin; perfect body, heavy body, lithe body}
    
    \item \textbf{Record your equipment} \\
    Basic Equipment: commlink \\
    Armor (choose 1): leather armor, arcane armor \\
    Weapons (choose 2): paired Ares Predators, katana, bo staff, paired combat knives, compound bow
    
    \item \textbf{Determine your Essence and Edge} \\
    Essence: 6 – total cost of all cyberware implants \\
    Edge: 1
    
    \item \textbf{Choose 2 Contacts} \\
    Temple master, gunsmith, underground fight club organizer, tea shop owner, yakuza soldier, talismonger
    
    \item \textbf{Establish Debts and Favors} \\
    Place one of your fellow runners’ names in at least one of the blanks below:
        \begin{easylist}
            ## If \_\_\_\_\_\_\_\_\_\_\_\_\_\_\_ hadn’t been there, I’d be dead right now.
            ## One day, I’ll make it up to \_\_\_\_\_\_\_\_\_\_\_\_\_\_\_ for letting that suspect walk.
            ## I let \_\_\_\_\_\_\_\_\_\_\_\_\_\_\_ skate on a serious charge once. 
            ## Letting \_\_\_\_\_\_\_\_\_\_\_\_\_\_\_ see that evidence earned me a formal reprimand.
        \end{easylist}
    
    \item \textbf{Starting Funds} \\
    You start play with 3d6 x 250\nuyen immediately available.
    
    \item \textbf{Starting Moves} \\
    You know all the Core and Secondary Moves. You also know the Enhanced Ability move, and one other Adept move.
\end{enumerate}


\subsubsection{ADEPT MOVES}
\paragraph{Enhanced Ability -} when you spend uninterrupted time (an hour or so) in quiet contemplation of your abilities, you gain +1 ongoing to one Stat of your choice, as long as you’re conscious or until you meditate again.

\paragraph{Gunfighter -} when you \move{Rock \& Roll} while wielding one or two handguns, you may spend 1 essence. In addition to the usual results of \move{Rock \& Roll}, choose 1:
    \begin{easylist}
        # you maneuver quickly and precisely, giving yourself the best shots possible while minimizing your opponents’ advantage; take +1 forward to \move{Rock \& Roll}
        # one of your targets is suppressed; take +1 forward to \move{Null Sweat}
        # you grab an opponent and use them as a human shield; split any damage taken between you and the enemy
        # you physically strike an enemy within melee range with your weapon, dealing 1d6 stun
    \end{easylist}

\paragraph{Killing Hands -} when you deal damage while unarmed, you can chose to deal lethal damage instead of stun. In addition, you can spend 1 essence to roll damage twice and take the better value.

\paragraph{Danger Sense -} when you open your mind to the world of subtle mundane and magical information in your environment, spend 1 essence and \roll{Sharp}. On 10+, you cannot be surprised. On 7-9, take +1 to \move{Null Sweat}.

\paragraph{The Sight -} when you take time to study an enemy, \roll{Sharp}. On 10+, take +1 forward or take +2 damage forward to your next attack. On 7-9, take +1 forward.

\paragraph{Astral Projection -} when you project your spirit into astral space, spend 1 Essence and \roll{Steady}. On 10+, you project successfully. On 7-9, you project, but your connection is tenuous; take -1 ongoing while in astral space.

\paragraph{Mystic Armor -} you gain +2 armor when naked or in normal clothes, or +1 armor when wearing mundane armor.

\paragraph{Traceless Walk -} your footsteps are silent and leave no trace, and you can walk on soft or brittle surface like snow, sand or broken glass without sinking. Whenever you’re trying to be sneaky and sound is of importance, you’re boosted.



\clearpage
\subsection{The Ex-Cop}
\textbf{The Ex-Cop} comes from Lone Star, Knight Errant, the military police, or any one of many law enforcement agencies in the confused landscape of the late 21st century. Possessed of a keen investigative mind, brutally effective combat skills, experience with the best and worst of humanity and connections deep into "the system", the ex-cop is a valuable asset.

\subsubsection{CREATING YOUR EX-COP}
\begin{enumerate}
    \item Follow through general steps 2 to 6. Some suggestions for Ex-Cop looks: \textit{cold eyes, tired eyes, wary eyes; close cropped hair, shaggy hair, bald; cheap suit, street clothes, hawaiian shirt; heavy body, fit body, injured body}
    
    \item \textbf{Choose your Equipment} \\
    Basic Equipment: commlink \\
    In addition, choose from the lists below: \\
        Armor: armor vest, form-fitting armor \\
        Service Pistol: Ruger Super Warhawk, Colt Manhunter \\
        Additional Weapon: HK 227, Remington 990 \\
        Installed Cyberware (optional): datajack (1 essence), cybereye with 2 enhancements (1 essence), skillwires (2 essence)
    
    \item \textbf{Determine your Essence and Edge} \\
    Essence: 6 – total cost of all cyberware implants \\
    Edge: 1
    
    \item \textbf{Choose 3 Contacts} \\
    Confidential informant (CI), precinct secretary, gang leader, prosecutor, journalist, former partner, defense attorney
    
    \item \textbf{Establish Debts and Favors} \\
    Place one of your fellow runners’ names in at least one of the blanks below:
        \begin{easylist}
            ## If \_\_\_\_\_\_\_\_\_\_\_\_\_\_\_ hadn’t been there, I’d be dead right now.
            ## One day, I’ll make it up to \_\_\_\_\_\_\_\_\_\_\_\_\_\_\_ for letting that suspect walk.
            ## I let \_\_\_\_\_\_\_\_\_\_\_\_\_\_\_ skate on a serious charge once.
            ## Letting \_\_\_\_\_\_\_\_\_\_\_\_\_\_\_ see that evidence earned me a formal reprimand.
        \end{easylist}
    
    \item \textbf{Starting Funds} \\
    You start play with 3d6 x 250¥ immediately available.
    
    \item \textbf{Starting Moves} \\
    You know all the Core and Secondary Moves.
    You know the Gumshoe move, and one other Cop move.
\end{enumerate}

\subsubsection{EX-COP MOVES}
\paragraph{Gumshoe -} when you examine the scene of an event, or interrogate someone about an event, \roll{Sharp}. On 10+, pick two of the following to learn (relevant to what you’re investigating). On 7-9, pick one:
    \begin{easylist}
        # \textbf{Scene:} when the events happened; whether magic was involved; how many individuals were involved; if this is the primary scene of the event
        # \textbf{Person:} if they’re connected to the event; whether they’re hiding something; what they stood to lose or gain; a useful personal detail (e.g, a tic, handedness, etc.)
    \end{easylist}

\paragraph{Work the System -} when you use your ex-LEO status to get help, \roll{Smooth}. On 10+, you have an old pal jam somebody up or cut them a break. On 7-9, you get the desired result, but (choose 1):
    \begin{easylist}
        # the person knows who helped or hindered them
        # your buddy got in trouble
        # your name got mentioned to the wrong ears        
    \end{easylist}

\paragraph{Takedown -} when you take control of a person physically, \roll{Hard}. On 10+, they are under your complete control. On 7-9, you gain control of them, but either you or your target must take 2 damage.

\paragraph{Interrogation -} when you attempt to make someone sweat, you may \roll{Skilled} instead of +Hard.

\paragraph{The Feds -} you have a connection in federal law enforcement; \roll{Smooth}. On 10+, pick 2. On 7-9, pick 1.
    \begin{easylist}
        # You get a tip-off on a big operation so you can steer clear
        # You gain interesting and useful information about your current run
        # You get access to federal data on an individual
        # You are listed as a “consultant” on a case        
    \end{easylist}

\paragraph{Doorkicker -} when you lead the team in an assault on the enemy, \roll{Steady}. On 10+, designate up to 3 enemies who are surprised. On 7-9, designate up to 2 enemies.



\clearpage
\subsection{The Face}
\epigraph{Living in the shadows isn’t all about big guns, major mojo, and dancing the electron two-step. Sometimes a more personal approach is needed, and that’s where the Face comes in. As the public persona of the shadowrunning team, he uses his charm and charisma to negotiate with Mr. Johnson, wine and dine information sources, and talk his way out of tricky situations where blazing guns aren’t the smart way to go.}{Shadowrun - 20th Anniversary Core Rulebook}

\subsubsection{CREATING YOUR FACE}
\begin{enumerate}
    \item Follow through general steps 2 to 6. Some suggestions for Face looks: \textit{wise eyes, jeweled eyes, laughing eyes; normal skin, perfect skin, synthetic skin; great smile, smoky stare, rugged good looks, regal bearing; fit body, compact body, androgynous body}
    
    \item \textbf{Choose your Equipment} \\
    Basic Equipment: commlink, fashionable clothing \\
    In addition, choose from the lists below: \\
    Armor: armorweave clothing, form fitting armor, light armor jacket \\
    Weapon: Colt L36, Beretta 101T, stun baton, taser \\
    Cyberware: datajack (1 essence), cybereyes with 2 enhancements (1 essence), hold-out cybergun (2 essence), voice modulator (1 essence)
    
    \item \textbf{Determine your Essence and Edge} \\
    Essence: 6 – total cost of all cyberware implants \\
    Edge: 1
    
    \item \textbf{Choose 4 Contacts} \\
    Club owner, Yakuza boss, car dealer, journalist, senator’s aide, money launderer, mafia capo, arms dealer, wealthy socialite
    
    \item \textbf{Establish Debts and Favors} \\
    Place one of your fellow runners’ names in at least one of the blanks below:
        \begin{easylist}
            ## \_\_\_\_\_\_\_\_\_\_\_\_\_\_\_ always answers my calls.
            ## \_\_\_\_\_\_\_\_\_\_\_\_\_\_\_ knows I screwed over their friend, and has never said anything about it.
            ## \_\_\_\_\_\_\_\_\_\_\_\_\_\_\_ hung me out to dry.
            ## I helped \_\_\_\_\_\_\_\_\_\_\_\_\_\_\_ lay low after that nasty business with Renraku.
        \end{easylist}
    
    \item \textbf{Starting Funds} \\
    You start play with 3d6 x 350¥ immediately available.
    
    \item \textbf{Starting Moves} \\
    You know all the Core and Secondary Moves. You know the Razor Insight move, and one other Face move.
\end{enumerate}

\subsubsection{FACE MOVES}
\paragraph{Razor Insight -} when you have a casual conversation with someone, \roll{Sharp}. On 10+, you learn three of the following. On 7-9, you learn 2.
    \begin{easylist}
        # Something they love        
        # Something they fear        
        # Something they need        
        # Something they lost        
        # Something they took
    \end{easylist}
If you use this information when fast talking, manipulating, or making them sweat, you are boosted.

\paragraph{Fast Talk -} when you need to convince somebody of something fast, \roll{Smooth}. On 10+, your quick thinking gets you through. On 7-9, they’re convinced, but (choose 1)
    \begin{easylist}
        # they check up on your story later
        # they get in serious trouble for believing you
        # one of your contacts somehow ends up involved…in a bad way   
    \end{easylist}
        
\paragraph{Build a Legend -} when you create a false identity, spend 1 day working on it and \roll{Skilled}. On 10+, your legend is solid and will hold up to any scrutiny. On 7-9, it holds up for now, but (choose 1):
    \begin{easylist}
        # you've only got 1d4+Skilled days before its blown
        # you run into someone who knows you…as someone else.
        # you have to do something unpleasant to maintain your cover.
    \end{easylist}

\paragraph{I Know A Guy -} when you need an illegal good or service, \roll{Smooth}. On 10+, you know someone who can get it for you immediately, and discreetly. On 7-9, they can get it, but (choose 1):
    \begin{easylist}
        # it takes 1 additional day
        # it costs twice as much as predicted
        # your fence has to drop your name to get it
    \end{easylist}

\paragraph{Honeyed Words -} when you make someone sweat, you may \roll{Smooth} instead of Hard.

\paragraph{Irresistible -} even if you anger, insult, or otherwise tick off a contact, they just can’t stay mad at you. They only avoid you for half as long as normal.



\clearpage
\subsection{The Decker}
\textbf{The Decker} is the master of the worldwide virtual reality network of the Matrix. Able to bend the Matrix their will, the Hacker is a critical member of the team. From finding crucial data on targets, to locating floor-plans of facilities, to shutting down security systems and sabotaging response efforts, the decker’s value is indisputable.

\subsubsection{CREATING YOUR DECKER}
\begin{enumerate}
    \item Follow through general steps 2 to 6. Some suggestions for Decker looks: \textit{strange eyes, glasses, unfocused eyes; no hair, unkempt hair, mohawk, ponytail; pale skin, bad skin, tattooed skin; thin body, heavy body, compact body, flabby body}
    
    \item \textbf{Choose your Equipment} \\
    Basic Equipment: commlink, Fuchi Cyber-4 or Fuchi Cyber-7 \\
    Installed Cyberware: datajack (1 essence) \\
    In addition, choose from the lists below: \\
    Armor (choose 1): trenchcoat, light armor jacket \\
    Weapon (choose 1): Fichetti Needler, Ares Lightfire 70, Combat Axe, Remington 990 \\
    Cyberware: cybereyes with 2 enhancements (1 essence), synaptic hardening (2 essence)
    
    \item \textbf{Determine your Essence and Edge} \\
    Essence: 6 – total cost of all cyberware implants \\
    Edge: 1
    
    \item \textbf{Choose 2 Contacts} \\
    Electronics dealer, military decker, gang member, former professor, matrix guru, white hat, script kiddie, poker dealer, money launderer
    
    \item \textbf{Establish Debts and Favors} \\
    Place one of your fellow runners’ names in at least one of the blanks below:
        \begin{easylist}
            ## I did a run with \_\_\_\_\_\_\_\_\_\_\_\_\_\_\_ that went bad…because of me.
            ## If \_\_\_\_\_\_\_\_\_\_\_\_\_\_\_ hadn’t unplugged me, that IC would have fried my brain.
            ## I scrubbed \_\_\_\_\_\_\_\_\_\_\_\_\_\_\_’s arrest record; they’re pure as driven snow. For now.
            ## I don’t work for free. But \_\_\_\_\_\_\_\_\_\_\_\_\_\_\_ can be very convincing.
        \end{easylist}
    
    \item \textbf{Starting Funds} \\
    You start play with 3d6 x 150¥ immediately available.
    
    \item \textbf{Starting Moves} \\
    You know all the Core and Secondary Moves. You know the Born Digital and Sling Code moves.
\end{enumerate}

\subsubsection{DECKER MOVES}
\paragraph{Born Digital -} while in the Matrix, when you:
    \begin{easylist}
        # \move{Null Sweat}: add your deck’s Mask rating to the roll
        # Take damage: subtract your deck’s Hardening rating from the damage
        # \move{Rock \& Roll}: \roll{Skilled} instead of +Hard
    \end{easylist}

\paragraph{Sling Code -} when you hack a Matrix node or device, \roll{Sharp}. On 10+, choose 3. On 7-9, choose 2:
    \begin{easylist}
        # The node or device remains unaware of the intrusion
        # You leave no trace of your presence
        # You don’t trigger IC
        # You learn a useful detail about another node connected to this one
    \end{easylist}
Once in control of a node, you can issue commands appropriate to it.

\paragraph{Matrix Overwatch -} when you defend a device or node against a matrix attack, \roll{Steady}. On 10+, the attack is ineffective. On 7-9, halve the damage or duration of the attack’s effect.

\paragraph{IC Killer -} when you inflict damage to IC, inflict +1 damage.

\paragraph{Multitasker -} you can hack multiple systems or devices simultaneously. \roll{Steady}. On 10+, you suffer no penalties to hack two systems. On 7-9, take -1 ongoing to the second system.

\paragraph{Tracer -} when you would deal damage to an enemy decker in Matrix combat, you can instead forgo damage to plant a tracer tag on their avatar. This tracer is active for 1+Training days.



\clearpage
\subsection{The Mage}
\textbf{The Mage}’s magical talent is focused on the arcane art of spellcasting - employing esoteric formulas, incantations, and the precepts of magical theory to shape reality itself. If you want an Arcana artillery company, someone to cloak the entire team in magical invisibility, or someone to provide astral overwatch for the team, look to the mage.

\subsubsection{CREATING YOUR MAGE}
\begin{enumerate}
    \item Follow through general steps 2 to 6. Some suggestions for Mage Looks: \textit{blank eyes, unnatural eyes, piercing eyes; Long hair, bald, wild hair; robes, street clothes, dress clothes; thin body, weak body, muscular body}
    
    \item \textbf{Choose your Equipment and Spells} \\
    Choose from the lists below: \\
    Armor: trenchcoat, light armor jacket, armor charm \\
    Weapon: Beretta 101T, Ruger Super Warhawk, Staff
    
    In addition, Choose 3 of the following 5 spell categories: \\
    Combat, Detection, Illusion, Health, Manipulation
    
    You know 2 spells in one of your chosen categories, 1 in the other caategories each.
    
    \item \textbf{Determine your Essence and Edge} \\
    Essence: 6 – total cost of all cyberware implants \\
    Edge: 1
    
    \item \textbf{Choose 2 Contacts} \\
    Wage Mage, Corporate Exec, Fetishmonger, Paranormal Animal Expert, Bartender, Street Cop, Professor of Magical Theory
    
    \item \textbf{Establish Debts and Favors} \\
    Place one of your fellow runners’ names in at least one of the blanks below:
        \begin{easylist}
            ## I’d still be a wage mage today if \_\_\_\_\_\_\_\_\_\_\_\_\_\_\_ hadn’t made that call.
            ## Those gangers would have waxed me if \_\_\_\_\_\_\_\_\_\_\_\_\_\_\_ hadn’t happened along.
            ## I helped get rid of a curse. You believe that? A curse.
            ## I sucked up a manabolt for \_\_\_\_\_\_\_\_\_.
        \end{easylist}
    
    \item \textbf{Starting Funds} \\
    You start play with 3d6 x 250¥ immediately available.
    
    \item \textbf{Starting Moves} \\
    You know all the Core and Secondary Moves. You know the Cast a Spell, Center, and Counterspell moves.
\end{enumerate}

\subsubsection{MAGE MOVES}
\paragraph{Cast a Spell -} When you cast a spell, spend the required essence and roll. The stat you add depends on the type of spell:
    \begin{easylist}
        # Combat: \roll{Hard}
        # Detection: \roll{Sharp}
        # Illusion: \roll{Smooth}
        # Health: \roll{Skilled}
        # Manipulation: \roll{Steady}
    \end{easylist}        
On 10+, the spell is cast. On 7-9, the spell is cast, but (choose 1):
    \begin{easylist}
        # it causes drain; take 1 stun
        # it causes astral feedback; take -1 to the next spell you cast
        # you must expose yourself to danger or an attack to cast the spell
    \end{easylist}

\paragraph{Center -} when you take a moment to concentrate and restore yourself, regain 1d6 essence.

\paragraph{Spell Defense -} when you defend an ally from a magic spell, spend 1 Essence and \roll{Sharp}. On 10+, choose 2. On 7-9, choose 1:
    \begin{easylist}
        # halve the spell’s damage
        # halve the spell’s duration
        # locate the spell’s caster
        # deal 1d6 damage to the caster
    \end{easylist}

\paragraph{Astral Trace -} when you observe a magical effect for which you cannot determine the source, \roll{Sharp}. On a 10+, the GM answers three of the following. On 7-9, two:
    \begin{easylist}
        # In what direction does the source of this magic lie?
        # Approximately how far away is the source?
        # What is the general nature of the source?
        # How powerful is the source?
    \end{easylist}

\paragraph{Hermetic Library -} you have permission to access an extensive library of hermetic lore. When you or a teammate uses the Citation Needed move to research magical history or theory, the move is boosted.

\paragraph{Initiate -} when you hit the books, you may also spend Prep on:
    \begin{easylist}
        # reducing a spell’s Essence cost by 1 (to a minimum of 0)
        # boosting a Cast a Spell move
        # regain 1 Essence
    \end{easylist}


\clearpage
\subsection{The Mercenary}
\textbf{The Mercenary} served in one of the many military forces found in the Sixth World, doing time in conflicts large and small and brought from that solid tactical abilities and a respectable repertoire of combat talents. Hardened mentally and physically from years in service, the merc is highly skilled in combat and has the added benefit of leadership experience that can save the team’s bacon when things get hairy.

\subsubsection{CREATING A MERCENARY}
\begin{enumerate}
    \item Follow through general steps 2 to 6. Some suggestions for Mercenary looks: \textit{dead eyes, cold eyes, soft eyes; boonie hat, high ‘n tight, ponytail, fauxhawk; combat fatigues, street clothes, nice suit; scarred skin, tough skin, soft skin}
    
    \item \textbf{Choose your Equipment} \\
    Basic Equipment: commlink \\
    Installed Cyberware: bone lacing (2 essence) \\
    In addition, choose from the lists below: \\
    Armor: ballistic vest, armor jacket, combat armor \\
    Weapon (choose 3): Ares Predator, Browning Max Power, AK-97K, Ingram Smartgun, Colt M22A2, AK-97, tomahawk, combat knife
    
    \item \textbf{Determine your Essence and Edge} \\
    Essence: 6 – total cost of all cyberware implants \\ Edge: 1
    
    \item \textbf{Choose 2 Contacts} \\
    Former CO, Terrorist Cell Member, Arms Dealer, Veterans Clinic Doctor, Old War Buddy, Street Pharmacist, Therapist
    
    \item \textbf{Establish Debts and Favors} \\
    Place one of your fellow runners’ names in at least one of the blanks below:
        \begin{easylist}
            ## \_\_\_\_\_\_\_\_\_\_ dragged me out when shit went sideways.
            ## \_\_\_\_\_\_\_\_\_\_ backed my play even when nobody else would.
            ## It was not fun explaining to my CO what happened to those weapons \_\_\_\_\_\_\_\_\_ "borrowed".
            ## First time I saw \_\_\_\_\_\_\_\_, it was at the other end of my gun.
        \end{easylist}
    
    \item \textbf{Starting Funds} \\
    You start play with 3d6 x 150¥ immediately available.
    
    \item \textbf{Starting Moves} \\
    You know all the Core and Secondary Moves. You know the Go Tactical move and one other Mercenary move.
\end{enumerate}

\subsubsection{MERC MOVES}
\paragraph{Go Tactical -} when you Check the Situation during combat, \roll{Hard} instead of +Sharp. On a 10+, instead of asking the GM questions, you may instead choose to Hold 3. On a 7-9, you may choose to Hold 1. You can then spend that Hold 1-for-1 to grant a bonus to any ally at any point during the combat.

\paragraph{Deadeye -} when you attack a surprised or defenseless enemy in ranged combat, you can deal damage or, name your target and \roll{Hard}:
    \begin{easylist}
        # Head: on 10+, you deal your damage and they fall to the ground, stunned. 7-9: they fall to the ground, stunned.
        # Arms: on 10+, you deal your damage, and they drop whatever they’re holding. 7-9: they drop whatever they’re holding.
        # Legs: on 10+, you deal your damage, and they are slowed or immobilized. 7-9: they are slowed or immobilized.
    \end{easylist}

\paragraph{Veteran -} when you \move{Null Sweat}, you take +1.

\paragraph{Adapt and Overcome -} when you fail a move, instead of marking Karma you may \roll{Skilled}. On a 10+ you take +2 forward on your next move. On 7-9 you take +1 forward.

\paragraph{Contracts Available -} you have contacts with a mercenary force or guild. \roll{Smooth}. On 10+, they can pass you a contract worth 10,000¥. On 7-9, they can pass you a contract worth 5,000¥.

\paragraph{Field Trial -} when you use your military connections to acquire military- only equipment, \roll{Smooth}. On 10+, you’re able to borrow the equipment for 5 days. On 7-9, you borrow it, but (choose 1):
    \begin{easylist}
        # There’s an unscheduled inventory inspection before you can return it
        # You need to pony up a sizeable ``security deposit''
        # You got a hangar queen. The equipment requires 1 day of maintenance, or it will fail at a most inopportune moment.
    \end{easylist}

\paragraph{Inspiring -} when you roll a 10+ when you \move{Null Sweat}, one ally who saw you can take +1 forward to their next move.




\clearpage
\subsection{The Rigger}
\textbf{The Rigger} is a cybered-up, drek-hot driving machine. When a team needs transportation, recon, or a flying drone to blow the enemy into bloody rags, they turn to their rigger. Riggers have the capability to virtually \textit{become} any drone or vehicle by using their control rig, fusing their consciousness to the machine and operating it at its peak. Getting into and out of an op and providing a little robotic fire support is the rigger’s specialty.

\subsubsection{CREATING YOUR RIGGER}
\begin{enumerate}
    \item Follow through general steps 2 to 6. Some suggestions for Rigger looks: \textit{goggles, alert eyes, obvious cybereyes; kaiser helmet, cowboy hat, pirate bandana; biker clothes, flight suit, street clothes, punk clothes; heavy body, built body, lean body}
    
    \item \textbf{Choose you Equipment} \\
    Basic equipment: commlink, 1 drone, 1 vehicle \\
    Installed cyberware: control rig (2 essence) \\
    Choose from the lists below: \\
        Armor: ballistic vest, lined coat \\
        Weapon (choose 2): Enfield AS-7, Browning Max Power, Ares Predator, AK-97K, combat axe \\
        Cyberware: cyberarm with 1 enhancement (2 essence), cybereyes with 2 enhancements (1 essence)
    
    \item \textbf{Determine your Essence and Edge} \\
    Essence: 6 – total cost of all cyberware implants \\
    Edge: 1
    
    \item \textbf{Choose 2 Contacts} \\
    Chop shop worker, go ganger, fence, trucker, arms dealer, mechanic, bartender, cargo pilot, car thief
    
    \item \textbf{Establish Debts and Favors} \\
    Place one of your fellow runners’ names in at least one of the blanks below:
        \begin{easylist}
            ## \_\_\_\_\_\_\_\_\_\_\_\_ tipped me off to some sweet (and lucrative) courier runs.
            ## When I ended up in the slam for the Dynagene job, \_\_\_\_\_\_ bailed me out.
            ## I wrecked my favorite ride working with \_\_\_\_\_\_\_\_\_\_\_\_. Took months to fix it.
            ## \_\_\_\_\_\_\_\_\_ jammed me up for a goddamned percentage.
        \end{easylist}
    
    \item \textbf{Starting Funds} \\
    You start play with 3d6 x 400¥ immediately available.
    
    \item \textbf{Starting Moves} \\
    You know all the Core and Secondary Moves. You know the Jumped In move and one other Rigger move.
\end{enumerate}

\subsubsection{RIGGER MOVES}
\paragraph{Jumped In -} while jacked into a vehicle or drone you own, when you:
    \begin{easylist}
        # \move{Rock \& Roll} or \move{Null Sweat}: \roll{Skilled}
        # Check the Situation: add the vehicle or drone’s Sensor rating to the roll
        # Fail a move involving the vehicle or drone, mark off 1 Fuel.
        # Take an action not related to controlling the vehicle or drone, take -2.
    \end{easylist}

\paragraph{Autonomous Mode -} when you put a drone in autonomous mode, indicate which mode setting you want, and \roll{Skilled}. On 10+, hold 2 to be spent on the drone’s moves. On 7-9, hold 1. Drone mode settings (and the rolls they use for moves) are:
    \begin{easylist}
        # Sentry: the drone can make the \move{Rock \& Roll} move; \roll{Tactical}
        # Recon: the drone can make the Check the Situation move; \roll{Sensor}
        # Evasion: the drone can make the \move{Null Sweat} move; \roll{Power}
    \end{easylist}

\paragraph{Split Personality -} when you launch a drone, \roll{Steady}. On 10+, you don’t take the normal -2 penalty to non-drone moves while controlling it. On 7-9, the penalty is reduced to -1.

\paragraph{Jury Rig -} when you have to make fast repairs to a vehicle or machine, \roll{Sharp}. On 10+, you get it running again and fast. On 7-9, you get it running, but (choose 1):
    \begin{easylist}
        # it will only run for 1d10 minutes
        # afterwards, it will be a total loss.
        # one of its qualities is reduced by 1, permanently
    \end{easylist}

\paragraph{Percussive Maintenance -} when you smack the hell out of a recalcitrant device, \roll{Hard}. On 10+, the device springs to life. On 7-9, the device works for only a moment, but you know what you need to do to fix it. Take +1 forward to Jury Rig.

\paragraph{Paint the Target -} when you point out a drone or vehicle’s weakness to your teammates, they take +1 forward to attacks against it.



\clearpage
\subsection{The Shaman}
\textbf{The Shaman} is a master of conjuring: summoning the spirits that dwell in the astral realm and compelling them to do the shaman’s bidding. The shaman’s spirits provide many services, from devastating combat abilities to protection from hostile intent to information and reconnaissance impossible for a mundane.

\subsubsection{CREATING A SHAMAN}
\begin{enumerate}
    \item Follow through general steps 2 to 6. Some suggestions for Shaman looks: \textit{heterochromic eyes, wise eyes, sunglasses; long hair, dreadlocks, shaved head; street clothes, anachronistic clothes, biker gear; wiry body, thin body, round body}
    
    \item \textbf{Choose your Totem} \\
    Choose a totem from the list, or make up one of your own.
    
    \item \textbf{Choose your Equipment} \\
    Choose from the lists below: \\
        Armor: Leather jacket, defensive charm, riot shield \\
        Weapon: Ruger Super Warhawk, Colt Manhunter, AK-97, combat axe, crossbow \\
        Spirits: choose 3 spirits from the gear section
    
    \item \textbf{Determine your Essence and Edge} \\
    Essence: 6 – total cost of all cyberware implants \\
    Edge: 1
    
    \item \textbf{Choose 2 Contacts} \\
    Wage mage, ork underground, gang thug, street cop, herbalist, university professor, diner owner, fetishmonger, art dealer, hedge wizard, houngan
    
    \item \textbf{Establish Debts and Favors} \\
    Place one of your fellow runners’ names in at least one of the blanks below:
        \begin{easylist}
            ## \_\_\_\_\_\_\_\_\_ had me in his sights, and let me live.
            ## \_\_\_\_\_\_\_\_\_ put their life on the line helping me battle a wild spirit.
            ## When \_\_\_\_\_\_\_\_ fell foul of that corp hit squad, I provided additional security.
            ## Getting the artifact \_\_\_\_\_\_\_\_\_ wanted wasn’t easy.
        \end{easylist}
    
    \item \textbf{Starting Funds} \\
    You start play with 3d6 x 150¥ immediately available.
    
    \item \textbf{Starting Moves} \\
    You know all the Core and Secondary Moves. You know the Conjure and Banish moves.
\end{enumerate}

\subsubsection{SHAMAN MOVES}
\paragraph{Conjure -} When you summon a spirit, spend at least 1 essence and roll. The stat you add to the roll depends on the spirit’s nature:
    \begin{easylist}
        # Destroyer: \roll{Hard}
        # Protector: \roll{Steady}
        # Watcher: \roll{Sharp}
        # Teacher: \roll{Skilled}
        # Seducer: \roll{Smooth}
    \end{easylist}

On 10+, the spirit is conjured and will perform a number of moves equal to the essence spent. On 7-9, the spirit is conjured, but (choose 1):
    \begin{easylist}
        # It can perform one fewer moves (you cannot choose this option if you spent only 1 essence)
        # It is draining; take 1 stun
        # You must expose yourself to danger or an attack
    \end{easylist}

When the spirit has used all of its moves, you regain the essence committed to the summoning. If the spirit is destroyed, you regain half the committed essence, round down.

On a failure, the spirit does not manifest, and the essence spent is lost. If you roll snake eyes, the spirit is summoned in an uncontrolled state, and the GM will control its actions until it is exhausted or banished.

\paragraph{Banish -} when you attempt to banish a spirit, \roll{Hard}. On 10+, you reduce the spirit’s available moves by 1. On 7-9, you reduce the spirit’s moves by 1, but it deals half its damage to you. If you reduce the spirit’s available moves to 0, it vanishes immediately.

\paragraph{Commune -} when you take a moment to mentally commune with your totem, you may gain its boons and flaws, or regain 1d6 essence.

\paragraph{Favored Spirit -} choose 1 spirit type (Watcher, Teacher, Protector, Destroyer, Seducer). This spirit type performs one free move.

\paragraph{Aura Mask -} you may conceal your magical nature. \roll{Skilled}. On 10+, you appear to be a mundane individual to anyone or anything that examines you. On 7-9, you appear mundane, but must spend 1 Essence to do so.

\paragraph{Spirit Master -} you may conjure multiple spirits simultaneously, dividing the commited Essence among them.



\clearpage
\subsection{The Street Doc}
\textbf{The Street Doc} brings medical expertise to the shadows, helping their team survive and recover from the inevitable injuries that they will incur in their particular line of work. Modern technology might make basic first aid a matter of a slap patch and a pain pill, but when you get caught by a frag grenade, basic first aid is not what you need. You need the Doc.

\subsubsection{CREATING YOUR STREET DOC}
\begin{enumerate}
    \item Follow through general steps 2 to 6. Some suggestions for Street Doc looks: \textit{clear eyes, old eyes, quick eyes; close cut hair, stylish hairdo, bandana; fit body, heavy body, compact body; business attire, street clothes, EMT jumpsuit}
    
    \item \textbf{Choose your Equipment} \\
    Basic equipment: commlink, MedKit with 6 supply \\
    In addition, choose from the lists below: \\
    Armor: ballistic vest, armor jacket \\
    Weapon: Narcoject rifle, Browning Max Power, HK227, stun baton, combat knife \\
    Cyberware: cyberarm with 2 enhancements (2 essence), skillwires (2 essence) \\
    
    \item \textbf{Determine your Essence and Edge} \\
    Essence: 6 – total cost of all cyberware implants \\
    Edge: 1
    
    \item \textbf{Choose 2 Contacts} \\
    ER doctor, morgue staffer, medical examiner, DocWagon driver, organlegger, black market organ dealer, blood bank worker, pharmacist
    
    \item \textbf{Establish Debts and Favors} \\
    Place one of your fellow runners’ names in at least one of the blanks below:
        \begin{easylist}
            ## \_\_\_\_\_\_\_\_\_\_\_ helped me get clean.
            ## \_\_\_\_\_\_\_\_\_\_\_ got their hands bloody helping me save a life.
            ## I arranged for \_\_\_\_\_\_ to receive a “mis-shipped” case of pharmaceuticals.
            ## I extracted information from a prisoner once for \_\_\_\_\_\_\_\_\_\_\_.
        \end{easylist}
    
    \item \textbf{Starting Funds} \\
    You start play with 3d6 x 400¥ immediately available.
    
    \item \textbf{Starting Moves} \\
    You know all the Core and Secondary Moves. You know the Combat Medic and Stay With Me moves.
\end{enumerate}

\subsubsection{STREET DOC MOVES}
\paragraph{Combat Medic -} when you provide medical aid to a person, \roll{Skilled} and mark off 1 Supply from your kit. On 10+, the patient heals 2d4b damage. On 7-9, the patient heals 1d4 damage.

\paragraph{Stay With Me -} when you attempt to stabilize a teammate who is bleeding out, \roll{Steady} and mark off 2 supply from your kit. On 10+, choose 3. On 7-9, choose 2:
    \begin{easylist}
        # they can be moved without a stretcher
        # it takes fewer supplies than expected - mark off only 1 supply
        # you do not expose yourself to danger to help them.
        # they will not have a chronic injury
    \end{easylist}
Your patient does not die if you fail this move, and you may take -1 and try again. A second failure, however, results in the death of the patient.

\paragraph{Grace Under Fire -} when you are working on a patient during a fight but not actively fighting, you have +1 armor.

\paragraph{We All Bleed Red -} when you take time to treat an injured enemy, mark off 1 supply and \roll{Smooth}. On 10+, they’re stable, and you can ask two questions which they will answer truthfully. On 7-9, you can ask only one question.

\paragraph{Pharmacy Is Open -} when you use a contact to obtain medical supplies (amounting to +1 supply), and \roll{Smooth}. On 10+, choose 2. On 7-9, choose 1:
    \begin{easylist}
        # you get +2 supply instead of +1
        # it takes 1 day to get the supplies instead of 2
        # nobody notices the supplies are missing
        # you receive an interesting piece of information as well
    \end{easylist}

\paragraph{You Got This -} whenever you walk someone through a medical procedure (such as first aid), \roll{Smooth}. On 10+, they are boosted. On 7-9, they take +1.



\clearpage
\subsection{The Street Samurai}
\textbf{The Street Samurai} is a combat master. Often one of toughest and most skilled combatants on the team, the street samurai is a warrior for hire whose super-human talents were bought with cybernetic upgrades, relentless training, and no small amount of spilled blood. The Street Samurai may be a hired gun, but they take the word “samurai” very seriously, and adhere to a code of their own devising. On the streets of the Sixth World, the samurai is a feared -- and respected -- enemy.

\subsubsection{CREATING A STREET SAMURAI}
\begin{enumerate}
    \item Follow through general steps 2 to 6. Some suggestions for Street Samurai looks: \textit{glowing eyes, silvered eyes, hard eyes; cropped hair, wild hair, topknot; tattooed skin, scarred skin, camo skin; bulky body, lithe body, skinny body}
    
    \item \textbf{Choose your Equipment} \\
    Basic Equipment: commlink, lined coat \\
    In addition, choose from the lists below: \\
    Armor: form-fitting armor, ballistic vest \\
    Weapon: choose four weapons from the list of melee and small arms \\
    Cyberware: choose up to 5 essence worth of cyberware
    
    \item \textbf{Determine your Essence and Edge} \\
    Essence: 6 – total cost of all cyberware implants \\
    Edge: 1
    
    \item \textbf{Choose 2 Contacts} \\
    Arms dealer, cybersurgeon, bartender, street clinic nurse, private investigator, dockworker, pilot, cab driver, retired runner, survival nut
    
    \item \textbf{Create your Code} \\
    The word “samurai” means something on these streets. Create the code of honor that you follow.
    
    \item \textbf{Establish Debts and Favors} \\
    Place one of your fellow runners’ names in at least one of the blanks below:
        \begin{easylist}
            ## \_\_\_\_\_\_\_\_\_\_ came back for me.
            ## Even with all this chrome, \_\_\_\_\_\_\_\_ still treats me like a real person.
            ## I got this scar taking a bullet for \_\_\_\_\_\_\_\_\_\_.
            ## \_\_\_\_\_\_\_\_\_\_’s “big score” ended with me in the lockup.
        \end{easylist}            
    
    \item \textbf{Starting Funds} \\
    You start play with 3d6 x 250¥ immediately available.
    
    \item \textbf{Starting Moves} \\
    You know all the Core and Secondary Moves. You know The Only Thing Faster is Light move and one other Street Samurai move.
\end{enumerate}

\subsubsection{STREET SAMURAI MOVES}
\paragraph{The Only Thing Faster is Light -} whenever you \move{Rock \& Roll}, on a 12+ you may deal your damage to a second target within range.

\paragraph{More Power -} when you attempt to bend, break through, or otherwise destroy something, \roll{Hard}. On 10+, you easily achieve your goal. On 7-9, you break it, but (choose 1):
    \begin{easylist}
        # It takes longer than expected
        # It makes a lot of noise
        # You take 1 stun in the process
    \end{easylist}

\paragraph{Pain Editor -} when you make a \move{Gut Check}, you are boosted. Additionally, when you reach 9 or more wounds, you may choose to accept a chronic injury rather than bleeding out. If you already have all of the chronic injuries, you cannot use this move.

\paragraph{Honorable -} when you uphold a tenet of your code, \roll{Smooth}. On a 10+, hold 2. On 7-9, hold 1. You may spend this hold to pull strings, manipulate, or make someone sweat.

\paragraph{CQC Expert -} when you \move{Rock \& Roll} using a melee weapon or while unarmed, deal +1d4 damage.

\paragraph{Perfect Instincts -} when you act on GM’s answers after Checking a situation, take +2 instead of +1.

\paragraph{Dodge This -} when you manage to get out of an enemy’s line of sight, \roll{Steady}. On 10+, you get the drop on that enemy when you reappear. On 7-9, you take +1 forward against that enemy when you reappear.
