\chapter{Combat \& Injuries} \label{combat}

\epigraph{\textit{That which does not kill me \\ is dead when I'm through with it.}}{-- Joel Neechi, Mercenary}

Shadowrunners tend to get themselves into lots of trouble, the kind that ends with some high-intensity interpersonal conflict resolution. In other words, combat. As you’ll find when you read through the rest of this document, most of combat (in fact, pretty much everything the player characters do, ever) is handled through the application of various moves as they intersect with the fiction. This section explains a few specific quirks of combat in Sixth World.


\section{Combat}
\subsection*{Armor}
Because a shadowrunner leads a dangerous life, a big premium is put on not getting hit or at least not taking all the damage. The obvious way to do so is to wear armor.

In Sixth World, armor reduces incoming damage on a 1 for 1 basis. The tradeoff, of course, is that you can’t spend all day walking around in combat armor— it’s hot, itchy, intimidating, and cops tend to notice.

Some metatypes and archetypes offer moves that let you reduce damage, or otherwise avoid some of the less pleasant outcomes of damage. For example, the Hard Bastard move (and ork metatype move) lets the character take +1 to \move{Gut Check}, and the troll move You’ll Just Make It Angry grants an additional wound box.

\textbox{TL;DR}{Each point of Armor reduces received damage by 1.}


\subsection*{Surpise}
The \move{Rock \& Roll} move and most other damage-dealing moves assume that your target can fight back. If that’s not a possibility (that is, if your target is surprised, helpless, etc.), the fiction can’t trigger the \move{Rock \& Roll} move. You just put a round in their head and move on.

When you get the drop on someone in combat, you don’t need to use a move to deal damage to them — you can simply deal your damage (or kill them outright, depending on the situation). Likewise, if someone gets the drop on you in combat, expect to eat some lead.


\subsection*{Fire Modes}
Weapons in the game can fire in semi-automatic, burst, or full-auto modes, depending on their specific capabilities.

Semi-auto (that is, a single shot) is the ``default'' assumption; in that mode you don't use up ammunition on a 10+ on the \move{Rock \& Roll} move and only if you choose to burn extra ammo on a 7-9.

Firing in burst or auto modes when using \move{Rock \& Roll} allows you to add +1 damage to your attack; however, it always uses 1 ammo (even if you roll 10+).

Finally, full-auto mode is very useful for suppression fire, and lets you take +1 when you use the Suppression Fire move.

\textbox{TL;DR}{\textbf{Semi-auto:} just R\&R \\ \textbf{Burst:} +1dmg, always mark Ammo \\ \textbf{Full-Auto:} +1 on Suppression Fire, always mark Ammo}


\subsection*{Range}
Range is handled abstractly in most cases in Sixth World, since the flow of the game is largely a back and forth. You are free to use a map to keep a general sense of the scene and relative positions, but there’s no particular need to count squares, inches, or specific range increments.


\subsection*{Reloading}
Most of the weapons indicate some ammo capacity using the ammo tag - this indicates how much ammunition a weapon can carry in its magazine or clip before it must be reloaded. If a weapon has 3 ammo, for instance, you have ammunition in the gun until you have marked off all three ammo. Ammo is an abstraction - 1 ammo does not represent a single round, but simply ``some ammunition''. The game assumes (for the most part) that a character fires multiple shots in a single move.

During combat, assume that combatants are reloading their weapons when appropriate, keeping them topped up. Mechanically, this is handled by the fact that \move{Rock \& Roll} only lets you mark ammo as described above under Fire Modes.

When you mark off all your ammo, you’ll need to reload. There is no specific move to reload a weapon. If taking the time to reload would not expose you to danger, then you can reload simply by saying so. On the other hand, if you’re reloading despite an imminent risk, that’s a job for the \move{Null Sweat} move.



\section{Damage \& Healing}

Inevitably, when you play with guns, magic, and sensitive secrets, somebody is going to get shot. Or burned, or hit with a brick, or drenched in elemental acid summoned from beyond the realm of mortal kin, or thrown out a window, or... well, you get the point.


\subsection*{DEALING DAMAGE}
When you make a move that has the potential to deal damage, the move will usually say, as a possible result, ``deal your damage'' or ``you deal damage''. Damage in the game is usually variable, based on the damage dice for the weapon being used (see the \nameref{Gear} section for information on weapons). This is the amount of damage that is applied to your target.

\textbox{Example}{Johnny Chopz hits a ghoul with his trusty katana. The katana deals 2d6b damage (meaning roll 2d6 and take the best result). Johnny's player rolls 2d6, getting [3, 5]. Thus, the attack deals 5 damage to the ghoul. Bad news, creep.}

If a move indicates that you deal half damage, roll the damage as normal, and then divide the result by two (rounding up) to get your final damage amount.

The most common situation in which you’ll deal half damage is if you’re shooting at a vehicle with small arms. Vehicles take only half damage (before armor) from small arms, and no damage from melee weaponry.

\textbox{Example}{A go-ganger on a motorcycle is chasing down Johnny, who turns to shoot at the onrushing psycho with his Ares Predator. When he rocks \& rolls with the ganger, he's able to deal his damage (1d8+1). He wants to hit the vehicle, though,not the ganger. Halving the 5 he rolled yields 3 damage (5 / 2, rounded up), meaning that a bullet just gets through the armor, but it ain't gonna stop it. If he'd pulled out his katana and stood his ground... well, what would happen is that he'd end up with a motorcycle wheel up his nose.}


\subsection*{GETTING HURT}
When a character takes damage in the game, it is recorded by marking wound boxes on the character’s playbook. Most weapons in the game deal physical damage; when taking damage from this kind of weapon, mark off a number of boxes on the Wound Track equal to the damage taken. Getting dealt 3 damage, for instance, would mean that (all else being equal) the player would mark 3 wounds on their playbook.

If a weapon specifies that it deals stun damage, you still check off boxes on the Wound track. However, if a weapon dealing stun damage is the one that takes you out, you are knocked unconscious. All characters have a maximum of 8 wounds they can take. Once they reach 8, the next wound will put them on the ground, thoroughly incapacitated (whether unconscious, or worse).


\subsection*{EXTRA WOUND BOXES}
Some moves (such as the You’ll Just Make It Angry move) or equipment (like Bone Lacing) grant an additional wound box. (In the archetype dossiers, these additional boxes are shown with dotted lines. If you do have an extra box, just darken the lines so you know where to start filling in wounds.) No matter what equipment or moves you have, you can never have more than 10 wound boxes.


\subsection*{GUT CHECKS}

When a character takes damage in the game, it is assumed that, until the last couple boxes, while they may ultimately prove to be significant injuries, they’re minor enough to ignore for the moment. There are two exceptions: Wound \#8: when you check off that last box of your Wound track, you must make the \move{Gut Check} move.

Major Trauma: if you take 6 or more damage (after applying armor) in a single hit, you have just taken Major Trauma. You will need to make the \move{Gut Check} move.


\subsection*{BLEEDING OUT}
Once a character takes a 9th wound (that is, takes any damage after reaching their 8th wound box), they are Bleeding Out. This basically means they’re incapacitated, unable to perform any sort of action, and badly hurt (it doesn’t actually mean there’s blood everywhere; “bleeding out” just sounds cool).

A character who is Bleeding Out must be stabilized, either via the First Aid move or any relevant archetype moves.


\subsection*{CHRONIC INJURY}
If a character reaches the Bleeding Out stage, and survives their precarious situation, they will be left with a Chronic Injury. This is a long-term (and possibly permanent) reminder of their brush with death. Chronic Injuries reduce the affected Stat by 1 point. When your character receives a chronic injury, choose one of the following:

    \begin{easylist}
        # \textbf{Dulled (-1 Sharp):} your injury interferes with your perceptiveness or instinct; perhaps you suffered vision damage or hearing injury.        
        # \textbf{Weak (-1 Hard):} you’ve lost a bit of the hard edge that makes you dangerous and effective; perhaps you can’t focus, or traumatic experiences are affecting you.        
        # \textbf{Rattled (-1 Steady):} you suffered an injury that hampers your ability to remain focus, keep your cool, and act in the face of danger.        
        # \textbf{Disfigured (-1 Smooth):} your injury left you with nasty scars that are immediately obvious and shocking to the people you interact with.        
        # \textbf{Confused (-1 Skilled):} you can’t think straight, or you’ve lost your touch with skills you used to be expert at; it’s harder to apply your education, intelligence, and training to your tasks.        
        # \textbf{Faded (-1 Essence):} whether it fed the unnatural thirst of some paranormal creature, fueled a dark ritual, or just got hacked away by someone meaner and faster than you, you lost a piece of yourself.
    \end{easylist}
        
You can’t have the same chronic injury twice. If you are already Faded, and you take a second chronic injury, you’ll have to choose something else. However, if you heal a chronic injury and recover the lost stat point, you could elect to take it again in the future.


\subsection*{GETTING BETTER}
Generally, as long as a character has not received more than 8 wounds, and has not failed a \move{Gut Check}, they are not incapacitated by injury (though they may be feeling very much the worse for wear). Recovery from this level of injury is really a matter of time, and perhaps a small amount of attention from their, ah... let’s say, primary care provider.

Mechanically, injury of this nature will be healed during downtime, assuming that they get approximately two days of rest and basic medical care for each wound box they have (if the damage was mostly dealt by stun weapons, then it takes much less time to heal — if you track damage differently, all stun goes away after a solid rest).

\textbox{Example}{Navy got hurt on her last run, but she was on her feet and processing oxygen at the end of it, so she considered it a job well done. She finished the run with 4 wound boxes checked. This means that she will need to have roughly 8 days of rest and medical care to heal those injuries, at which point she’s good as new.}


\subsection*{HEALING CHRONIC INJURIES}
Chronic Injuries are not necessarily permanent injuries, unless the player wishes them to be. However, they can only be healed or ameliorated by major or long-term treatment. A chronic physical injury may be fixed via cybernetic replacement, for instance, which is a major surgical intervention. Chronic psychological injury may require therapy over a long term as well.

It is up to the GM and players to negotiate the specific plan for removal of a Chronic Injury. It may be that recovery may evolve into a shadowrun of its own, but that is not required: spending funds to pay for therapy, new cyberware, surgery or the like is sufficient if you want to keep the story of the recovery as a background event.


\subsection*{GETTING BURIED}
With the rules covering stabilization, chronic injury, armor and so forth, it’s actually fairly hard to all-the-way die in Sixth World. However, it can happen in a few different ways.

    \begin{easylist}
        # \textbf{Failed to Stabilize:} if the person attempting to provide First Aid to Bleeding Out character fails their move, the wounded character cannot be stabilized and dies at the end of the encounter.
        # \textbf{Continued Damage:} if a character takes 6 damage beyond that 8th wound box (armor still counts!) they’re too badly mangled to be saved. Players, understand that this can happen; GM’s, be really careful with this one.
        # \textbf{Overwhelming Kaboom:} if a character is hit with an attack of such overwhelming power that surviving it overly strains all suspension of disbelief, they’re killed immediately. For example, if a character is, say, hit by an anti-aircraft missile, or falls into a crucible of molten iron... then forget it, they’re gone.
    \end{easylist}
        