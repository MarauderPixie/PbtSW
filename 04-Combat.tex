\chapter{Combat} \label{combat}

Es geht um nicht weniger als die Beteiligung der Runner am Unabhängigkeitsprozess von Seatlle. Nach den Erignissen in/um Detroit und Ares und inmitten des Tumults der Aufkündigung der BRA von den UCAS und den Blackouts an der Ostküste werden die Runner angeheuert, um Informationen im Rahmen eines Diplomatiegipfels zu beschaffen.

\section{Szene 0: Interludium}

Auf dem Weg heim, durch die Mall: BREAKING NEWS - Arthur Vogel's Statement.

\gesicht{Detroit Free Press: BREAKING NEWS -- Wir unterbrechen unser reguläres Programm, um zum aktuellen Stand der Lage in Detroit zu berichten. Der neue Vorstandsvorsitzende von Ares Macrotechnology, Arthur Vogel, gibt ein erstes Statement nach Detroit-Krise ab.}

Yu lässt sich die Muskeln straffen - 3 Tage knocked out - genug für ein Wissenstalent.

Tusk will 'ne Monofilament-Peitsche haben. Best Bet: Benjamin Flowers. ``Nicht, dass ich dir in deine Ethik quatschen will, aber dir sollte klar sein, dass solche Teile eher als endgültige Lösung gedacht sind - und nicht gerade diplomatisch?'' -- ``Ich denke, sowas lässt sich auftreiben, aber dafür bist du mir einen Gefallen schuldig.''

Außerdem ist da noch die Challenge, Wapeka "Skillful" Becerra zu besiegen, um die Initiation zu beginnen:

\textbox{Wapeka}{
    Ini: 16, 
    NK: 13, 
    VT: 11 (11), 
    Waffe: Stab (4B, 10 AT), 
    Sonst: Athletik 12, 
    HP: 12K/12B
}

Rude bekommt von Hez mitgeteilt, dass sein Block aufgekauft wurde - von Vivaldi Immobilien Ltd., if that helps anything.

\subsubsection{Wrapup}

\begin{easylist}
    # Tusk hat Wapeka besiegt und kann jetzt ihre Initiation beginnen.
    # Yu hat die OP gut überstanden: +3 \skill{Geschick}, not bad at all
    # Rude macht sich Gedanken zur Runner-WG, ahem, zu einem gemeinsamen Safehouse; an den Grenzen von Puyallup oder Redmond in angenehmere Gegenden
\end{easylist}




\section{Preludium: UC\texorpdfstring{\textsubscript{r}AS\textsuperscript{h}}{UCrASh}}

\gesicht{Knapp 3 Wochen sind vergangen, seit die Krise in Detroit für beendet erklärt wurde. Aber ruhig blieb es nicht, ganz im Gegenteil: keine drei Tage später haben die UCAS als Reaktion auf den Bruch mit Ares eine metaphorische Atombombe gezündet, indem sie unilateral den BRA - den \textit{Business Recognition Accords} - aufgekündigt haben. In den BRA ist festgesetzt, dass der Konzerngerichtshof als alleinige Instanz über die Exterritorialität von Konzernen entscheidet. Die UCAS haben den AAA-Megakonzernen also nicht weniger als ihre Geschäftsgrundlagen wie einen Teppich unter den Füßen weggezogen. Im dadurch entstandenen Chaos, auch oder sogar besonders in Seattle, wurden dadurch die Rufe nach der Klärung der Frage nach Seattles Unabhängigkeit lauter. Eine der vielen Nachrichtenmeldungen zu dem Thema lautete zB folgendermaßen: }


\subsection{Die Lage in Seattle}

Aus \textit{Free Seattle - Einleitung} - alles hier gehört direkt \textit{\textbf{ins Gesicht}}.
 
\subsubsection{Seattles Zukunft steht auf dem Spiel}

Vor etwas mehr als einem Jahr wurde die Gouverneurin von Seattle, Corinne Potter, mit einem Programm gewählt, das ihr die Stimmenmehrheit brachte, aber nur wenige Versprechungen enthielt. Brackhaven hatte ihr viele ungelöste Probleme hinterlassen, und Potter versprach, sich um jedes einzelne zu kümmern, auch wenn ihre Kampagne nur wenige Details dazu enthielt. Seit ihrer Wahl hat Potter Berater und Experten hinzugezogen, um zu versuchen, die richtige Lösung für jedes dieser Probleme zu finden. Mit ihren Entscheidungen hat sie ihre Wähler manchmal verstimmt, da diese vielleicht erwartet hatten, dass sie eher auf ihr Herz als auf ihren Verstand hört, aber jetzt kümmert sie sich um das vielleicht kontroverseste Thema ihrer Kampagne: Seattles Unabhängigkeitsbewegung. 
Während der Wahl gab es mehrere Forderungen nach einem freien und unabhängigen Seattle. Diese Rufe stießen fast gleichermaßen auf Zustimmung und Ablehnung. Seattle ist in dieser Frage eindeutig tief gespalten. Um ihr bei der Lösung dieser Krise zu helfen, hat Potter Vertreter mehrerer Länder und Megakonzerne eingeladen, die ein besonderes Interesse daran haben, ob Seattle unabhängig wird oder bei den UCAS verbleibt.
Es überrascht nicht, dass Potter mehrere Vertreter des Konzerngerichtshofs eingeladen hat, insbesondere Major Brenda Reed von Ares, Thomas Miranda von Horizon und Takuto Nakagawa von Renraku; dazu kommt noch die Pacific Prosperity Group, die Wuxing-Exec Dewei T’ao an den Verhandlungstisch geschickt hat. Darüber hinaus haben Seattles Nachbarn ein Mitspracherecht: Der Salish-Shidhe-Rat hat John Abernathy vom Salish-Stamm geschickt, die UCAS haben den frischgewählten Kongressabgeordneten Carl Derrick entsandt, um sicherzustellen, dass die Dinge so bleiben, wie sie sind, und Tír Tairngire wird durch Margaret Telestrian vertreten. Die vielleicht überraschendste Einladung zum Treffen ging an die Seedrachin, die auf der Konferenz noch nicht erschienen ist. Was die Anwesenheit eines der umstrittensten Drachen der Welt für diese Konferenz bedeutet, lassen wir mal dahingestellt. 
Im Laufe der nächsten Woche wird Seattles Zukunft zum Teil von diesen Parteien bestimmt werden. Nur die Zeit wird zeigen, ob die Stimme von Seattles Bevölkerung wichtiger sein wird als die Stimmen von Seattles Megakonzernen.

\vspace{1em}
\textbf{An Tusk}: Deine Connection April Summers schrieb in einem Kommentar dazu: 

\gesicht{Das ist vielleicht das klügste politische Spiel, das Potter spielen konnte. Sie kann öffentlich Unterstützung sammeln und bei einigen Mächten das Terrain sondieren, und wenn es schiefläuft, kann sie immer noch den anderen die Schuld zuschieben. Allerdings bedeutet das auch, dass sie das Rampenlicht meidet und nicht die volle Anerkennung für einen möglichen Erfolg erhält. Sie verspielt einen Teil der Publicity des Erfolgs, um die Kritik an einem möglichen Scheitern zu dämpfen. Das wiederum könnte letztendlich als rückgratlos angesehen werden.}


\subsection{Die Lage der Runner}

All das ist natürlich auch in eurem Umfeld ein großes Thema: Yu, du weißt zB von Mia, dass sie Feuer und Flamme für ein unabhängiges Seattle ist:

``Was haben die in DC denn jemals für uns getan? Ihre scheiß Armee sorgt nur für mehr Spannung zwischen dem Council und den anderen NANs als für irgendwelchen angeblichen Schutz! In keinem anderen Metroplex der Welt herrscht ein solches Machtgleichgewicht wie hier, keiner der Megas hier hat dsa Sagen - nicht Ares, keiner der Japanokons, \textbf{niemand} - nichtmal das goldene Würmchen hat hier viel zu melden. Und das liegt sicher nicht an DC, sondern allein an \textbf{uns}, an den Seattler Schatten. Wir sind \textbf{niemandes} Schoßhunde, hier suchen \textbf{wir} uns aus, für wen wir laufen.''. 

Rude, im Burning Hole werden Wetten darauf abgehalten, welche Repräsentanten am Leben bleiben und du hast mit Hez darauf gewettet, dass Seattle unabhängig wird - die Chancen stehen 2:1, auch wenn dort absolut niemand auch nur im Ansatz genug von der Sache versteht, um das einschätzen zu können. Du weißt auch, dass Hez es lieber wäre, wenn es nicht dazu kommt: ``DC ist zwar auch nur ein weiterer Sumpf, aber wenigstens ein weit entfernter. Und immer noch besser, als wenn die Kons hier komplett den Laden übernehmen. Ich mein, die Containment Zone in Chicago strahlt nach wie vor stärker als die Sonne über Fujiyama. Und guck dir doch an, wie es in Detroit aussieht: alles liegt in Schutt und Asche und was macht Ares? Sie verschwinden. Einfach so. Aus ihrer \textit{eigenen} Amerikanischer-als-Uncle-Sam-Vorzeigeenklave. Einfach, weil ein Wiederaufbau zu teuer ist. Und von den anderen fangen wir am besten gar nicht an. Du weißt genauso gut wie ich, wieviele Trogs es bei den Japanos auch nur in der kleinsten Tochter gibt - nicht einen einzigen.''




\section{Szene 1: Glanz, Gosse, Gloria}

Mia hat die Runner zu sich gerufen, um einen Job zu vermitteln; Gilroy 'Romeo' Steele ist auf die Runner aufmerksam geworden und sucht ein Team, dass für die Dauer des Gipfels verschiedene Aufträge für die Gipfelteilnehmer erledigen und ihm Bericht erstatten wird.


\subsection{Das Pan Pacific Hotel}

\gesicht{Das Hotel könnte beeindruckender nicht sein. Der Weg zum Eingang führt bereits an einem perfekt manikürten Design-Vorgarten vorbei. Spätestens, als ihr das Foyer betretet, ist klar: Ihr betretet das luxuriöseste Hotel, das ihr jemals gesehen habt: das Pan Pacific Seattle. Alles hier ist luxe, De-luxe, extra-luxe. Wenn es möglich wäre, dass die Einrichtung noch mehr Grandeur vermittelt, hätte sie eine eigene Postleitzahl.}

\skill{Wahrnehmung:}

\subparagraph{(1)} Der Stil ist neo-modernistisch gehalten: Smaragdapplikationen auf den Möbeln, echte Messingglocken an Telefonen, die Laufburschen tragen Fez, die Zimmermädchen Rüschen-Schürze. 
\subparagraph{(2)} In diesem perfekten Luxus erscheint es fast etwas seltsam, dass hier ein Hausmeister am Fahrstuhl mit einem Mop den Boden wischt. 
\subparagraph{(3)} du hast den Eindruck, dass er euch beobachtet
\subparagraph{(4)} er hat eine cyberbuchse hinter dem Ohr, ungewöhnlich für einfache Hausmeister (wenn yu das erfährt: erinnerung 2: iwas mit hiding in plain sight)


\subsection{Gilroy 'Romeo' Steele}

Das Team soll offiziell bestimmte Aufgaben für die Gipfelteilnehmer erledigen. Danach sollen die Runner Romeo über jeden einzelnen dieser Jobs berichten: was dabei jeweils passiert ist \textbf{und was das Team über den jeweiligen Job denkt}.

Er wird das Team mit einem Safehouse in \textit{Puyallup}, der elfischen Version der Barrens, ausstatten. Romeo möchte, dass die Runner während der Aufträge, die sie ausführen, auch sämtliche verwertbaren Informationen und Erkenntnisse über ihre Auftraggeber sammeln, die sie finden können.

Er bezahlt für alle Daten, die die Runner über die Teilnehmer ausfindig machen können, egal, wie sie sie bekommen. Sie können die Informationen von Connections, lokalen Quellen, Aufzeichnungen, Datenspeichern oder Zeugen erhalten. \textbf{siehe \textit{Legwork} zu den Personen als exzellente Quelle für Informationen, die ich irgendwo einbauen kann!} 

\gesicht{``Wie ihr an diese Infos kommt, ist mir egal. Ihr könnt eure Connections bemühen, lokalen Quellen ausfindig machen, Aufzeichnungen und Datenspeicher auftreiben oder Zeugen befragen, also lasst euch was einfallen. Und vor allem: seid aufmerksam und lasst euch nichts durch die Finger gehen! Ihr kennt ja das Sprichwort: Haltet euch den Rücken frei, spart Muni, zielt genau und lasst euch nicht mit Drachen ein.}

Erster Auftrag: einen Kontakt auf der Gala treffen (\probe{Einfluss}{gg Romeo} um zu erfahren, wen). Am besten früh erscheinen, keine Waffen.


\subsection{Puyallup Safehouse}

\subsubsection{Sag's ihnen ins Gesicht}

Romeo hat sein Wort gehalten und dafür gesorgt, dass ihr für die Dauer eures Aufenthaltes in einem Stadthaus in Puyallup wohnen könnt. Puyallup ist nicht gerade das beste Viertel, nur etwas besser als die Redmond Barrens. Es ist eine Gemeinschaft der Entrechteten, vor allem Elfen, die aber fast jeden akzeptieren, der ein Verstoßener ist. Es ist ein großartiger Ort, um sich zu verstecken, aber es ist auch gefährlich und unberechenbar. Wie das eben so ist, wenn man auf einer Vulkancaldera sitzt. 

Einige der Straßen sind mit recyceltem Kunststoff und gekühlter Lava gepflastert. Die Straßen sind an manchen Stellen eben, an anderen unwegsam, und gelegentlich haben sich Felsen und Krater gebildet. Nicht gut zum Fahren oder Gehen, aber besser als gar kein Pflaster.

\textbox{Wahrnehmung:}{
    \subparagraph{(1)} Ok, die Gegend ist ein Slum... Die Wände sind voller Risse, die Fenster regelmäßig zerbrochen. Alles ist voller Graffitis. 
    \subparagraph{(2)} Du kannst einige davon ausmachen; zB \textit{Soykaf ist Menschen!} oder \textit{Azzies raus, Ancients rein!} Tatsächlich auch verblasste Logos der Gang, aber auch die vieler anderer. 
    \subparagraph{(3)} die meisten Logos, und auch die neusten, sind eine Art stilisierte Krabbe, oder ein Hummer. Irgendein Krustentier zumindest. (roll Gang Knowledge: (3) Rock Lobsters)
}

Euer Safehouse scheint ein verlassenes Stadthaus zu sein. Die Fenster sind mit einer dicken Schmutzschicht bedeckt, und die Fassadenverkleidung ist stellenweise abgefallen. Aber die Türschlösser sehen stabil aus, und die Substanz des Hauses scheint gut erhalten zu sein. Das deutet darauf hin, dass das schmuddelige Aussehen beabsichtigt ist. 

Ihr findet die formelle Kleidung in einem der Schlafzimmerschränke. Sie steht in scharfem Kontrast zu der abblätternden Farbe und den fehlerhaften AR-Displays im restlichen Haus. Aber ihr habt schon an schlimmeren Orten gewohnt. Ihr habt noch ein paar Stunden zum Ausruhen und vorbereiten, bevor ihr zum Ballsaal des Pan Pacific Hotels fahrt.


\subsection{Die Gala}

Das Hotel ist bereits früh am Abend von riesigen Scheinwerfern beleuchtet, manche Gäste kommen offenbar per Hubschrauber - einer landet gerade im ``Vorgarten'' - andere in dicken Bonzenkarren. 

\subsubsection{Sag's ihnen ins Gesicht}

Als ihr den Ballsaal betretet, seid ihr sofort von der hier zur Schau gestellten Opulenz überwältigt. Die Leute tragen Kleidung, die mehr kostet als der Monatslohn eines Lohnsklaven. Die Nahrung ist echte Nahrung, kein Soja oder Mykoprotein. Umgeben von einer Menschentraube steht eine strahlende Elfe, die zu euch herüberkommt, als sie euch erblickt. Ihr Kleid funkelt und leuchtet, und winzige Sterne lassen eine Aura der Schönheit um sie herum entstehen. Ihr kurz geschnittenes Haar betont die Länge ihrer Ohren, und sie nähert sich euch mit dem anmutigen, selbstsicheren Schritt eines Raubtiers.

\subsubsection{Der Run}

Im Grunde bin ich hier ziemlich durch gerushed; die anderen hatten kaum Möglichkeitenm, was zu tun (mangels Beschreibungen auch) und ich hab völlig allerlei Dinge vergessen: Bezahlung? Was genau klauen? Wo abgeben? Siehe v.a. Wrapup Szene 2.

Telestrian will alles über die Pläne des SSC in den Verhandlungen wissen. Dafür zahlt sie 2.000\nuyen pro Kopf und lässt sich (ggf.) auf 2.500, d.h. 100 pro Nettoerfolg hochhandeln.




\section{Szene 2: Council Island}

Beschreibung von Council Island; Beschreibung der Situation vor Ort; welche Paydata liegt wie/wo vor; Karte und Icons vorbereiten.

\subsubsection{An den Docks}

\gesicht{Ihr fahrt zu den Docks von Tacoma und findet schnell Pier 25. Am Ende des Piers tanzt ein kleines U-Boot auf dem Wasser. Das U-Boot bietet gerade so Platz für drei Passagiere - erst recht, wenn einer davon ein bulliger Troll und eine weitere eine kräftige Orkin ist. Es ist also an der Zeit, zu kuscheln. Zum Glück seid ihr kein Team das auf schwere Hardware oder Panzerung setzt, denn dafür wäre hier eindeutig kein Platz gewesen.}


\subsection{Die Botschaft}

\textbf{Hauptziel} ist: ein Textdokument über die Pläne des SSC während der Verhandlungen (siehe \textit{Legwork - John Abernathy}). 

Bonus A: ein Ordner zur Telestrian Industries Corporation. 

Bonus B: ein weiterer Ordner über Dewei 'Dewey' T'ao.

\subsubsection{Und sonst so...}

\begin{easylist}
    # Wissensproben sagen was über:
    ## Die Ausrüstung der Wachen (wg. Protesten)
    ## Möglichkeiten, Kameras zu umgehen
    ## Druckgeflechte
    ## die Sicherheitsspinne
    ## den Aufbau der Botschaft: 
    ### $\rightarrow$ zwei Stockwerke mit identischem Layout
    ### $\rightarrow$ Botschafterbüro vmtl in prominenter (mittiger) Lage im oberen
    # Hauptziel liegt auf dem Schreibtisch; ja, einfach so
    # beide Boni fallen mit \probe{Wahrnehmung}{3} auf
    ## Bonus A in einer verschlossenen Schublade (Stufe 2: ``ein einfaches, mechanisches Schloss; von den meisten als kurios bezeichnet, von euch als 'leicht zu öffnen'.'')
    ### $\rightarrow$ zum Öffnen: Dietrich-Set und \probe{Mechanik}{2}
    ## Bonus B in einem Safe (Stufe 4) hinter einem Gemälde des ersten SSC Häuptlings Jon Moses
    ### $\rightarrow$ erst \skill{Wahrnehmung}, dann \probe{Mechanik}{4} + Würfel aus der vorherigen WN
\end{easylist}


\subsection{Wrapup}

Dinge, die in \textbf{Szene 1 und 2} passiert sind:

\begin{easylist}
    # es gab 20.000\nuyen Vorschuss von Romeo
    # die Runner haben für 1500\nuyen einen GMC Bulldog gemietet
    # Rude, Frosty und Tusk haben Actioneer(?) Anzüge bekommen; zumindest aber angemessen formelle Kleidung
    # die Bezahlung für den Botschafts-Run beträgt 2.200\nuyen pro Kopf
    # Beute vom Einbruch:
    ## Akte ``Unabhängigkeitsstrategie Seattle'' (\textbf{Hauptziel}, s. \textit{Legwork - John Abernathy})
    ## Dossier ``Telestrian Industries Corporation'' (\textbf{Bonus A}, s. \textit{Legwork - Margret Telestrian})
    ## eine Flasche Rum, wenn Yu das zugelassen hat
    ## eine Schachtel Zigarren
\end{easylist}


\section{Szene 3: Dirty Laundry}

Konfrontation mit den \textbf{Rock Lobsters}? Debriefing mit Romeo. Neuer Auftrag von Major Brenda Reed; Bunraku Salon schonmal vorbereiten.


\subsection{Gang }

Wir starten in den Ubooten von \textit{Council Island} nach \textit{Tacoma}. Auf der Fahrt, sowie der Fahrt zurück zum Safehouse haben die Runner ca. 2,5 Stunden, um ihr Material zu sichten. Bei ihrer Ankunft werden sie von den Rock Lobsters angegangen.

\gesicht{Es ist eindeutig, dass ihr immer weiter nach Puyallup hinein zurückkehrt: die Gebäude werden heruntergekommener, die Straßenlöcher häufiger und größer, der Geruch von Meer uns Salz weicht dem von Asche und Staub. Als ihr euren Wohnblock erreicht, fallen euch mehrere Grüppchen auf, die um brennende Fässer herumstehen.}

\textbox{Wahrnehmung:}{
    \subparagraph{(1)} Ein ganz normaler Spätsommerabend in den Slums. 
    \subparagraph{(2)} Seit einer Weile fährt auch ein weiteres Fahrzeug ein Stück weit hinter euch - und scheint euch zu verfolgen. 
    \subparagraph{(3)} Tatsächlich sind es mehrere Quads und Crossbikes, auf denen jeweils auch mehrere Leute sitzen. 
}

Vor dem safehouse steigen sie aus: 8 Leute, offensichtlich Go-Ganger. Auf manchen der Synthlederjacken ist hinten ein ähnliches Logo zu erkennen, das euch schon tagsüber als sehr präsent aufgefallen ist: ein stilisierter Hummer.


\subsection{Debriefing}

Romeo ist am nächsten morgen in der Küche und hat Kaffee gekocht - zumindest falls keine Wache bestimmt wurde.

1. Was denkt ihr über Margaret?

2. Wie ist der Run abgelaufen; was wollte Maggie? Gab es Probleme?

3. Was habt ihr über John Abernathy herausgefunden?

4. Und über andere Parteien bzw. deren Abgesandte?

5. Was haltet ihr von dem ganzen? Denkt ihr, der SSC wusste Bescheid? Dass Maggie weiß, was der SSC über sie und Telestrian weiß?

Es gibt insgesamt 9.000\nuyen: die mit Magaret ausgehandelten 2.200 pro Kopf + 2.400 für das Telestrian Industries Corp. Dossier.

``Bereitet euch schon mal auf den nächsten Run vor. Im Laufe des Tages wird es wieder was zu tun geben.''


\subsection{Der Run}

\gesicht{Am Nachmittag, gegen 15 Uhr, ertönt ein Klopfen an der Tür. \textbf{Reaktion abwarten!} Ihr macht eure Waffen bereit und schaut vorsichtig durch den Türspion. Auf der anderen Seite steht ein gelangweilt aussehender Teenager – ein Zwergenmädchen mit Pickeln im Gesicht und einem rosafarbenen Iro, das etwas kaut, von dem ihr hofft, dass es Kaugummi ist. Als ihr die Tür öffnet, gibt sie euch einen Datenchip, dreht sich um und geht weg, ohne ein Wort zu sagen. Auf dem Datenchip seht ihr eine Nachricht von einer Frau, die keinem von euch sofort bekannt vorkommt: eine Asiatin mittleren Alters. Ihr Gesicht ist streng und ernst, was durch ihren Dutt
noch betont wird, und man erkennt sogar auf der Trid-Projektion noch die harten, flachen Muskeln
einer Athletin. Sie sagt, ihr sollt sie ``Major'' nennen und sie will, dass ihr in einen Bunraku-Salon, dem \textbf{Cherry Patch} in \textit{Nord-Tacoma}, geht und euch dort um einen der Kunden kümmert. Die restlichen Details befinden sich auf dem Datenchip.}

Die Zielperson ist Carl Derrick, der Abgesandte der UCAS und die Mission besteht darin, einen Weg zu finden, ihn dazu zu bringen, eine der Prostituierten zu töten. Die Bezahlung beträgt 2.500\nuyen pro Person und bergeweise dreckige Wäsche, von der zumindest ausgegangen werden kann, dass sie in einem solchen ``Upper Class'' Bunraku-Salon zu finden ist. 



\subsection{Change of Plans?}

Die Runner stehen vor einem moralischen Dilemma. Well, und ein Spieler mimimi't mit dem Setting (\textit{nach} mehrfachem Fragen ob wetwork ok sei, ah well...).

\subsubsection{Status Report}

\begin{easylist}
    # Carl Derrick ist Abgeordneter der erzkonservativen Partei, äußerst patriotisch.
    # Infos über Reed sind den Runnern bekannt; Gerüchte über Kriegsverbrechen sind nicht öffentlich.
    # Die Runner haben einen Datenchip mit einem Trideo, in dem sie von einer deutlich erkennbaren Major Reed zum Mord/Blackmail aufgefordert / angeheuert werden
    # Sie können dem einfach nachgehen (Abentuer verläuft wie vorgesehen)
    ## Magische Beeinflusing von Derrick; leider nicht vorhanden
    ## Hacken der Persona-Chips (und des Hosts); leider kein Decker verfügbar
    ## 
    # Sie können Carl Derrick
\end{easylist}


\subsubsection{Vor dem Salon}

\gesicht{Der Salon befindet sich in einem beschäftigten Bezirk in Nord-Tacoma. Die Straßen sind voller Leute, die geschäftig ihren Vorhaben nachgehen, Werbung strahlt in knaliggem AR- und Neon-Bild und Ton auf die Straße, von den Ramenständen und ähnlichen hört man hektisches Treiben. In der Luft liegt ein Geruch von Abgasen, altem Fett und Asphalt, der vom letzten Regenschauer trocknet.}

\textbf{Bild zeigen!} 

\textbox{Wahrnehmung:}{
    \subparagraph{(1)} Der Bunraku-Salon selber ist, wenig überraschend, als solcher kaum zu erkennen. Er wirkt eher wie eine hochklassige Cocktailbar. Davor steht allerdings ein äußerst massiger, äußerst großer Mann mit Glatze, offensichtlich asiatischer Abstammung. (falls die Runner den Eingang zunächst beobachten, fällt hier auch auf, dass jeder, der rein will, einen kurzen Austausch mit dem Steher hat (\probe{}{2}: evtl. eine Losung für den Eintritt)
    \subparagraph{(2)} Dir bemerkst einen offensichtlich betrunkenen Gast, der aus dem Laden torkelt (\probe{Menschenkenntnis}{2}: ...und dir fällt auf: wer raus kommt, muss vorher auch reingekommen sein.)
    \subparagraph{(3)} An der Seite, in Griffreichweite des Türstehers, siehst du - etwas bedeckt - etwas, das ein Gehstock oder ein Regenschirm sein könnte. Du erkennst aber: es ist ein Katana, das in einer Scheide steckt. Wer nicht wirklich genau hinschaut, würde das nicht sehen.
}

Der Türsteher hält euch auf: ``Passwort?''; oder eben 5.000\nuyen.


\subsubsection{Im Salon}

Nachdem die Angelegenheit mit dem Zutritt auf die ein oder andere Art gelöst wurde, können die Runner den Salon betreten:

\gesicht{Ihr betretet zunächst eine Art Vorraum, der bereits relativ dunkel und eher modern als traditionell gehalten ist. Eine weitere Tür bringt euch in den eigentlichen Salon: von einer relativ niedrigen Decke vermittelt dämmeriges, blaues Licht ein Gefühl von Diskretion. Es gibt eine Full-Service Theke mit etwa einem Dutzend Hockern, an der zwei vereinzelte Gäste sitzen und im Hauptbereich sorgt eine Tänzerin für Unterhaltung, während die Anwesenden Kunden an Tischen auf ihre Suite warten - einige der anwesenden hier sind allerdings offensichtliche Yak-Enforcer. Im hinteren Bereich führen Treppen nach oben, offenbar zu besagten Suites und auf der anderen Seite der Treppe (nicht gegenüber, sondern quasi \textit{unter} der Treppe) befindet sich eine weitere Tür.}

% Wenn sich die Runner umschauen, werden sie Lucky Strike sehen?

Derrick selbst wird abgeklärt sein. Und ein (patriotisches) Arschloch. Er weiß von ihren allmorgendlichen Läufen und hat von ihrer Nahkämpferei gehört.


\subsubsection{Wrapup}

Yu hat konnte sich mit Derrick unterhalten, ihm das Nitro ins Getränk mischen und anschließend hinterher schleichen. Rude und Yu haben einen Enforcer ausgeschaltet und stehen nun vor der Tür von Derrick's Suite. Tusk sitzt draußen im Van und hat D's Gorilla-1 und den SUV im Blick. Blöderweise habe ich angekündigt, dass KE schon mit fetter Mannschaft anrollen wird, wenn sie einen entsprechenden Tip bekommen. Ergibt halt keinen Sinn, aber das ist nun, wie es ist...

\begin{easylist}
    # Derrick wird seinen Panic-Button drücken (wollen)
\end{easylist}





\clearpage
\section{Hauptdarsteller}

\subsection{Gilroy 'Romeo' Steele}

Zwerg, Information Broker, Spin-Doctor

\subsection{Margaret Telestrian}

vertritt das Tír Tairngire (``Tier Ta'en'gier''), eine der Vorsitzenden von Telestrian Industries.
Sie hat die hohen Wangenknochen und spitzen Ohren, die die meisten Elfen haben, und sie schneidet sich bewusst die Haare kurz, um diese Unterschiede hervorzuheben.