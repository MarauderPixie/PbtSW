\chapter{Magic} \label{magic}

In the Sixth World, the magic has returned to the world, and dormant powers have reawakened. Magic is fueled by Essence, one of the variable point pools each character has.



\section{The Awakening of Metahumanity} % oder so. ESSENCE und ESSENCE RECOVERY sind zumindest keine eigenen sections...

\subsection{Essence}
Three archetypes in the game - the Adept, the Mage, and the Shaman - are magically gifted, which means that they are able to spend their Essence to use their magical abilities.

\textbf{The Adept:} adepts turn their magical ability inward to improve themselves, sometimes to superhuman levels. An adept spends essence to temporarily modify his or her capabilities (for example the Enhanced Ability or Killing Hands moves).

\textbf{The Mage:} when a mage spends essence to power a spell, the player marks off the spent essence. That essence is not available for future spells until it is recovered. The amount of essence spent is a general indicator of the power, or Force, of the spell.

\textbf{The Shaman:} when a shaman spends Essence to summon a spirit or elemental, they are in effect committing or wagering some amount of essence to do so.

The Essence spent indicates the number of services the spirit will perform (mechanically, the number of moves the spirit may make before dissipating). A Shaman allocates this essence at the time of summoning. If the summoning fails, the wagered essence is lost. If the summoning succeeds, then the essence is “tied up” until the spirit is dispelled/destroyed.


\subsection{Essence Recovery}

All magic users may recover essence by resting. A substantial rest (usually a night’s sleep) will recover all Essence spent. Some archetypes have additional means of recovering essence, as described below:

\textbf{The Mage:} mages may use the Center move to recover some Essence without resting, simply by taking a moment to concentrate and recenter him- or herself.

\textbf{The Shaman:} because the essence used to summon a spirit is in effect a wager, when the spirit has performed its actions (or is dispelled by the shaman who summoned it), the essence “tied up” in the spirit immediately returns to the shaman. If the spirit is dispelled by another person, or destroyed, only half the wagered essence (round up) is recovered.



\section{Astral Space}

Much like the Matrix, Astral Space is a sort of alternate universe adjacent to our own. It is where spells, spirits, magical creatures, wards and more reside. When an individual perceives the Astral, they can see the entities existing in Astral Space. All three arcane archetypes can astrally perceive. In addition, they can perceive emotional auras of living beings, as well as background magical nature of the area. When an individual projects themselves into astral space, they transfer their consciousness from their physical body to the astral plane, and can fully interact with other Astral entities and traverse great distances. The Shaman and Mage can astrally project. The following effects occur while perceiving or projecting: 

\textbf{Perceiving:} while astrally perceiving, take -2 ongoing to any moves in the physical world.

\textbf{Projecting:} you cannot take action in the physical world (your body is unconscious and helpless).



\section{Astral Quests}

The Astral also serves as a huge deposit of magical information, though most of the deepest knowledge is hidden in the metaplanes. Metaplanes are the planes beyond the Astral, the real sources of all magic. Every metaplane has a citadel, a core of pure magical energy that can alter the magical world. Accessing it can let you destroy a spirit permanently, learn some information such as the true name of a spirit, or learn an individual’s true aura. Note, however, an astral quest may only have a single goal. Astral Quests are also dangerous in that you are stuck in a metaplane until you either complete your Quest or fail. You can’t give up, and you can never go back, only forward.


\subsection{Domains}
To go on an Astral Quest, you must visit various metalocations known as domains, similar to Nodes in the Matrix (in fact, mapping these domains is a useful tool to keep play on track and engaging). The number and nature of these domains depends on the quest you are undertaking, but each one presents a challenge the character must complete in order to move on to the next domain. This could be fierce combat, a riddle, a puzzle or any variety of things.

Minor quests usually have 3 or 4 domains, while major quests can have up to 10 or more, all of which lead, ultimately, to the Citadel, where the quester will find the object or information they seek. Moving from domain to domain is as simple as willing yourself there once the task in the current domain is completed.


\subsection{The Dweller}
The first domain you encounter is always the Domain of the Dweller, a mystical being who blocks the entrance to the metaplanes. The Dweller knows everything about the quester, and will always question the nature your quest before granting passage. The Dweller is an enigmatic trickster, but if you go on quests often, you’ll get to know this being quite well.






\section{Spells}

Like equipment, spells are described in terms of tags. Spells have the following special tags:

\begin{easylist}
    # \textbf{Drain}: the minimum Essence expenditure required to cast the spell
    # \textbf{Element}: the spell has effects related to a particular element (e.g. fire, electricity, etc.)
    # \textbf{Sustained}: this spell may stay in effect as long as essence is committed to it. A caster cannot use the \move{Centering} move while sustaining a spell
    # \textbf{Exhausting}: this spell is quite difficult to cast; take 1 stun damage from drain when casting it
\end{easylist}


\subsection{Combat}
% usage:
% \spell{namee}{range & dmg & misc & essence}{description}

\spelldmg{Mana Bolt}{LoS & 1d8 & --- & 2}{Deals raw damage to a creature or spirit, ignores armor.}

% Mana Bolt: deals 1d8 damage (bypassing armor) to creatures or spirits at short/medium ranges. Tags: range s/m/l, dmg 1d8, ignores armor, essence 2

\spelldmg{Fire bolt}{LoS & 1d6 & fire & 1}{Deals fire damage and effects to a creature.}

\spelldmg{Lightning bolt}{LoS & 1d8+1 & shock & 2}{Deals shock damage and effects to a creature.}

\spelldmg{Acid Stream}{LoS & 1d8 & acid & 2}{Deals acid damage and effects to a creature or object.}

\spelldmg{Fireball}{LoS; AoE & 1d8+1 & fire & 3}{Deals fire damage and effects to all creatures and objects in a small area.}

\spelldmg{Manaball}{LoS; AoE & 1d8+1 & --- & 3}{Deals raw damage to all creature and spirits in a small area, ignores armor.}

\spelldmg{Knockout}{Touch & 1d8 & stun & 2}{Deals stun damage to a creature in touch range, ignores armor.}


\subsection{Detection/Perception}

\spell{Analyze Device}{touch & --- & 1}{Take +1 to your next move involving the device being analyzed, or learn what the device does.}

\spell{Clairvoyance}{LoS & sustained & 3}{When you \move{Check the Situation}, you can ask questions about a location you cannot see within the range of the spell.}

\spell{Combat Sense}{Touch & subtle, sustained & 2}{While you sustain this spell, you cannot be surprised, and take +1 to your first \move{Rock \& Roll} or \move{Null Sweat} move when combat starts.}

\spell{Mind Probe}{Touch & --- & 2}{When you touch the target, you get to know one thing as per Face’s \move{Razor Insight} move.}

\spell{Detect Life}{LoS & --- & 2}{When you look for living creatures in an area, take +2. (Anm.: weirder move iwie; Askennen lässt einen lebende Entitäten erkennen - then again: Askennen gibt einem eher das Gefühl, weniger eine konkrete Position und das hier vllt mehr so wie ein HUD und erfordert nicht den "Sichtwechsel" auf die Astralebene?}



\subsection{Health}

\spell{Antidote}{Touch & --- & 2}{When you touch the target, you halt poisons or other toxins affecting the target.}

\spell{Heal}{Touch & exhausting & X}{When you touch the target, heal a number of wounds equal to 1 + the amount of essence spent on the spell. (Anm.: das braucht noch eine Interaktion mit der Essenz des Ziels.)}

\spell{Antidote}{Touch & sustained & 2}{When you touch the target, choose a stat. Moves using that stat take +1 while the spell is sustained.}


\subsection{Illusion}

\spell{Chaotic World}{LoS & --- & 2}{When you cast this spell, you can hold 1 to spend on your or your teammate’s moves.}

\spell{Group Invisibility}{Touch & sustained & 1+X}{While you sustain this spell, you conceal a number of creatures equal to the essence spent from being seen by creatures or metahumans. X is the number of targets.}

\spell{Silence}{LoS; AoE & sustained & 1+X}{While you sustain this spell, all sound is silenced in the area you specify. Drain varies by the size of the area.}

\spell{Stink}{LoS; AoE & sustained & 2}{While you sustain this spell, all creatures in the affected area have to either leave the area or use air filters or take 1 stun.}


\subsection{Manipulation}

\spell{Mana Barrier}{LoS; AoE & sustained & 2}{While you sustain this spell, you create a barrier that blocks living creatures and spirits.}

\spell{Light}{LoS; AoE & sustained & 1}{While you sustain this spell, an area you specify is illuminated by bright light.}

\spell{Shadow}{LoS; AoE & sustained & 1}{Basically the opposite of the Light spell; completely unsurprising, too! While you sustain this spell, an area you specify is cloaked in arcane darkness.}

\spell{Fling}{Touch & --- & 1}{When you cast this spell on a target you are touching, you hurl the target out of melee range.}



\section{Spirits}

Spirits are the companions and tools of the Shaman, who summons them from the astral plane to perform services for him. Spirits have the following special tags: aspect: the spirit takes on the appearance of their domain, and is invisible in their domain unless it chooses to be seen. Elementals automatically gain this tag, otherwise it requires 1 spirit point.

desert: a spirit of the forbidding landscape of the deserts
earth: a spirit who dwells in the earth, caves, or landscape; earth spirits are widespread
elemental: these spirits represent the basic four elements, air, earth, fire, and water, and can be summoned anywhere.
engulf: the spirit may enclose a target in the ubstance of its domain, typically (but not always) dealing damage.
enthrall: use this stat for the Enthrall move
forest: a spirit of the forests, woods, or similar areas
generous: the spirit will perform one extra move; adding this tag costs 1 spirit point.
guard: use this stat for the Guard move
harm: use this stat for the Harm move
insubstantial: damage dealt and taken is halved
mentor: use this stat for the Mentor move
mountain: a spirit that dwell in foothills, crags, ridges, and other mountainous terrain
natural: natural spirits are spirits associated with particular domains (such as “city spirits” or “mountain spirits”).
plains: a spirit of the open plains, grasslands, fields, and farms
robust: the spirit is particularly resistant to damage; all damage rolls against it are [w]. Adding this tag costs 1 spirit point.
search: use this stat for the Search move
sky: a spirit of the open sky
storm: a spirit of storms and harsh weather
swamps: a spirits of the depths of the swamp, bayou, or wetlands
urban: a spirit dwelling in urban or developed lands, especially cities
water: a spirit of lake, river, or ocean
weakness (specify): the spirit has a weakness to a particular material or element which ignores insubstantiality, armor, and robustness. Adding this tag allows the free addition of another tag.
wild: this spirit has an extra spirit point, but the shaman must take -1 when he or she conjures it


\subsection{Spirit Moves}

Spirits are independent entities and have their own stats and moves. Their moves correspond to the harm, search, guard, enthrall, and mentor tags.

\paragraph{Harm:} When a spirit attacks someone or something, \roll{Harm}. On 10+, the spirit deals its damage. On 7-9, the spirit deals damage, but also takes damage. On 5 or lower the spirit only takes damage.

\paragraph{Search:} When the spirit attempts to locate individuals or items within its domain, \roll{Search}. On 10+, the spirit locates the item and can tell the Shaman where it is. On 7-9, the spirit can tell the shaman whether the item or person is within its domain, but not it’s specific location. Note: the GM and player should determine the search range for elementals.

\paragraph{Guard:} When a spirit stands in defense of its domain or inhabitants thereof, \roll{Guard}. On 10+, the spirit prevents damage or hostile effects from occurring. On 7-9, the spirit halves damage or the potency of a hostile effect.

\paragraph{Enthrall:} When a spirit attempts to control someone’s actions or thoughts, \roll{Enthrall}. If the target is:
    \begin{easylist}
        # An NPC: On a 10+, the spirit issues two instructions that the NPC must follow, or take 3 damage. On 7-9, the spirit may issue one instruction.
        # A PC: On a 10+, both of the following apply. On 7-9, only 1 applies:
            ## If the character complies, they mark XP
            ## If the character refuses, they must \move{Null Sweat}
    \end{easylist}

\paragraph{Mentor:} When a spirit imparts knowledge or truth, \roll{Mentor}. On 10+, the GM provides, in secret, a useful or interesting piece of information to the target. On 7-9, the GM provides an interesting piece of information.


\subsection{Spirit Examples}
There are 5 general spirit natures: Watchers simply observe and report. Teachers seek to instruct and guide others, but are reluctant to do harm. Protectors seek to defend their domain and its inhabitants, while Destroyers seek battle, blood, and vengeance. Finally, Seducer spirits desire control and devotion.

\subsubsection{Elementals}
Fire Elemental (destroyer, aspect, harm 2, search -1, guard 1, enthrall 1, mentor 0, dmg 1d10, armor 2, wounds 9)

Water Elemental (seducer, aspect, harm -1, search 2, guard 0, enthrall 3, mentor 1, dmg 1d4, armor 1, wounds 8)

Air Elemental (teacher, aspect, harm -2, search 2, guard 0, enthrall 1, mentor 2, dmg 1d4, armor 2, wounds 7)

Earth Elemental (protector, aspect, harm 1, search 2, guard 2, enthrall -1, mentor 0, dmg 1d8, armor 1, wounds 10)

\subsubsection{Nature Spirits}
Forest Protector (natural, forest, harm 1, search 1, guard 2, enthrall -1, mentor 0, dmg 1d8, aspect, armor 1, wounds 8)

Forest Watcher (natural, forest, search 3, guard 0, enthrall 1, mentor 1, aspect, armor 1, wounds 6, special: may not Harm)

Sky Watcher (natural, aspect, search 3, guard 0, enthrall 0, mentor 2, armor 1, wounds 6, special:may not Harm)

Urban Destroyer (natural, harm 2, search 0, guard 1, enthrall 1, mentor -1, dmg 1d10, armor 2, wounds 9)

Urban Seducer (natural, seducer, harm 0, search 2, guard 0, enthrall 2, mentor 1, dmg 1d4, armor 1, wounds 7)

Mountain Teacher (natural, aspect, harm 0, search 0, guard 2, enthrall 0, mentor 2, dmg 1d4, armor 1, wounds 8)

Swamp Destroyer (natural, aspect, harm 2, search 2, guard 0, enthrall 0, mentor -1, dmg 1d10, armor 2, wounds 9)

\subsection{Creating New Spirits}
The spirits above are just examples; the procedures that follow describe how to “build” a new spirit to suit your preferences.

Choose the spirit’s Type: elemental or nature.

Choose the spirit’s Domain, and record the base Armor and Wounds.

Choose the spirit’s Nature, and modify the basic spirit tags as needed.

Distribute 4 spirit points among spirit’s Moves, adjusting for the spirit’s purpose. No spirit move may have a modifier higher than +3.

Add additional tags if desired (see Other Spirit Tags).

\textbox{Example}{Pam is playing a Shaman named Chert and is developing the initial three spirits Chert can summon. Pam decides the first one will be a nature spirit of the forest, a protector of the dwindling unspoiled lands. With those decisions made, the spirit’s qualities so far are nature, forest, protector, armor 1, wounds 10, dmg 1d8, guard 1, enthrall -1. \\ \\
Pam also wants the spirit to blend in with the forest and to be an excellent guardian of its inhabitants. She spends one spirit point (out of 4) to gain the aspect tag, and then spends the remaining three to boost the Guard move twice, and the Harm move once. \\ \\
The final spirit looks like this: nature, forest, protector, harm 2, guard 3, search 0, enthrall -1, mentor 0, armor 1, wounds 10.}


\subsection{Spirit Types}
Elemental: these spirits represent the basic four elements, air, earth, fire, and water, and can be summoned anywhere.

Nature: natural spirits are spirits associated with particular domains (such as “city spirits” or “mountain spirits”). Natural spirits may enter other domains freely, but they can only be summoned within their own, and if they cross domains, there’s always a chance they attract unwanted attention from other spirits who don’t like intruders.

\subsubsection{Basic Spirit Tags}
Domain represents the spirit’s preferred environment or the area in which it may be summoned. A natural spirit summoned in its domain always has the generous tag. The domain of an elemental is considered to be the same as its element (though they gain no benefit from being within their domain).

Urban: spirits that dwell in urban or developed lands, especially cities

Plains: spirits that dwell in open plains, grasslands, open fields, and farms

Forest: spirits that dwell in forests, woods, and similar areas

Mountain: spirits that dwell in foothills, crags, ridges, and other mountainous terrain

Earth: spirits that dwell underground or in caves; the domains of earth spirits are widespread.

Deserts: spirits that dwell in the sere, forbidding landscape of the deserts

Sky: spirits dwelling in the open skies.

Storm: spirits of storm and disruption

Swamps: spirits who dwell where earth and water are one

Water: spirits of the water, be it lakes, rivers, or the open sea

There are two things to be aware of regarding domains. First, domains are relatively confined—a mountain spirit’s domain is not all mountains, nor even all of a specific mountain. Rather, it is usually a region with a radius of around 250 meters, within a mountainous region. Overlap among domains is possible, and the byzantine negotiations that take place between spirits defy understanding even by the most gifted shamans.

Also remember that multiple domains may exist within a larger area that seems uniform. In other words, city spirits (for example) are the only kind of spirit you’ll run across in a city—a park within a city may be the home of a forest spirit, and you may find a river spirit fighting to protect it’s home from polluted runoff in some industrial area.

Armor represents the spirit’s innate magical resistance to damage; spirit armor cannot be ignored, nor reduced by weapons. All spirits have 1 armor.

Wounds simply represent the spirit’s innate health; all spirits, by default, have 8 wounds.


\subsection{Spirit Nature}
Every spirit has a nature, which indicates its sense of purpose and the activities to which it is drawn. A spirit’s nature also affects its basic tags and moves (see Spirit Moves, below) in various ways.

Watcher spirits observe, find, and note. They are incapable of dealing harm to anyone or anything. Watcher spirits have the following modifiers: Search +2, Wounds -2, may not Harm.

Teacher spirits wish to inform and instruct, and find it difficult to inflict damage upon those they could otherwise teach. Teacher spirits have the following modifiers: Mentor +2, Harm -2, dmg 1d4.

Protector spirits preserve, defend, and support their domain. They are unconcerned with influencing intruders, preferring to throw them out instead. Protector spirits have the following modifiers: Guard +1, Enthrall -1, Wounds +2, dmg 1d8.

Destroyer spirits are warrior spirits who revel in combat and bloodletting. They are fearsome enemies, though somewhat limited in imagination. Destroyer spirits have the following modifiers: Harm +2, Mentor -1, Search -2, Wounds +1, Armor +1, dmg 1d10.

Seducer spirits wish to influence, to inspire love, and to acquire servants, though they do not typically enjoy directly harming others. Seducer spirits have the following modifiers: Enthrall +2, Harm -1, Wounds -1, dmg 1d4.


\subsection{Other Spirit Tags}
Robust: the spirit is particularly resistant to damage; all damage rolls against it are [w]. Adding this tag costs 1 spirit point.

Aspect: the spirit takes on the appearance of their domain, and is invisible in their domain unless it chooses to be seen. All spirits have this tag.

Generous: the spirit will perform one extra move; adding this tag costs 1 spirit point.

Insubstantial: damage dealt and taken is halved Weakness (specify): the spirit has a weakness to a particular material or element which ignores insubstantiality, armor, and robustness. Adding this tag allows the free addition of another tag.

Engulf: the spirit may enclose a target in the substance of its domain, typically (but not always) dealing damage.

Wild: this spirit has an extra spirit point, but the shaman must take -2 whenever he or she conjures it.



\section{Totems}

Shaman characters must select a totem, representing their connection to one of the great spirits. Alle Beschreibungen sind dem jeweiligen Eintrag in der Shadowhelix entnommen.

\paragraph{BÄR} ist stark, aber auch gelassen, freundlich, sanft und weise, solange er nicht verletzt oder über die Maßen gereizt wird, und dies gilt auch für die Schamanen, die ihm folgen.

\textbf{Boon:} reduce essence cost to conjure protector spirits by 1 (to a minimum of 1)

\textbf{Flaw:} when injured, roll 1d6. On 1 or 2, the shaman goes berserk).


\paragraph{KATZE} ist verstohlen, schlau und arrogant und spielt gerne mit ihren Opfern, indem sie sie bedroht, verfolgt und verwirrt - und dann zum Todesstoß ansetzt.

\textbf{Boon:} gain low-light vision; you cannot be surprised

\textbf{Flaw:} you cannot deal lethal damage to your enemy


\paragraph{COYOTE} wird in der Magietheorie als listiger Täuscher, Schwindler, Trickser und - manchmal hinterhältiger und geradezu bösartiger - Scherzbold unter den Schutzgeistern interpretiert. 

\textbf{Boon:} take +1 to conjure Teacher spirits

\textbf{Flaw:} destroyer spirits summoned lose 1 spirit point


\paragraph{HUND} ist treu gegenüber Freunden und Familie. Er kämpft verbissen, um sein Heim und seine Familie zu schützen. Personen, die freundlich ihm gegenüber sind, schützt er ebenso, allerdings tendiert er in diesem Streben zur Sturheit.

\textbf{Boon:} and take +1 to conjure protector spirits or city spirits

\textbf{Flaw:} your moves are glitched if you have left an ally behind or in danger


\paragraph{GATOR} und die Schamanen, die ihm folgen - gelten als gierig aber auch ausgesprochen geduldig, was von manchen als faul und träge missinterpretiert wird. Wird diese Geduld allerdings überstrapaziert, und er zu sehr gereizt, kommt die brutale und gnadenlose Seite von Alligator durch, und er wird zum kaltblütigen Killer. 

\textbf{Boon:} take +1 to conjure water spirits.

\textbf{Flaw:} You are exceptionally greedy


\paragraph{ADLER} ist stolz und einsam. Er fliegt hoch am Himmel und sieht alles, was sich unter ihm am Boden abspielt. Als Einzelgänger gelten Adler - und seine Schamanen - als Verteidiger der Reinheit der Natur, und gehen häufig hohe, persönliche Risiken im Kampf gegen Umweltverschmutzer und andere Übeltäter ein.

\textbf{Boon:} take +1 to conjure watcher spirits or air elementals

\textbf{Flaw:} you have an allergy to something relatively common, and take -1 ongoing when exposed


\paragraph{LÖWE} ist ein tapferer und starker Krieger. Er schätzt es wenn andere Leute seine Arbeit erledigen, während er seine Kräfte schont. Falls nötig setzte er jedoch auch selbst seine enormen Kräfte ein. Löwe ist eingebildet und hat höchste Anforderungen an sich selbst, weshalb er einen hohen Lebensstil pflegt und vonseiten Untergebenen höchste Loyalität und Respekt erwartet. 

\textbf{Boon:} take +1 to conjure protector or plains spirits

\textbf{Flaw:} Take -1 on Gut Checks


\paragraph{EULE} wird oft mit Weisheit, Prophezeiungen und/oder dem Tod assoziiert. Eulenschamanen haben ein besonderes Gespür für die Welt der Toten welches ihnen bei der Geisterbeschwörung zugute kommen kann. Desweiteren sind sie für ihre Wachsamkeit bekannt. 

\textbf{Boon:} gain low-light vision, take +1 to conjure teacher spirits

\textbf{Flaw:} Spells cost 1 more essence to cast in the day


\paragraph{WASCHBÄR} ist ein cleverer, neugieriger Bandit, der jede Falle überwinden kann um an den Köder zu kommen. Falls nötig lässt er sich auf einen Kampf ein, bevorzugt aber Strategien und Tricks um diesen zu umgehen. Am liebsten klaut er anderen ihren jeweils wertvollsten Besitz. 

\textbf{Boon:} and take +1 to conjure watcher spirits

\textbf{Flaw:} must \move{Null Sweat} to avoid letting his curiosity get to him


\paragraph{RATTE} findet man überall dort, wo Menschen leben und Abfall produzieren. Ratte ist ein Aasfresser der sich aus dem Müll das nimmt, was sie braucht. Ratte hasst offene Flächen und bleibt lieber im Verborgenen und vermeidet Kämpfe. Falls sie doch mal kämpft, tötet sie schnell und verschwindet dann wieder.

\textbf{Boon:} take +1 to conjure city spirits

\textbf{Flaw:} when combat starts, you must \move{Null Sweat}, or flee


\paragraph{RABE} ist auf der ganzen Welt als Bote des Unglücks verschrien und er täuscht und blendet andere. Dieser hinterlistige Blender lebt von den Hinterlassenschaften anderer, die meist durch Krieg und Chaos zustande kommen. Allerdings löst er diese Dinge nicht selbst aus, sondern ist ein Opportunist, der nur Gelegenheiten ergreift, die sich ihm bieten. Er liebt gutes Essen und schlägt daher selten eine Einladung aus. Zudem ist er ein Vetter von Kojote, der ebenso zum Schwindel tendiert. Allerdings teilt Rabe nicht dessen Hang zu Düsternis und Gemetzel.

\textbf{Boon:} take +1 to conjure watcher spirits

\textbf{Flaw:} you must take advantage of others’ misfortune when you can


\paragraph{HAI} Gnadenlos und kalt jagt Hai seine Ziele. Jeder, der am Meer lebt, kennt seine Macht und weiß, dass Hai gnadenlos ist und vom Blut seiner Feinde in wilde Raserei versetzt wird. Dementsprechend sind auch seine Anhänger zumeist ruhelose Wanderer und wilde, tödliche Kämpfer, die nicht ruhen, bis ihr Ziel tot ist.

\textbf{Boon:} take +1 to conjure destroyer spirits

\textbf{Flaw:} when injured, roll 1d6: on <4, the shaman goes berserk


\paragraph{SCHLANGE} Schlange gilt den magisch Begabten rund um den Globus in erster Linie als Hüter, Bewahrer und Vermittler von Wissen. Anhänger von Schlange sind dementsprechend häufig magische Lehrer und Tutoren, die ihr Wissen oft gegen alles tauschen, was sie im Gegenzug dafür erhalten können.

\textbf{Boon:} and take +1 to conjure seducer spirits 

\textbf{Flaw:} take -1 ongoing to \move{Rock \& Roll}


\paragraph{WOLF} gilt in vielen Kulturen rund um den Globus als Krieger und Jäger, und ein Sprichwort besagt, dass er jeden Kampf gewinnen würde - außer seinen letzten. Er ist gegenüber den Mitgliedern seines Rudels absolut loyal, was auch auf seine Anhänger abfärbt, die wie ihr Schutzpatron stets bereit sind, für ihre Freunde und die erweiterte Familie notfalls bis zum Tod zu kämpfen, da sie diese jeweils als Angehörige des Rudels betrachten.

\textbf{Boon:} take +1 to conjure protector spirits 

\textbf{Flaw:} you must \move{Null Sweat} to retreat from combat

