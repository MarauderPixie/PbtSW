\chapter{Gear} \label{Gear}

\epigraph{\textit{We're in the minority; Runners who are not jacked, rigged or wakened. We live by our guts and wits.}}{-- Jazzman Harker, Shadowrunner}

In this section you’ll find example equipment (weapons, cyberdecks, vehicles, etc.) available in the Sixth World. This isn’t an exhaustive list of what’s available; rather, they’re just samples of some classic items to help you get playing quickly.


\section{Tags} \label{tags}

Equipment — like many items in Sixth World — is described in terms of tags, which are short keywords that indicate various capabilities or qualities. Certain tags apply to multiple kinds of equipment (such as obvious, supply, or armor). Tags that only apply to specific kinds of equipment are described in the listing of that kind of item. The following tags apply to multiple types of equipment.

\fullbox{Tags}{
    \textbf{2-hand} & this item must be used with both hands \\
    \textbf{armor +n} & grants a +n bonus to existing armor \\
    \textbf{armor n} & grants n Armor (armor rating for vehicles or drones) \\
    \textbf{arcane} & can only be used by magical archetypes \\
    \textbf{area} & affects multiple targets \\
    \textbf{avail} & the availability of the item on the shadow markets \\    
    \textbf{+bonus} & grants a bonus to a particular move; e.g. +1 to \move{Null Sweat} \\
    \textbf{conceal} & this weapon or item is easily hidden and will not be spotted by enemies \\
    \textbf{damage n} & the amount of damage a weapon or other item deals; abbreviated dmg \\
    \textbf{heal n} & restores n wounds \\
    \textbf{ignores armor} & bypasses the target’s armor \\
    \textbf{loud} & noisy and audible to anyone with functioning hearing; for weapons, it means the weapon cannot be suppressed \\
    \textbf{messy} & deals damage in a particularly gruesome way \\
    \textbf{obvious} & cannot be concealed, or is immediately visible to any observer \\
    \textbf{range} & the range(s) at which the weapon or other attack is effective. Ranges are touch (t), close (c), short (s), medium (m), and long (l) \\
    \textbf{shock} & the weapon deals electrical shock \\
    \textbf{special (description)} & if the effect of the item requires explanation, use this tag \\
    \textbf{stun} & this weapon or attack deals Stun damage only \\
    \textbf{subtle} & not easily noticed (as opposed to conceal, which means it is unnoticeable) \\
    \textbf{Supply n} & the amount of supplies or uses you can get out of an item. Each use of the item consumes 1 supply (unless otherwise stated)
}



\section{Weapons}

The tags below apply to weapons. Feel free to customize the example equipment with these tags (subject to GM approval) to create your own gear, or recreate classic gear from Shadowrun.

\fullbox{Weapon Tags}{
    \textbf{auto} & this weapon can fire in full auto mode; abbreviated FA \\
    \textbf{burst} & this weapon can fire in burst mode, mark off 1 additional Ammo to deal +1 damage; abbreviated BF \\
    \textbf{chem} & this weapon delivers a chemical agent of some kind to the target; depending on the delivery mechanism, armor may be ignored \\
    \textbf{forceful} & when this weapon deals damage, it also deals 1 stun \\
    \textbf{fuzed} & this weapon cannot be used at less than the shortest range increment listed \\
    \textbf{reload} & after using this weapon, it takes more than a moment to reload it \\
    \textbf{semiauto} & this weapon fires one shot every time the trigger is pulled; abbreviated SA \\
    \textbf{stabilized} & this weapon cannot be fired except from a bipod, tripod, or supported position \\
    \textbf{suppressed} & this weapon makes little to no noise when fired \\
    \textbf{thrown} & this item can be throw. If thrown, the range is short \\
    \textbf{vented} & the weapon has recoil venting, granting +1 to Suppression Fire
}


\subsection{Weapon Conversions}

[\textit{Don't spend too much attention to this section for now, it's probably gonna change quite a lot anyway.}]

The (online-only) Simple Edition mentioned in the preamble provides quite a few items which will also be listed in this chapter. The Digest Edition 1.3 on the other hand provides some guidelines to convert items from Shadowrun 5 sourcebooks. Additionally, I aim to provide further guidelines to convert Shadowrun 6 gear.

\subsubsection{From Shadowrun 5}
Rather than reproduce a listing of shadowrun weapons here, or provide an overly generic "heavy pistols do this, and assault rifles do this", the following guidelines should help you convert weaponry from Shadowrun 4th or 5th edition core rulebooks. Keep in mind these are guidelines and not hard and fast rules; feel free to adjust weapons by hand to get them "just so".

\begin{easylist}
     # \textbf{Weapon Type:} Self-explanatory
     # \textbf{Weapon Range:} generally, melee weapons are range c, pistols and SMGs either range s or range s/m, and rifles and other longarms are tagged range s/m/l. Exceptions to this include  sniper rifles, which are optimal at long range (range l) only, and heavy weapons which generally are best tagged range m/l.
     # \textbf{Damage Type:} stun weapons should get the stun tag
     # \textbf{Damage Value:} the damage value of a weapon in Sixth World should be roughly one-half the damage value of the weapon as listed in the Shadowrun core books. Damage can either be fixed value or dice-based. For weapons that incorporate Strength into their damage ratings, you can add the character’s Hard rating (for instance, 2+Hard dmg).
     # \textbf{Armor Piercing:} divide the AP value in the Shadowrun core books by 2 to get the Sixth World equivalent AP value.
     # \textbf{Ammo:} divide the weapon capacity listed in the Shadowrun core books by 5 to arrive at the ammo value for the Sixth World equivalent weapon (note that some weapons may require some adjustment by hand on this point, and single-shot weapons should have ammo 1).
     # \textbf{Other Tags:} assign other tags as appropriate (such as firing modes, whether they require two hands, and so forth) to round out or customize the weapon.
     # \textbf{Cost:} dividing the cost by between 2 and 4 will generate an appropriate price for Sixth World use.
\end{easylist}

\textbox{Example}{The Ares Predator V does 8P damage according to the Shadowrun Fifth Edition core book. In Sixth World, the damage would be either 4, or a dice value approximating that (e.g., 1d8).}

\subsubsection{From Shadowrun 6}
\textit{yet about to come}


\subsection{Melee Weapons}

\wpnbox{Melee Weapons}{
    Staff        & stun  & 1d6+2 & --- & 100¥ \\
    Combat Axe   & messy & 1d6+2 & --- & 1.250¥ \\
    Combat Knife & ---   & 2d4b  & --- & 300¥ \\
    Fists/Feet   & stun  & 1d6   & --- & --- \\
    Katana       & ---   & 2d6b  & --- & 1.000¥ \\
    Spiked Glove & ---   & 1d4+1 & --- & 50¥ \\
    Stun Baton   & stun, shock, ignores armor & 1d4 & --- & 750¥ \\
    Tomahawk     & messy, thrown & 1d6 & --- & 200¥ 
}


\subsection{Hold-Out Pistols}

A pocket pistol is any small, pocket-sized semi-automatic pistol (or less commonly referencing either derringers, or small revolvers), and is suitable for concealed carry in either a coat, jacket, or trouser pocket.

\wpnbox{Hold-Outs}{
    Streetline Special & SA, conceal    & 2d4b  & 3 & 250¥ \\
    Fichetti Needler   & conceal        & 2d4b  & 3 & 400¥ \\
    Walther PP         & SA/BF, conceal & 1d4+1 & 1 & 325¥
}


\subsection{Light Pistols}

\wpnbox{Light Pistols}{
    Colt L36 & SA, conceal & 1d6 & 3 & 500¥ \\
    Beretta 101T & SA/BF, subtle & 1d6 & 2 & 450¥ \\
    Ares Lightfire 70 & SA, conceal & 1d6 & 3 & 350¥
}


\subsection{Heavy Pistols}

\wpnbox{Heavy Pistols}{
    Ares Predator       & SA & 1d8+1  & 3 & 675¥ \\
    Colt Manhunter      & SA/BF & 1d8 & 3 & 560¥ \\
    Ruger Super Warhawk & SA, loud & 1d10 & 2 & 560¥ \\
    Browning Max Power  & SA & 2d8b & 3 & 675¥
}


\subsection{Submachine Guns}

\wpnbox{Submachine Guns}{
    HK227  & SA/BF, suppressed & 1d8 & 4 & 900¥ \\
    AK-97K & SA/FA & 1d8 & 3 & 1.000¥ \\
    Ingram Smartgun & BF/FA & 1d6+1 & 3 & 950¥
}


\subsection{Assault Rifles}

\wpnbox{Assault Rifles}{
    AK-97      & SA/FA, 2-hand, obvious & 1d10 & 3 & 800¥ \\
    Ares Alpha & SA/BF/FA, 2-hand, obvious & 2d8b & 4 & 1.150¥ \\
    Colt M22A2 & SA/BF, 2-hand, obvious & 1d10 & 3 & 850¥ \\
    FN-HAR     & SA/BF, 2-hand, obvious, loud & 2d8b & 3 & 1.050¥
}


\subsection{Shotguns}

\wpnbox{Shotguns}{
    Remington 990 & SA, obvious, loud, forceful & 1d10+1 & 2 & 750¥ \\
    Enfield AS7   & SA/BF, 2-hand, obvious, loud, forceful & 1d10 & 3 & 900¥
}

\subsection{Sniper Rifles}

\wpnbox{Sniper Rifles}{
    Ranger Arms    & SA, 2-hand & 1d10+1 & 3 & 1.150¥ \\
    Walther WA2100 & SA, 2-hand & 1d12   & 4 & 1.100¥
}


\subsection{Heavy Weapons}

\wpnbox{Heavy Weapons}{
    Ingram Valiant LMG & FA, 2-hand, loud, stabilize, obvious, loud, messy & 1d12 & 4 & 2.000¥ \\
    Stoner M202 HMG & BF/FA, 2-hand, loud, stabilize, obvious, loud, messy & 2d10b & 3 & 2.500¥
}


\subsection{Special Weapons}

\wpnbox{Speacial Weapons}{
    Compound Bow & 2-hand & 1d6+1 & 1 & 500¥ \\
    Narcoject Rifle & stun, suppressed, chem, slow & 1d8+1 & 1 & 700¥ \\
    Taser & stun, shock, slow & 1d8 & --- & 500¥ \\
    Crossbow & 2-hand, suppressed & 1d6 & 1 & 400¥
}


\subsection{Grenades}

\begin{strip}
    \textbox{Grenades}{
        \rowcolors{1}{gray!30!white}{}
        \begin{tabular}{m{0.15\textwidth} m{0.5\textwidth} m{0.1\textwidth} m{0.15\textwidth}}
            \textbf{\textsf{\textcolor{purplefont}{TYPE}}} & 
            \textbf{\textsf{\textcolor{purplefont}{TAGS}}} &
            \textbf{\textsf{\textcolor{purplefont}{DMG}}}  &
            \textbf{\textsf{\textcolor{purplefont}{PRICE}}}  \\
            EMP   & thrown, area, shock; disables electronics & --- & 95¥ \\
            Flash & thrown, area, stun, +1 to \move{Rock \& Roll}/\move{Null Sweat} & --- & 125¥ \\
            Frag  & thrown, area, forceful & 2d6b & 100¥ \\
            Incendiary & thrown, area, burn & 2d6b & 75¥ \\
            Smoke & thrown, area, +1 to \move{Null Sweat} & --- & 40¥ \\
            Stun  & thrown, area, stun & 2d6b & 100¥
        \end{tabular}
    }
\end{strip}


\subsection{Armor}

Armor provides protection against incoming attack, reducing the damage dealt by the armor value. Armor of the same type (e.g inherent) does not stack. Armor of differing types can stack. Armor has the following unique tags:
\textbox{Armor Types}{
    \rowcolors{1}{gray!30!white}{}
    \begin{tabular}{m{0.2\textwidth} m{0.7\textwidth}}
        \textbf{inherent} & this armor is either implanted, or occurs naturally. Cyberware armor is inherent armor. \\
        \textbf{worn} & this armor is worn on the body \\
        \textbf{mystic} & this armor is magical in nature
    \end{tabular}
}

\subsubsection{Sample Armor}
\armbox{Armor Examples}{
    Lined Coat     & 2 & obvious & worn & 600¥ \\
    Ballistic Vest & 2 & obvious & worn & 750¥ \\
    Armorweave Professional Wear & 1 & subtle & worn & 1.500¥ \\
    Leather Armor  & 1 & subtle  & worn & 250¥ \\
    Armor Charm    & +1 & conceal & mystic & 400¥ \\
    Light Armor Jacket & 1 & subtle & worn & 850¥ \\
    Combat Armor   & 3 & obvious & worn & 2.500¥  \\
    Form-fitting Armor & 1 & conceal & worn & 550¥ \\
    Riot Shield    & 2 & occupies one hand & worn & 700¥
}


\section{Cyberware}

Cyberware works slightly differently from other equipment. Instead of simply being something that has some tags or stats, each piece of cyberware provides new moves or modifies existing moves, based on the augmentation’s function.

Adding cyberware costs essence, which does have a significant effect on magic users, so magic archetypes who choose cyberware do so at the cost of their magical potency.

When you have downtime, you may elect to have cyberware installed. Installation and recovery from cybersurgery takes a number of days equal to 3x Essence cost of the implant.

\subsection{Containers}
The listed essence costs assume that the augmentation is directly implanted into the body. However, full-replacement cyberware (such as cybereyes, cyberears, and cyberarms) have the capacity to hold other implants without costing additional essence. Each of these items can hold additional augmentations equal to 1 + their Essence cost (for example, cybereyes cost 1 essence, and therefore can contain up to 2 essence worth of additional augmentations).


\subsection{Headware}
% FULL REPLACEMENTS

\subsubsection{Vision Enhancements}
\paragraph{Cybereyes -} Capacity for 2 essence worth of vision enhancement augmentations. Include low-light vision system. Cost: 1 essence, 4.000¥.

\begin{easylist}
    # \textbf{Low-light Vision:} you can see in the dark, as long as there’s at least some light, not complete darkness. Included in Cybereyes for no extra money or essence. Cost: 1 essence, 1,000¥.    
    # \textbf{Thermographic Vision:} when you Check the Situation, you may ask one additional question from the list. Cost: 1 essence, 1,500¥.    
    # \textbf{Recorder:} when you use this device, you gain 1 hold to spend on researching the individual, location, or event you recorded. Cost: 1 essence, 1,000¥.    
    # \textbf{Flare Compensator:} you do not suffer the effects of bright light (such as flash-bang grenades). Cost: 1 essence, 1,000¥.    
    # \textbf{Smartlink:} when you \move{Rock \& Roll}, you never graze the target. Additionally, on 10+, ignore 1 armor. Cost: 1 essence, 2,000¥.
\end{easylist}
    

\subsubsection{Auditory Enhancement}
\paragraph{Cyberears -} Capacity for 2 essence worth of hearing enhancement or auditory augmentations. Cost: 1 essence, 4.000¥.

\begin{easylist}
    # \textbf{Hearing Enhancement:} when you Check the Situation, you may ask one additional question. Cost: 1 essence, 4,000¥.
    # \textbf{Sound Damper:} you do not suffer the effects of loud noises. Cost: 1 essence, 2,000¥.
    # \textbf{Ultrasound system:} you can "see" in total darkness, or even while blind. Ultrasound is detectable if someone is listening for it, however. You can also determine the size of an enclosed space automatically. Cost: 1 essence, 10,000¥.    
    # \textbf{Balance Augmentation:} when performing an acrobatic or tricky maneuver, you are boosted. Cost: 1 essence, 6,000¥.
\end{easylist}


\subsubsection{Other}
\paragraph{Headware device -} you have a device built in to your head. Cost: 1 essence, device cost + 2,000¥.

\paragraph{Control Rig -} you can interface with vehicles and drones and control them directly. Control rigs include a datajack. Cost: 2 essence, 40,000¥.

\paragraph{Synaptic Hardening -} you gain +1 armor against Matrix attacks. Cost: 2 essence, 10,000¥.

\paragraph{Voice Modulator -} you can control your voice perfectly, imitating any sound you’ve heard or any voice you’ve heard. Cost: 1 essence, 6,500¥.


\subsection{Bodyware}

\paragraph{Bone Lacing -} when you make an unarmed attack, you deal lethal damage. Additionally, you take +1 to \move{Gut Check}. Cost: 2 essence, 15,000¥.

\paragraph{Cyberarm -} Capacity for 3 essence of additional implants. Deal +1 damage in melee. This replacement has the obvious tag by default. Increase the cost by 5,000¥ to remove the obvious tag. Cost: 2 essence, 15,000¥.

\paragraph{Cybergun -} you have a permanently implanted weapon. Choose a hold-out pistol or light pistol. This weapon gains the conceal and reload tags. Cost: 2 essence, 2,000¥ (hold out) or 3,900¥ (light pistol).

\paragraph{Datajack -} you are able to interface with a multitude of electronic devices. Datajacks can also be installed in any full-replacement item. Cost: 1 essence, 1,000¥.

\paragraph{Dermal Plating -} you gain +1 armor. This armor stacks with other armor, and has the obvious tag. Cost: 2 essence, 3,000¥.

\paragraph{Gyrostabilizer -} take +1 forward to Suppression Fire. Must be installed in a cyberarm. Cost: 3,000¥.

\paragraph{Hand Razors -} you have a permanently implanted weapon equivalent to a Combat Knife. This weapon can be extended or retracted at your discretion, and gains the conceal tag. Cost: 1 essence, 2,500¥.

\paragraph{Skillwires -} \label{skillwires} when you have an appropriate skillsoft, take +1 ongoing to Drop Science. Additionally, you may \roll{Skilled} to \move{Null Sweat} or Check the Situation. Cost: 2 essence, 10,000¥.

\paragraph{Wired Reflexes -} while active, when you fail a roll and would take damage or be attacked, \roll{Sharp}. On 10+, the damage or effect is halved. On 7-9, you take the damage, but boost your next move. Cost: 3 essence, 50,000¥.



\section{Cyberdecks}

Cyberdecks are the essential tool of the Decker. They are the Decker’s connection to the Matrix. Cyberdecks have the following special tags:

\textbox{Cyberdeck Stats}{
    \rowcolors{1}{gray!30!white}{}
    \begin{tabular}{m{0.2\textwidth} m{0.7\textwidth}}
        \textbf{System} & the power and system stability of the deck; this is the equivalent of the deck’s wounds. A deck whose System is reduced to zero is fried, and can’t be used until repaired  \\
        \textbf{Mask} & the stealthiness of a cyberdeck \\
        \textbf{Hardening} & the deck’s resistance to damage; this acts as armor protecting the decker \\
        \textbf{Storage} & the deck’s capacity for loaded programs
    \end{tabular}
}

\subsection{Converting Cyberdecks}

\textit{Some space for rules to come.}


\subsection{Example Decks}

\deckbox{Cyberdecks}{
    Allegiance Alpha & 5 & 1 & 1 & 5 & 25.000¥ \\
    Fuchi Cyber-4    & 6 & 2 & 1 & 6 & 50.000¥ \\
    Fuchi Cyber-7    & 6 & 1 & 2 & 6 & 75.000¥ \\
    Fairlight Excalibur & 8 & 2 & 1 & 8 & 100.000¥
}


\subsection{Programs}

Programs are the tools and weapons of the decker. Programs can modify a deck’s attributes, allow a decker to deal damage, or offer special moves to a decker. Programs must be loaded into deck storage to be running; each program has a size rating indicating how much storage the program occupies.

\subsubsection{Utilities}
\paragraph{Analyze:} when you examine a node, \roll{Skilled}. On 10+, hold 2 toward hacking the node. On 7-9, hold 1. Size 3, 750¥.

\paragraph{Decrypt:} take +1 forward to hacking Datastore nodes. Size 3, 750¥.

\paragraph{Interface:} take +1 forward to hacking Control nodes. Size 3, 750¥.

\paragraph{Interference:} slows hostile program alarm triggers. Size 2, 500¥.

\paragraph{Patch:} when you attempt to restore system stability to your deck, \roll{Skilled}. On 10+, restore 2 System to your deck. On 7-9, restore 1. Size 2, 500¥.

\paragraph{Reflect:} when you take damage in the matrix, \roll{Steady}. On 10+, redirect the damage to a matrix program or node of your choice. On 7-9, redirect half the damage. Size 3, 750¥.

\paragraph{Stealth:} your deck gains +1 Mask while this program is running. Size 2, 500¥.

\subsubsection{Combat}
\paragraph{Armor:} your deck gains +1 Hardening while this program is running. Size 2, 500¥.

\paragraph{Stunner:} deal 1d4 damage in matrix combat. Size 1, 250¥.

\paragraph{Hammer:} deal 1d6 damage in matrix combat. Size 2, 500¥.

\paragraph{Black Hammer:} deal 1d8 damage in matrix combat. Size 3, 750¥.

\paragraph{Static:} when you \move{Rock \& Roll} in the matrix, you may choose to forgo dealing damage, and instead hold 2 to grant to any ally’s roll. You can only spend 1 hold at a time. Size 3, 750¥.


\section{Vehicles}

Vehicles have the following special tags (or stats, rly, innit).

\textbox{Vehicle Stats}{
    \rowcolors{1}{gray!30!white}{}
    \begin{tabular}{m{0.2\textwidth} m{0.7\textwidth}}
        \textbf{Power} & the vehicle’s horsepower, speed, and acceleration \\
        \textbf{Armor} & the vehicle or drone’s armor rating \\
        \textbf{Frame} & the vehicle’s or drone’s resilience. This is the equivalent of a vehicle’s wounds. Remember that small arms deal half damage to vehicles \\
        \textbf{Sensor} & thethe quality of the vehicle’s sensors (used when Checking the Situation while driving or piloting the vehicle) \\
        \textbf{Seats} & the number of people who can normally occupy the vehicle, including the driver or pilot \\
        \textbf{Fuel} & fuel or battery capacity
    \end{tabular}
}


\subsection{Bikes}

\vroombox{Bikes}{
    Dodge Scoot     & 1 & 1 & 0 & 4 & 0 & 4 & 1.800¥ \\
    Yamaha Rapier   & 1 & 2 & 0 & 4 & 1 & 4 & 9.500¥ \\
    Harley Scorpion & 2 & 2 & 1 & 7 & 1 & 2 & 17.500¥
}

\subsection{Cars \& Trucks}

\vroombox{Cars \& Trucks}{
    C-N Jackrabbit   & 3 & 1 & 0 & 6 & 0 & 3 & 10.000¥ \\
    Ford Americar    & 4 & 1 & 0 & 8 & 1 & 3 & 16.000¥ \\
    Eurocar Westwind & 6 & 3 & 1 & 9 & 1 & 3 & 200.000¥ \\
    GMC Bulldog      & 8 & 2 & 1 & 9 & 1 & 3 & 45.000¥ \\
    Ares Roadmaster  & 6 & 3 & 2 & 11 & 1 & 2 & 52.000¥
}


\subsection{Drones}

Drones have most of the same qualities as vehicles, although they lack the seats tag, and replace it with the following:

Tactical: the quality of the drone’s tactical expert system, which comes into play when the drone is in autonomous mode. Abbreviated tac.
Armed drones also use the damage tag, indicating the damage of their built-in weapon systems.

\subsubsection{Ground Drones}
\dronebox{Ground Drones}{
    Aztechnology Crawler & 1 & 5 & 0 & 2 & 0 & ---  & 3 & 4.000¥ \\
    GM-Nissan Doberman   & 1 & 7 & 1 & 1 & 1 & 1d6  & 3 & 5.000¥ \\
    Steel Lynx           & 1 & 9 & 2 & 1 & 2 & 2d6b & 2 & 9.500¥
}

\subsubsection{Airborne Drones}
\dronebox{Airborne Drones}{
    Lockheed Optic-X & 1 & 2 & 0 & 2 & 1 & ---  & 2 & 12.500¥ \\
    MCT Roto-Drone   & 2 & 5 & 0 & 1 & 1 & 2d4b & 2 & 15.750¥ \\
    CD Dalmatian     & 1 & 8 & 1 & 0 & 2 & 1d8  & 3 & 22.000¥
}


\section{Other Equipment}

\subsection{Skillsofts}
Skillsofts are data chips that allow an individual to "slot" particular skills into their \refname{skillwire} system, gaining the benefit of the prerecorded knowledge. When you purchase a skillsoft, you must specify what skill area it covers. The following examples are not exhaustive, but should give an idea of what one Skillsoft covers. Skillsofts cannot be used without Skillwires.

\paragraph{Skillsoft -} choose a skill: Biotech, Electronics, Etiquette, Survival, Investigation, Mechanics, Academic Discipline, Pilot, Language; cost: 1,000¥

\textbf{Note:} You must specify a specific area for Etiquette, Pilot and Language and Academic Discipline. \textit{Language (Russian)}, for example, or \textit{Academia - anything you can have, say, a bachelors degree in}.


\subsection{Drugs}
Costs listed below are per dose.

\drugs{Drugs}{
    Bliss & take +1 to \move{Gut Check} & 2h & 15¥ \\
    Cram & take +1 to \move{Null Sweat} & 3h & 10¥ \\
    Deepweed & user can perceive astrally & 1h & 400¥ \\
    Jazz & take +2 to \move{Null Sweat} & 30min & 75¥ \\ 
    Kamikaze & take +1 to \move{Rock \& Roll} and \move{Gut Check} & 1h & 100¥ \\
    Long Haul & you can go without sleep for four days with no consequence & 4d & 50¥ \\
    Nitro & take +2 to \move{Rock \& Roll} and +1 to \move{Gut Check} & 30min & 75¥ \\
    Novacoke & take +1 to Make ‘em Sweat and Check the Situation & 2h & 10¥ \\
    Psyche & take +1 to Drop Science & 3h & 200¥ \\
    Zen & take +1 to \move{Null Sweat} & 30min & 5¥ \\
    BTLs & allow you to experience almost anything virtually & depends & 20-100¥
}


\subsection{Miscellaneous}

\miscbox{Miscellaneous}{
    Medic Patch     & 1 & heal 2 & 500¥ \\ 
    Stimulant Patch & 1 & take +2 to next move, take 1 stun afterwards & 175¥ \\ 
    Antidote Patch  & 1 & halts poison damage & 200¥ \\ 
    Trauma Patch    & 1 & +1 to First Aid move & 300¥ \\ 
    Quik-H4x Kit    & 4 & bypasses low-grade security locks/electronic devices & 350¥ \\ 
    Spy Kit         & 4 & +1 to Citation Needed or Check the Situation (assuming bugs haven’t been found) & 4.000¥ \\ 
    Counter-surveillance Kit & 4 & +1 to Check the Situation to search for bugs & 3.000¥ \\ 
    Infiltrator’s Kit & 4 & +1 to \move{Null Sweat} to infiltrate or avoid detection & 1.000¥
}



\section{Magical Supplies}

\subsection{Foci}
A focus is a mundane item that has been imbued with an astral construct. When used by someone to which it is attuned, a focus helps them channel to astral power and greatly enhances their abilities.

\subsubsection{Attuning}
Before a focus can be used, the user must attune themselves to it. To do so, they must invest at least one point of essence into the focus. Essence committed in this fashion remains spent until the user de-attunes themselves from the focus, or the focus is destroyed, at which point the essence is recovered.

\subsubsection{Types of Foci}
\paragraph{Spell Focus -} enhances the casting of a specific spell. When attuned, the mage using the spell focus has \textsf{\textbf{hold}} equal to the Essence spent attuning the focus. Spend this hold toward casting that specific spell.

\paragraph{Spirit Focus -} enhances the summoning of a specific type of spirit. When attuned, the shaman has \textsf{\textbf{hold}} equal to the essence invested in the focus toward summoning that specific spirit type.

\paragraph{Weapon Focus -} primarily used by adepts. When attuned to a weapon focus, the adept using it has \textsf{\textbf{hold}} equal to the invested Essence to spend on the \move{Rock \& Roll} move or on dealing damage.

\subsubsection{Creating a Focus}
Although foci may be purchased from talismongers, street contacts and other sources, sometimes a magic user wishes to create one of their own. To do so, the user must spend two days researching and preparing the object, at the end of which they make the Imbue Focus move:

\paragraph{Imbue Focus -} When you imbue astral power into an object to create a focus, \roll{Skilled}. On 10+, the focus is created normally. On 7-9, the focus is weakly imbued, and requires one additional Essence point to attune (this essence does not count toward the Hold granted by the focus).


\subsection{Fetishes}

Fetishes are essentially one-shot magical supplies — small mundane objects imbued with the structure and energy to cast a spell or to summon a spirit, needing only to be triggered by the mage or shaman.

\subsubsection{Infusing}
To create a fetish, the mage or shaman decides what spell or spirit to place into the fetish, and then infuses the fetish with power, spending the Essence required for the spell or the essence they wish to provide to the spirit. Essence invested in a fetish in this manner remains spent until the fetish is used, at which point it immediately returns.

\subsubsection{Activating a Fetish}
Normally, to cast a spell or to summon a spirit, the mage or shaman must make the Cast a Spell or Conjure moves. With a fetish, this is no longer the case: instead, they can simply declare that they’re using it (making any other moves that the fiction would dictate; \move{Null Sweat} for instance). Once triggered, the stored spell or summoning is immediately carried out. The fetish is good for a single use, after which it crumbles to dust.
