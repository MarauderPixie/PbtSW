\chapter{Loose Ends}

Laufende (Meta-)Plots, Charakterprogression, noch ausstehende Dinge. Vor allem Dinge, die bereits passiert \textit{sind} oder \textit{definitiv} passieren \textit{werden}, bzw. die meine Spieler*innen direkt betreffen. Das Kapitel \textbf{R\&D} ist dagegen für eher generelle oder szenario-spezifische Sachen da.




\section{Loose indeed}


\subsection{Progress}

Was wollen meine Runner trainieren, lernen, worin wollen sie besser werden?

\begin{easylist}
    # Frosty hat \textit{Stille} und \textit{Gerät analysieren} gelernt - aber Leonie weiß noch nicht bescheid
    # Tusk kann mit dem \textit{Initiationstraining} beginnen und trainiert außerdem den Umgang mit \skill{Exotische Waffen: Monofilamentpeitsche}
    # Yu möchte \skill{Edge}... trainieren? 
    # Rude und Yu haben bereits, glaube ich, \skill{Wissen: Trollgesellschaft (Seattle)} (Rude) und \skill{Spezialisierung: Pistolen} (Yu) gelernt
\end{easylist}




\subsection{Frostburns Van}

Frosty hat ihren Van (einen Ford F-151; \textit{Toyota Gopher} in den Regeln) vor vier Wochen in Downtown geparkt, weil sie von einem Errant angehalten wurde und meinte, gesehen zu haben, dass dieser den Van verwanzt hat. Als sie Yu gebeten hat, den Van einzusammeln, war er nicht mehr da. Die Vermutung liegt nahe, dass er einfach abgeschleppt wurde.




\subsection{Yu's Wohnung}

Yu's Wohnung wurde verwanzt. Der Einbruch machte zwar einen durchaus professionellen Eindruck, allerdings wurde ziemlich ranzige Tech verwendet, was den Runnern wiederum äußerst widersprüchlich erscheint. Neuron konnte die Wanze bis nach \textit{Glow City} in den \textit{Redmond Barrens} zurückverfolgen.

\subsubsection{Behind the Scenes}

Der Hausmeisterei-Dienstleister von Yu's Wohngebäude ist die \textbf{Slupinski Gebäudewartung}(?), die als solche keine großen Probleme hatte, in die Wohnung einzudringen.




\subsection{Meta-Timeline}

Am 12.09. geht der Jackpoint down, dann ist bis zum 01.10. Stille. Detroits ``Befreiung'' findet in der Zeit statt. Huh. Tatsächlich vllt sogar schon am 10.09., zumindest ist Vogels Statement von dann.

Zurück zum 10.09.: Ares hat also einen neuen CEO und ist jetzt in Atlanta. Zwei Tage später, nach den harten Anschuldigungen Vogels, kündigen die UCAS den Business Recognition Accord auf.

Der Konzerngerichtshof braucht 16 Tage, um darauf zu reagieren - u.a. damit, dass Polizei- und Medizindienstleister regulär weitermachen können sollen (ob das bis dahin so war, steht nicht in \textit{Blackout}.

Der erste Blackout ist ca. am 1. November in Philadelphia, weitere einen Tag später. 30T3N spielt auch im November. 


\subsubsection{Der UCrASh}

Der Zusammenbruch der UCAS (bzw. der \textit{\textbf{UCrASh}} findet während der Blackouts statt:

\begin{easylist}
    # 01.11.: Colloton ruft Notstand aus, Wahlen werden verschoben
    # 14.11.: Quebec marschiert in die UCAS ein
    # 25.11.: Waffenstillstand vereinbart, nachdem UCAS militärische Erfolge verbuchen konnten
    # 12.12.: NAN-Streitkräfte greifen UCAS an; das SSC lässt verlauten, \textit{``dass sämtliche Versuche, die Joint Task Force Seattle des UCAS-Militärs durch Salish-Shidhe-Gebiet zu verlegen, als kriegerischer Akt angesehen und entsprechend beantwortet werden würden''}
    # 21.12.: Kentucky verlässt die UCAS und tritt den CAS bei
    # 26.12.: SSC lässt die JTFS auf bestimmten Routen abmarschieren, Marine wird eskortiert, etc; die UCAS haben keinerlei Militärpräsenz an der Westküste mehr
    # 28.12.: Seattle erklärt seine Unabhängigkeit
    # 02.01.: St. Louis ebenfalls
    # 25.01.: die UCAS unterzeichnen Wiedereintritt in die BRA
    # 27.01.: Vizepräsidentin Martin spricht sich für Unabhängigkeit Kanadas aus
\end{easylist}

% bzgl. der Alpha-Bilder:
%
% > Was ist mit den echten Kampfaufnahmen aus Detroit, die eindeutig diese Bugs zeigen?
% Treadle
% > Ein bisschen davon ist durchgesickert, aber die „Experten“ erklären sie schnell für falsch, nennen sie manipuliert und einen Versuch, Ares zu verleumden, indem man sich „die Tragödie in Chicago zunutze macht“ und mit einer falschen Story „von der Wahrheit ablenkt“. Das ärgert mich wirklich massiv. Nach all diesen Jahren hat niemand irgendetwas gelernt.
% Bull








\section{All tied up}

\subsection{Meta-Timeline}

Das III. Armeekorps, knapp 100.000 Leute, verschwinden spurlos am 10. August.

Riflemans Bericht über die Lage in Detroit ist von Ende August. Anfang bis Mitte September ist in Detroit Straßenkampf und Krieg, Deadlines Reports sind aus der Zeit (vllt nochmal nachschauen).

Zum Zeitpunkt von \textit{Midnight Run} ist es eigentlich Mitte August, um den 20en rum (\textbf{Anm. von später:} ich springe ein bisschen vor bis Anfang September). Bei uns könnte die UCAS Armee am nächsten Tag verschwinden, wenn die Runner in Vancouver sind. Die Bundesbehörden werden daraufhin auf höchste Alarmbereitschaft versetzt:

\textbox{Newsflash}{
Drittes Armeekorps der UCAS Army spurlos verschwunden. Regierung ratlos. Bundesbehörden, Militär und lokale Sicherheitsdienstleister sind in Alarmbereitschaft versetzt und beziehen defensive Positionen zur Sicherung ``strategisch wichtiger Punkte''. Präsidentin Colloton zu den immernoch anhaltenden Kämpfen in Detroit: ``Darum soll sich Ares kümmern.''
}


\subsubsection{Der Weg zu Free Seattle}

Am Ende von \textit{Deck Building} sollen die Kämpfe in Detroit enden und Vogel's Business-Bullshit-Rede vorkommen. Der Auslöser für die konkreten Unabhängigkeitsverhandlungen(?) Seattles aus den UCAS sollen die Kündigung der UCAS der BRA sein. Das heißt: die Runner müssen noch mehr vom Turmoil in den UCAS mitbekommen, vom Turmoil innerhalb Seattles deswegen (von wegen mehr Druck auf die Gouverneurin, ihre Wahlversprechen in der Angelegenheit zu verwirklichen), den Austritt aus den BRA und entsprechend das vermeintliche Chaos, das deswegen (auch in Seattle) entsteht. Puuuh...




\subsection{M-TOC Mark I}

Neurons neues Spielzeug.

\begin{easylist}
    # Stand: 30/30 \skill{Elektronik} 
    ## bei 6: Edge umverteilen als Nebenhandlung - alle
    ## bei 12: \textbf{Junk Wall} - +2 auf die Firewall einer verbundenen Cyberbuchse
    ## bei 18: Zugriff auf Ausrüstung, Kommunikation und Vitalmonitordaten (falls vorhanden)
    ## bei 24: Smartgun-Waffen +1 ATK
    ## bei 30: \textbf{Team Leader} - 3 frei auf- und verteilbare Würfel für \skill{Teamwork, Navigation, Wahrnehmung, Kampfmanöver}
\end{easylist}

Yu hat sein Kommlink gepaired, Rude sein Kommlink, seine Ingram-Smartgun und seine Cyberaugen.




\subsection{Deck Building}

Die Runner erhalten den Auftrag am zweiten September.

Yu ist hat sich am gleichen Tag, an dem sie den Auftrag erhalten für ``morgen oder übermorgen'' mit Four-Finger Wong verabredet, um sich eine \skill{Muskelstraffung (Stufe 3)} verpassen zu lassen (Kosten: 90.000\nuyen).

Frosty hat \textit{Stille} zur Hälfte gelernt, \textit{Gerät analysieren} noch gar nicht; sie braucht für beides noch jeweils einen 10-12 Stunden-Tag Ruhe.