\chapter{R \& D}

Hier kommen lose Gedanken, Ideen, Szenen und dergleichen rein.




\section{Spielleitung}

Tips, Hinweise, allgemeine Herangehensweisen.

\subsection{Hinweise geben}

As shown by Mr. Jeremy Crawford himself: 

``Während du \$dies\_und\_das betrachtest (oder darüber nachdenkst), wunderst du dich \textbf{ob} [die Dinge vielleicht so oder so sind].''

Damit kann man wunderbar Leute auf eine Fährte bringen!! Zur Not auch mit Probe: \skill{Logik + Intuition} vllt.

\textbox{Beispiel:}{``Während du dir die geöffneten Dokumente auf dem Terminal anschaust, fragst du dich, wie gut es wohl um die Auftragslage der Firma steht und in wie fern sich eine gesprengte Brücke darauf wohl auswirkt. (oder zumindest: ...wie fern eine Brückensprengung damit wohl in Verbindung steht.}


\subsection{Cheatsheets}

\begin{easylist}
    # Objektwiderstand-Übersicht
    # Attribute und Fähigkeitslisten! Am besten vllt direkt an den Bildschirm kleben
    # Edge-Übersicht bereit halten
    # Verteidigungsproben! (versch. Zauber)
    # Frosty's Zauber auf ein Blatt
\end{easylist}

\subsection{Hoststufen}

Grobe Richtlinien aus der 5ten Edition; in der 6ten evtl. etwas höher?

\textbox{Beispiele}{
    {\rowcolors{1}{gray!30!white}{}
        \begin{tabular}{m{0.7\textwidth} m{0.15\textwidth}}
            {\fontspec{Njord.otf}{Host}} & {\fontspec{Njord.otf}{Stufe}} \\
            Persönliche Sites, Piratenarchive, öffentliche Bildungseinrichtungen & \lcr{c}{1-2} \\
            Kleinunternehmer, private Geschäftshosts, öffentliche Bibliotheken, kleine Policlubs & \lcr{c}{3–4} \\
            Soziale Netzwerke, kleine Colleges und Universitäten, lokale Polizei, internationale Policlubs & \lcr{c}{5–6} \\
            Matrixspiele, lokale Konzernhosts, große Universitäten, niederrangige Regierungsbehörden & \lcr{c}{7–8} \\
            Reiche Gruppen, regionale Konzernhosts, wichtige Regierungseinrichtungen, gesicherte Einrichtungen & \lcr{c}{9–10} \\
            Hauptquartiere von Megakons, Militärhauptquartiere, Geheimdienste & \lcr{c}{11–12}
        \end{tabular}
    }
}





\section{Midnight Run}

\subsection{Loose Ends}

Brackhaus kontaktiert die Runner später:

\textbox{Sag's ihnen ins Gesicht}{``Wann haben Sie eigentlich zuletzt Urlaub gemacht? Schließlich haben Sie sich das redlich verdient. Ich habe gehört, die bayerischen Alpen seien zu dieser Jahrszeit sehr hübsch. Ich kann Ihnen das Touristikbüro XY empfehlen; wenn Sie meinen Namen erwähnen, erhalten Sie mit Sicherheit ein sehr attraktives Angebot.''}





\section{Szenen}

\subsection{spezifisch}

\subsubsection{Bajonette}
Beim nächsten Besuch im \textit{BBC Weapon Works} fragt Vladek nach Tusk: ``Wo ist Freundin? Große Ork, war letzte Male auch immer mitgekommen? Fragt immer nach Klingen, vielleicht chabe was in Angebot: guckst du hier!'' Er zeigt eine längliche, eher dünne Klinge, die an ihrem Ende keinen Griff, sondern so etwas wie eine Schiene hat. Du/Rude erkennst sofort, um was es sich handelt. Vladek fährt fort: ``Bajonett! Nicht so beliebt seit ganze Zeit schon, aber wird machen großes Comeback! Kommt immer mal wer zu nah dran, aber dann...'' Er deutet einen Schlag von oben an, wie ein Messer, dass man auf einen Tisch rammt - oder jemandem in den Hals, macht eine Vierteldrehung mit der Hand und stößt nach vorne. ``Zack! Ha! Ist perfekt für solche Situation. Musst du Bewegung vielleicht üben, aber sonst...''

\subsubsection{Bouncer / Jimmy Vitello}
Die Runner kommen genau in dem Moment an, als \textbf{Jimmy Vitello} jemanden rauswirft; ``Ihr hört Tumult und als ihr um die Ecke geht, seht ihr eine Figur (Beschreibung! ein Elf? Ein Ancient gar? s.u.) ein paar Meter aus dem Haupteingang fliegen und auf dem Rücken landen. Eine halbde Sekunde später seht ihr einen eher kleiner, aber dennoch beeindruckend massiv gebauten Mensch in einem chiquen Hemd mit hochgekrempelten Ärmeln hinaustreten, sich die Hände abklopfen und hört ihn rufen: 'Pass auf, dass du oder einer deiner Drekhead-Kumpels sich hier nicht wieder blicken lassen. Ihr habt hier verdammt nochmal nichts verloren, kapisch?!''

\vspace{0.5em}
``Versteh mich nicht falsch, Rude. Ich hab nichts gegen die Löwenzahnfresser, aber mit den Drekheads aus Puyallup haben wir in letzter Zeit nur Ärger. Behaupten, wir würden armen Leute dort um ihre Existenz bringen. Puah! \textit{(er deutet ein Ausspucken an)} Dabei sind wir es doch, die dafür sorgen, dass sie sich keine Sorgen um ihre Gschäfte machen müssen und nachts unbehelligt die Straße runtergehen können! Da ist es nur fair, wenn wir für unsere Mühen auch eine kleine Gegenleistung erhalten. Für die Errants existiert die Gegend hier praktisch gar nicht und als ob diese neon-grünen Möchtegern-Helden solche Aufgaben übernehmen könnten...!''

\subsection{unspezifisch}

\subsubsection{Kopfgeld (während der Blackouts}
In \textit{Blackout} gibt es den Abschnitt ``Defende nos in Proelio'' (ab S. 72) bzgl. Magischer Bedrohungen und Kopfgeldern; da lässt sich was draus machen!



\epigraph{She sent me a red letter\\ a personal vendetta\\ if that makes something better \\ then that is how it ends}{\textit{Radkey - Red Letter (misheard)}}


\subsubsection{Graffitis}

\textbox{}{SOYKAF IST MENSCHEN // IS PEOPLE}

\textbox{}{AZZIES RAUS - ATZEN REIN}

\textbox{}{BAD WOLF}



\subsubsection{Nothing Personal}
``Nun, Mister Yu, ich bin mir sicher, dass sie Verständnis dafür haben, dass wir als professionelle Akteure in unserem Geschäft im Sinne eines erfolgreichen Geschäftsabschlusses nicht nur einfach Sicherheiten abwägen, sondern auch dafür sorgen müssen. Ich danke Ihnen daher für Ihre Mitarbeit. Ihre Dienste werden nicht nun länger benötigt.'' -> Überraschungsprobe!

\subsubsection{Indoor Lauschen}

Die Runnerin lauscht an einer Tür einem Gespräch. Je nach Anzahl Erfolgen mit \skill{Wahrnehmung}:

\begin{easylist}
    # (1): unverständliches Gemurmel nur einer Stimme, irgendwann ein Stuhl-Rücken
    # (2): Anzahl (sprechender) Personen, immernoch unverständlich
    # (3): einzelne Wortfetzen
    # (4+): genug, um den Inhalt/Sinn zu erschließen
\end{easylist}

\textbf{$\rightarrow$ Perfekt für Midnight Run!}

% Idee: Makro für Infobox mit Liste, die wie oben aufgebaut ist; d.h. Anzahl Erfolge in Klammern


\subsubsection{Verpflichtungen: Jules}

Wenn Frosty das nächste Mal Jules besucht, ist er nicht da (wenn sie anruft, geht er nicht ran: sie denkt sich dann ``fahr ich halt hin''). Freunde von ihm sind da, die Frosty nur entfernt kennt. [NAME!] wird sofort sehr ernst, sobald sie nach Jules fragt: er sei im Krankenhaus, irgendwas mit seinem Cyberarm. Hintergrund: das Ding hat sich von innen schwer entzündet und die Nerven angegriffen. Die Behandlung ist teuer und Jules kann sich das nicht leisten; 2000¥ vielleicht? Oder 10\% von allem, was sie bisher verdient hat. :D


\subsubsection{Die Geister die wer rief}

\textit{Irgendeine} Art Poltergeist: schwebende Gegenstände - z.B. Küchenutensilien, wie Messer und Beile (Gear-Regeln!) - die die Runner angreifen. Der Geist ist... ein freier? Muss (trotzdem) gebannt werden?

\subsubsection{Gedrängel}

Bei Misserfolg von zwei Leuten, durch eine Tür, oder zu etwas hin zu hasten:

\textbox{Sag's ihnen ins Gesicht}{Rude (ggf) ist größer, schneller, stärker und - zugegeben - auch determinierter als du. Er drückt sich an dir vorbei, schiebt dich dabei etwas zur Seite und das bringt dich stärker aus dem Gleichgewicht, als erwartet. Du stürzt zu Boden (Probe?) - Status: \textit{liegend}.}






\section{Free Seattle}








\section{Elven Blood}

Sollte es doch jemals was werden... x)

Vllt. mach ich tatsächlich diese drittklassige Shakespeare-Chose?

\textbox{Belial}{Sodann, heldenhafte Schattenläufer! Möget ihr die Inkarnation der Ancient-Gerechtigkeit sein, die auf die Schergen der Dunkelheit darnieder regnet! Lasset unsere Widersacher den Zorn des Belial spüren, auf dass ihr fürwahr ein Exempel an ihnen statuieren mögt! Auf nun, wackere Helden! Zeigt, dass Beliar wahrhaftig der würdigste Erbe Green Luzifers ist!}









\section{Detroit Arc}
 
Bzgl. Matrix:
 
Ares kontrolliert alle Matrix-Nodes in und um Detroit und ist natürlich heftigst in GOD involviert, weswegen es zwar Matrixkapazitäten in Detroit gibt, der Informationsfluss aus dem und in den Greater Detroit Metroplex aber alles andere als einfach möglich ist!!


\subsubsection{Wanted: Balor}

Auf Balor ist (bzw. war) ein Kopfgeld ausgesetzt:

``Einlösbar im Knight Errant Hauptquartier, Abteilung für unabhängige Vertragsnehmer. Viel Erfolg das unter den aktuellen Umständen zu drehen. Das DIMF (Dunkelzahninstitut für Magische Forschung) zahlt aber soweit ich weiß nach wie vor für ausgeschaltete Toxiker -- sofern ihr das Kopfgeld tatsächlich an ihren Paragraphendrehern vorbei ausgezahlt bekommt.'' (oder so; siehe Blackout -- Kopfgelder; vllt wars auch die Draco Foundation. Was war nochmal der Unterschied?)


\subsection{Reaver}

...könnte jetzt bei den 162ern mitmischen. Warum nicht auch in Seattle? Vielleicht steigt deren Aktivität an, in Puyallup oder den Barrens und die Runner werden dafür angeheuert. Vielleicht ist das sogar ein guter Hook im Anschluss oder Vorfeld an die erste Elven Blood Mission? Belial oder ein anderer Ancient würde sie dann anheuern; eher ein anderer Ancient-Captain und Belial hört davon - enter Elven Blood.


\subsubsection{Reaver's Klinge}
Frosty \& Tusk sehen die Klinge schwach türkis glühen. Alle anderen nicht.

Hebt jemand außer den beiden die Klinge auf, spüren sie ein leichtes kribbeln, wie einen Schlag von einer Batterie - schwächer, aber auch länger.

Frosty \& Tusk spüren deutlich den Fluss des Mana, der sich aus der Klinge erst zu ihrer Hand, dann über den Unterarm und schließlich hoch bis zur Schulter erstreckt. Entfernt und undeutlich bildet sich in deinen Gedanken wie ein Flüstern, aber weder geschrieben noch gesprochen, ein Wort - nein, ein Name: Skinbane? Verminfate? Verminbane? Mudcarver?







\section{abgehandelt}

\subsubsection{Four-Finger Wong}
Wenn Yu das nächste Mal jemand in einer ruhigen Situation da ist, wird er von einer Katze begrüßt: ``Nachdem du einige Schritte in die Praxis getan und Wong begrüßt hast, streift etwas dein Bein, nein, es drückt sich was dagegen. Dann dein anderes. Als du hinschaust, siehst du eine gelbliche oder eher hellbraune Katze eine acht um deine Beine laufen und sich dabei an dir reiben. Nach einem kurzen Augenblick der Orientierung erkennst sie wieder: das ist die gleiche Katze, die du aus den ISTAR Facilities mitgebracht und bei Wong gelassen hast.'' (Sie heißt jetzt Tsu, Sunzi, oder sogar Sun Tsu)

\vspace{0.5em}
Sollten Sie mit FFW sprechen wollen (z.B. über Cyberimplantate o.ä.), geht er zur Tür, und dreht ein altes Emaille Schild um, auf dem ganz oben auf englisch und in großen Lettern "CLOSED" steht. Darunter noch in anderen Sprachen udn Schriftzeiche (Yu erkennt eine Zeile als kantonesisch). Als er sich den Runnern wieder zuwendet, zeigt das Schild ihnen "OPEN".