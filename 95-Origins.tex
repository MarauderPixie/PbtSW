\chapter{Origins}

\epigraph{Shame on us, doomed from the start\\God have mercy on our dirty little hearts}{\textit{Nine Inch Nails - Zero Sum}}

\section{Frostburn}
% \paragraph{Orkin, 2.00m groß, 25 Jahre alt (geb. 2055);}
\textbox{Profil}{
    \begin{tabular}{l l l l}
        \textbf{Metatyp:} & Ork & \textbf{Alter:} & 25 Jahre \\
        \textbf{Größe:} & 2.00m & \textbf{Gewicht:} & 115kg
    \end{tabular}
}
Frostburn konnte, ebenso wie ihr großer Bruder Jules, nie besonders viel mit den schamanistischen Traditionen ihrer Leute im allgemeinen - den \textit{Cascade Ork} - und ihrer Familie im besonderen anfangen. Im Gegensatz zu Jules, der in seiner Besessenheit mit allem was Chrom, Öl und Metall ist, per se wenig mit spirituellen Dingen am Hut hat, liegen die Dinge bei Frosty anders: als eine erwachte Person der sechsten Welt wird von ihr erwartet, den Kontakt zu den Stammesgeistern zu pflegen. Sie dagegen suchte ihre ``Erleuchtung'' allerdings lieber in der Wissenschaft und Hermeneutik, studierte an der Universität Seattle und begann eine Karriere als Lohnmagierin im konzerneigenen Geheimdienst des ehemaligen Technologie-Giganten \textit{NeoNET} - wo sie auch an ihren Decknamen kam. Nachdem \textit{NeoNET} 2078 für die KFS-Krise verantwortlich gemacht und zerschlagen wurde, konnte Frostburn aufgrund ihrer Tätigkeiten jedoch nicht ohne weiteres bei der Konkurrenz anheuern (bei denen sie zwar auch auf Headhunting-Listen steht, allerdings im Sinne eines eher feindschaftlichen Personalmanagements) und eine Rückkehr zu den \textit{Cascade} stand ebenfalls außer Frage. Es lag also nahe, dass eine Person mit Frostburns Fähigkeiten und Interessen sich ihren Unterhalt in den Schatten verdient.



\section{Neuron}
% \paragraph{Elfin, 1.78m groß, 43 Jahre alt (geb. 2037);}
\textbox{Profil}{
    \begin{tabular}{l l l l}
        \textbf{Metatyp:} & Elf & \textbf{Alter:} & 43 Jahre \\
        \textbf{Größe:} & 1.78m & \textbf{Gewicht:} & 63kg
    \end{tabular}
}
Laut Kilian: Neuron hat das Studium \$aus\_Gründen nicht geschafft, was in Elfenkreisen schon irgendwie arg peinlich ist und sie geradezu unausweichlich zum schwarzen Schaf macht. Sie hat sich auf der Uni politisiert, was ihr starker Interesse und Engagement an allen Tätigkeiten bzgl. Humanis-Policlubs sowie (evtl.?) die Bekanntschaft mit Brent erklärt (?). Im Zuge der Aneignung (uff) ihrer krassen Skillz hat Neuron Netcat und/oder Metatron kennen gelernt (erstere evtl auch wg. der Politik? vllt auch beides) und nebenbei eine üble Abhängigkeit entwickelt (Anxiety? Depression? Schuldgefühle "versagt zu haben"? egal?).



\section{Rude}
% \paragraph{Troll, 2.40m groß, 29 Jahre alt (geb. 2051);}
\textbox{Profil}{
    \begin{tabular}{l l l l}
        \textbf{Metatyp:} & Troll & \textbf{Alter:} & 29 Jahre \\
        \textbf{Größe:} & 2.40m & \textbf{Gewicht:} & 310kg
    \end{tabular}
}
Rude wuchs im Seattler \textit{Ork-Untergrund} auf, und zwar lange bevor Proposition 23 angenommen und der Untergrund als offizieller Stadtteil anerkannt wurde. Entsprechend hat er kaum formelle Bildung erhalten, was aber nicht 'gar keine' heißt: Seine Rolemodels waren diejenigen unter den Skraacha, die (im Gegensatz zum Star oder den Knights) für Recht und Ordnung unter dem Sprawl gesorgt haben. Sein Handwerk dagegen hat er bei den zwielichtigeren Tätigkeiten der Gang und ihren ``Vertragspartnern'' erlernt - u.a. Teilen des O'Malley Syndicats, also der Casa Nostra Seattles. Was als scheinbar ehrliche und anständige Arbeit begann - Gebäude- und Personenschutz, Fahrdienste, Dealer aus der Gegend schmeißen, etc. - entpuppte sich mehr und mehr als schlicht die andere Seite der gleichen Medaille und kollidierte zunehmend mit Rudes ausgeprägtem Sinn für Gerechtigkeit. Nun schlägt er sich - auf sich allein gestellt oder mit einigen zuverlässigen Kollegen - durch die Schatten, während er dabei zusehen muss, wie sich der \textit{Ork-Untergrund} seiner Kindheit und Jugend durch den Erfolg von Prop 23 fortschreitend gentrifiziert und zu nichts weiter als einer weiteren Konzernenklave wird.



\section{Tusk}
% \paragraph{Orkin, 2.06m groß, 27 Jahre alt (geb. 2053);}
\textbox{Profil}{
    \begin{tabular}{l l l l}
        \textbf{Metatyp:} & Ork & \textbf{Alter:} & 27 Jahre \\
        \textbf{Größe:} & 2.06m & \textbf{Gewicht:} & 122kg
    \end{tabular}
}
Tusk ist eigentlich professionelle Kampfsportlerin. An sich ein toller Job, aber mitunter etwas schlecht bzw. unzuverlässig bezahlt. Außerdem kann die Nahkampfadeptin in einem so stark reglementierten Umfeld wie den traditionellen asiatischen Kampfkünsten nur schwerlich ihr ganzes Potential entfalten (oder ausleben). Die Schatten bieten die perfekte Ergänzung dazu, sowohl was die Herausforderung und den Nervenkitzel angeht, als auch den meißt durchaus soliden Verdienst. Das Problem ist nur: in der der Philosophie des Lethani, die Tusk's Großmeister Arvid lehrt, stellt die Anwendung von Gewalt zu anderen als rein defensiven Zwecken einen absoluten Bruch dar. Eine schwierige Vorraussetzung, um in den Schatten \textit{irgendetwas} zu erreichen...



\section{Yu}
% \paragraph{Elf, 1.80 groß, 28 Jahre alt (geb. 2052);}
\textbox{Profil}{
    \begin{tabular}{l l l l}
        \textbf{Metatyp:} & Elf & \textbf{Alter:} & 28 Jahre \\
        \textbf{Größe:} & 1.80m & \textbf{Gewicht:} & 77kg
    \end{tabular}
}

Yu wuchs inmitten des Schattens der Octagon Triade in Seattle auf und durch seine geradezu intuitive Eloquenz und Ausstrahlung sowie sein diebisches Talent, praktisch überall unter- oder einzutauchen (oder auch an Orte zu gelangen, an denen er eigentlich nichts zu suchen hat), war ihm ein rascher Aufstieg zum Unterhändler der Triade garantiert.

Womit sich Yu jedoch noch nie anfreunden konnte, ist der geradezu lächerlich kräftige Aberglaube, der sich innerhalb des Seattler Kapitels der Octagon nicht nur in den höheren Führungsebenen der Triade zeigt, sondern sogar von dort ausgeht. Es gibt Stimmen, die behaupten, dass darin der Grund liegt, weswegen die Octagon in Seattle im Gegensatz zu ihrem Hong Konger Pendant im Krieg gegen den Yellow Lotus zu unterliegen drohen und führte letzten Endes auch zum Bruch von Yu mit dieser Familie: bei einer Verhandlung mit den Yellow Lotus, die zum Beenden des langjährigen Kriegs zwischen den beiden Triaden beitragen sollte, hat einer der Fusssoldaten in Yu's Gruppe die Nerven verloren, als ein Yellow Lotus - vermutlich schlicht aus versehen - das Feng Shui des Treffpunktes störte. Die Situation eskalierte und endete in einem Blutbad. Yu konnte zwar schwer angeschlagen entkommen und sich von seinem Freund (und Ersatz-Großvater? Mentor? rationalen Bruder im Geiste?) Four-Finger Wong wieder aufpäppeln lassen, aber mit den Octagon hat er seitdem abgeschlossen. 




% Sidenote: der Gelbe Lotus operiert in den Barrens. Vielleicht geht da ja was?
%
% David Gao, der Shan Chu der Octagon in Seattle, ist ziemlich unfähig. Offenbar.
%
% irgendwas mit Octagon-Triade; Untergebene der \textbf{Red Dragon Association} \textit{(Hung Lung Mun)}, die wiederum praktisch vom großen Drachen Lung kontrolliert wird. Während die Octagon (und deren Meister RDA) in Hong Kong ihre Gegner vom \textbf{Gelben Lotus} 2068 praktisch völlig vernichtet haben, ist das in Seattle mitnichten der Fall - den Gelben Lotus geht's gut, die Octagon strugglen (wiki sagt: "die kleinste unter den drei bedeutsamen in Seattle).