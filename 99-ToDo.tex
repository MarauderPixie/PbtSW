\chapter{To-Do}

\subsubsection{Layout \& Optik}

\begin{easylist}
    #  \textbf{Unbedingt}
    % ## Fix Textboxes; Zeilenabstand, Einheitlichkeit, Fontsize...
    % ## eso-pic FG auf Kapitelanfangs-Seiten
    % ## Fussnoten weiter in die Ecke
    % ### ...und irgenwie die Kapitel-Fussnoten fixen (make it "Section // Chapter // Page", auch auf Kapitelseiten)
    % ## Sans- und Bold-Font Shenanigans fixen
    ## Linien in über und unter Kapitel-Titeln
    ### ...und bei Subchapters auf Breite des Textes anpassen
    
    #  \textbf{Optional}
    % ## Fontsizes!
    ## feinere Möglichkeiten für Absätze (vor und nach Listen zB evtl)
    ## Titles fancy machen (///// und so)
    ## bessere Alternative zu Amplitude finden 
    ## ``Jackpoint''-Style Kommentare; evtl. als Makro?
    % ## immer gleichen Abstand von Text zu Kapitelbild (statt abhängig von der Anzahl der Zeilen des Kapitel-Titels)
    ## alle titlespacings zwischen gleichem (rigidchapters) und ungleichem Abstand (rubberchapters) angleichen!
    % ## Fusszeilen
    % ## Durchsichtigkeit der Textboxen
    % ## generelle Farbüberarbeitung; Boxenhintergrund, Linien und anderes lila; ggf. feiner unterteilen 
    ## Kapitel-Logo überarbeiten? Vllt.? Muss auch nicht...
    ## Geometries überarbeiten (dt. Layout hat insgesamt weniger Ränder)
    ## Ppuntklinien zw. Titeln und Seitenzahl reichen noch suboptimal an Seitenzahl heran...
\end{easylist}

\subsubsection{Struktur \& Inhalt}

Evtl. will ich auch die \textit{kanonische} Strukturierung übernehmen, also die Abschnitte:

\begin{easylist}
    # Scan this
    # Tell it to them straight
    # Behind the scenes
    # Debugging
\end{easylist}

Insbesondere wegen der letzten beiden Punkte! Evtl. auch \textbf{nur} die beiden letzten; auf jeden Fall lohnt es sich, die beim Planen/Vorbereiten zu berücksichtigen.