\chapter{Free Seattle}

\section{Preludium: UC\texorpdfstring{\textsubscript{r}AS\textsuperscript{h}}{Preludium: UCrASh}}

\gesicht{Knapp 3 Wochen sind vergangen, seit die Krise in Detroit für beendet erklärt wurde. Aber ruhig blieb es nicht, ganz im Gegenteil: keine drei Tage später haben die UCAS als Reaktion auf den Bruch mit Ares eine metaphorische Atombombe gezündet, indem sie unilateral den BRA - den \textit{Business Recognition Accords} - aufgekündigt haben. In den BRA ist festgesetzt, dass der Konzerngerichtshof als alleinige Instanz über die Exterritorialität von Konzernen entscheidet. Die UCAS haben den AAA-Megakonzernen also nicht weniger als ihre Geschäftsgrundlagen wie einen Teppich unter den Füßen weggezogen. Im dadurch entstandenen Chaos, auch oder sogar besonders in Seattle, wurden dadurch die Rufe nach der Klärung der Frage nach Seattles Unabhängigkeit lauter. Eine der vielen Nachrichtenmeldungen zu dem Thema lautete zB folgendermaßen:}


\subsubsection{Seattles Zukunft steht auf dem Spiel}

Vor etwas mehr als einem Jahr wurde die Gouverneurin von Seattle, Corinne Potter, mit einem Programm gewählt, das ihr die Stimmenmehrheit brachte, aber nur wenige Versprechungen enthielt. Brackhaven hatte ihr viele ungelöste Probleme hinterlassen, und Potter versprach, sich um jedes einzelne zu kümmern, auch wenn ihre Kampagne nur wenige Details dazu enthielt. Seit ihrer Wahl hat Potter Berater und Experten hinzugezogen, um zu versuchen, die richtige Lösung für jedes dieser Probleme zu finden. Mit ihren Entscheidungen hat sie ihre Wähler manchmal verstimmt, da diese vielleicht erwartet hatten, dass sie eher auf ihr Herz als auf ihren Verstand hört, aber jetzt kümmert sie sich um das vielleicht kontroverseste Thema ihrer Kampagne: Seattles Unabhängigkeitsbewegung. 

Während der Wahl gab es mehrere Forderungen nach einem freien und unabhängigen Seattle. Diese Rufe stießen fast gleichermaßen auf Zustimmung und Ablehnung. Seattle ist in dieser Frage eindeutig tief gespalten. Um ihr bei der Lösung dieser Krise zu helfen, hat Potter Vertreter mehrerer Länder und Megakonzerne eingeladen, die ein besonderes Interesse daran haben, ob Seattle unabhängig wird oder bei den UCAS verbleibt.

Es überrascht nicht, dass Potter mehrere Vertreter des Konzerngerichtshofs eingeladen hat, insbesondere \textbf{Major Brenda Reed} von Ares, \textbf{Thomas Miranda} von Horizon und \textbf{Takuto Nakagawa} von Renraku; dazu kommt noch die Pacific Prosperity Group, die Wuxing-Exec \textbf{Dewei T’ao} an den Verhandlungstisch geschickt hat. Darüber hinaus haben Seattles Nachbarn ein Mitspracherecht: Der Salish-Shidhe-Rat hat \textbf{John Abernathy} vom Salish-Stamm geschickt, die UCAS haben den frischgewählten Kongressabgeordneten \textbf{Carl Derrick} entsandt, um sicherzustellen, dass die Dinge so bleiben, wie sie sind, und Tír Tairngire\footnote{Ich sprech das in etwa \textit{Tier Ta(y)'en'gier} aus, mit stummen Ypsilon.} wird durch \textbf{Margaret Telestrian} vertreten. Die vielleicht überraschendste Einladung zum Treffen ging an \textbf{die Seedrachin}, die auf der Konferenz noch nicht erschienen ist. Was die Anwesenheit eines der umstrittensten Drachen der Welt für diese Konferenz bedeutet, lassen wir mal dahingestellt. 

Im Laufe der nächsten Woche wird Seattles Zukunft zum Teil von diesen Parteien bestimmt werden. Nur die Zeit wird zeigen, ob die Stimme von Seattles Bevölkerung wichtiger sein wird als die Stimmen von Seattles Megakonzernen.


\vspace{1em}
\textbf{An Tusk}: Deine Connection April Summers schrieb für ihre Zeitung in einem Kommentar dazu: 

\gesicht{Das ist vielleicht das klügste politische Spiel, das Potter spielen konnte. Sie kann öffentlich Unterstützung sammeln und bei einigen Mächten das Terrain sondieren, und wenn es schiefläuft, kann sie immer noch den anderen die Schuld zuschieben. Allerdings bedeutet das auch, dass sie das Rampenlicht meidet und nicht die volle Anerkennung für einen möglichen Erfolg erhält. Sie verspielt einen Teil der Publicity des Erfolgs, um die Kritik an einem möglichen Scheitern zu dämpfen. Das wiederum könnte letztendlich als rückgratlos angesehen werden.}

\subsection{Die Lage der Runner}

All das ist natürlich auch in eurem Umfeld ein großes Thema: Yu, du weißt zB von Mia, dass sie Feuer und Flamme für ein unabhängiges Seattle ist:

``Was haben die in DC denn jemals für uns getan? Ihre scheiß Armee sorgt nur für mehr Spannung zwischen dem Council und den anderen NANs als für irgendwelchen angeblichen Schutz! In keinem anderen Metroplex der Welt herrscht ein solches Machtgleichgewicht wie hier, keiner der Megas hier hat dsa Sagen - nicht Ares, keiner der Japanokons, \textbf{niemand} - nichtmal das goldene Würmchen hat hier viel zu melden. Und das liegt sicher nicht an DC, sondern allein an \textbf{uns}, an den Seattler Schatten. Wir sind \textbf{niemandes} Schoßhunde, hier suchen \textbf{wir} uns aus, für wen wir laufen.''. 

Rude, im Burning Hole werden Wetten darauf abgehalten, welche Repräsentanten am Leben bleiben und du hast mit Hez darauf gewettet, dass Seattle unabhängig wird - die Chancen stehen 2:1, auch wenn dort absolut niemand auch nur im Ansatz genug von der Sache versteht, um das einschätzen zu können. Du weißt auch, dass Hez es lieber wäre, wenn es nicht dazu kommt: ``DC ist zwar auch nur ein weiterer Sumpf, aber wenigstens ein weit entfernter. Und immer noch besser, als wenn die Kons hier komplett den Laden übernehmen. Ich mein, die Containment Zone in Chicago strahlt nach wie vor stärker als die Sonne über Fujiyama. Und guck dir doch an, wie es in Detroit aussieht: alles liegt in Schutt und Asche und was macht Ares? Sie verschwinden. Einfach so. Aus ihrer \textit{eigenen} Amerikanischer-als-Uncle-Sam-Vorzeigeenklave. Einfach, weil ein Wiederaufbau zu teuer ist. Und von den anderen fangen wir am besten gar nicht an. Du weißt genauso gut wie ich, wieviele Trogs es bei den Japanos auch nur in der kleinsten Tochter gibt - nicht einen einzigen.''


\section{Ludus Magnus}

Mia hat die Runner zu sich gerufen, um einen Job zu vermitteln; Gilroy 'Romeo' Steele ist auf die Runner aufmerksam geworden und sucht ein Team, dass für die Dauer des Gipfels verschiedene Aufträge für die Gipfelteilnehmer erledigen und ihm Bericht erstatten wird.

\gesicht{``Ich will, dass ihr während der Aufträge sämtliche verwertbaren Informationen und Erkenntnisse über ihre \textbf{Auftraggeber} sammelt, die ihr finden könnt - was bei den Runs jeweils passiert ist \textbf{und was ihr über den Job denkt}. Wie ihr an diese Infos kommt, ist mir egal. Ihr könnt eure Connections bemühen, lokale Quellen ausfindig machen, Aufzeichnungen und Datenspeicher auftreiben oder Zeugen befragen, also lasst euch was einfallen. Und vor allem: seid aufmerksam und lasst euch nichts durch die Finger gehen! Ihr kennt ja das Sprichwort: \textit{Haltet euch den Rücken frei, spart Muni, zielt genau und lasst euch nicht mit Drachen ein.}''}