\chapter{easylist}

Ähnliche Räume wie schon zuvor: etwas größer und mit angrenzenden Kammern und Betten, mehr oder weniger gemacht:

\begin{easylist}
\ListProperties(Style1*=\textendash\hspace{2mm})
    # \probe{Wahrnehmung}{1}: ein Bett ist aufgewühlter als die anderen, offene Kiste davor
    # \probe{Wahrnehmung}{4}: auf einem Bett ist ein seltsamer Knick im Blick, wie ein zusammengeklebtes Bild, das leicht verrutscht ist, sieht eine Kante des Bettes leicht versetzt aus
\end{easylist}

\section{Sektion}

Die beiden Räume sehen folgendermaßen aus:

\begin{easylist}
    # \textbf{Raum A}:
        ## relativ sauber; nicht gerade so klinisch, dass man sich hier ein Cyberdeck implantieren lassen würde, aber definitiv nicht so feucht-dreckig-gammelig wir die meisten Räume
        ## Blanker Metalltisch an der Wand
        ## Werkzeuge an den Wänden: Messer, Haken und Fleischerbeile in verschiedenen Größen
        ## ``eine Kiste, deren Bauart manchen seltsam vertraut ist: mit Stromkabel verbunden und digitaler Anzeige: -8°C''

    # \textbf{Raum B}:
        ## ein abscheulicher Schrein aus Haut und Knochen
        ## Askennen: \probe{Selbstbeherrschung}{2}, sonst 2B Schaden
        ## eine Welle der Aggression und des Hasses 
\end{easylist}

Der zuvor von Derek geschlossene Deckel nach oben wird wieder unter rostigem Quietschen geöffnet - allerdings erst nach einiger Zeit und wenn die Runner ausreichend Krach machen. Sollte das der Fall sein, ist gerade eine Patroullie in der Nähe.

\subsection{Subsektion}

Es gibt zwei (bzw. drei) Möglichkeiten, fortzufahren:

Sie kehren mit der Patroullie nach Ghoultown zurück oder versuchen selbst, einen Weg zurück zu finden (entweder bereits vor dem Öffnen des Deckels oder danach). 

Der zuvor von Derek geschlossene Deckel nach oben wird wieder unter rostigem Quietschen geöffnet - allerdings erst nach einiger Zeit und wenn die Runner ausreichend Krach machen. Sollte das der Fall sein, ist gerade eine Patroullie in der Nähe.

\subsubsection{Subsubsektion}
Sollten sie es auf eigene Faust versuchen, braucht es eine Ausgedehnte Probe, entweder \probe{Natur}{12, 2h}, falls sie auf der Ebene Ghoultowns sind, oder \probe{Natur}{18, 2h} falls nicht. Alle 10h steigt \emph{Erschöpfung} um 1. Jeweils auf der Hälfte, also nach 6 bzw. 12 Erfolgen, erreichen sie die oberste Kanalisationsebene und am Ende kommen sie mitten in Detroit heraus. 
