\documentclass{Shadowrun}

% no extra space beween itemlists
%\usepackage{enumitem}
%\setlist{nosep}

\usepackage[sharp]{easylist}
\ListProperties(Style1*=\textbullet\hspace{2mm}, 
                Style2*=\textendash\hspace{2mm}, 
                Hide=10, Hang=true, Progressive=1em)

% \usepackage{tablestyles}
% \usepackage{tabularx}

%%% Links / clickable Inhaltsverzeichnis
% offenbar ist die Reihenfolge, in der Pakete geladen werden, wichtig >_>
\usepackage[hidelinks]{hyperref} % 'hidelinks' macht, dass keine roten Boxen gezeichnet werden



% Yen Symbol:
% ¥

% document formalia
\title{Shadowrun: Sechste Welt}
\author{Tobias Anton}
\date{\today}

\sloppy

\begin{document}
    
    \customtitle
    \makebackground
    \tableofcontents
    
    % \chapter{To-Do}

\subsubsection{Layout \& Optik}

\begin{easylist}
    #  \textbf{Unbedingt}
    % ## Fix Textboxes; Zeilenabstand, Einheitlichkeit, Fontsize...
    % ## eso-pic FG auf Kapitelanfangs-Seiten
    % ## Fussnoten weiter in die Ecke
    % ### ...und irgenwie die Kapitel-Fussnoten fixen (make it "Section // Chapter // Page", auch auf Kapitelseiten)
    % ## Sans- und Bold-Font Shenanigans fixen
    ## Linien in über und unter Kapitel-Titeln
    ### ...und bei Subchapters auf Breite des Textes anpassen
    
    #  \textbf{Optional}
    % ## Fontsizes!
    ## feinere Möglichkeiten für Absätze (vor und nach Listen zB evtl)
    ## Titles fancy machen (///// und so)
    ## bessere Alternative zu Amplitude finden 
    ## ``Jackpoint''-Style Kommentare; evtl. als Makro?
    % ## immer gleichen Abstand von Text zu Kapitelbild (statt abhängig von der Anzahl der Zeilen des Kapitel-Titels)
    ## alle titlespacings zwischen gleichem (rigidchapters) und ungleichem Abstand (rubberchapters) angleichen!
    % ## Fusszeilen
    % ## Durchsichtigkeit der Textboxen
    % ## generelle Farbüberarbeitung; Boxenhintergrund, Linien und anderes lila; ggf. feiner unterteilen 
    ## Kapitel-Logo überarbeiten? Vllt.? Muss auch nicht...
    ## Geometries überarbeiten (dt. Layout hat insgesamt weniger Ränder)
    ## Ppuntklinien zw. Titeln und Seitenzahl reichen noch suboptimal an Seitenzahl heran...
\end{easylist}

\subsubsection{Struktur \& Inhalt}

Evtl. will ich auch die \textit{kanonische} Strukturierung übernehmen, also die Abschnitte:

\begin{easylist}
    # Scan this
    # Tell it to them straight
    # Behind the scenes
    # Debugging
\end{easylist}

Insbesondere wegen der letzten beiden Punkte! Evtl. auch \textbf{nur} die beiden letzten; auf jeden Fall lohnt es sich, die beim Planen/Vorbereiten zu berücksichtigen.

    \chapter{Vorwort}

Hoi, Chummers!

Willkommen in meinem kleinen Wust aus Aufzeichnungen, Vorbereitungen und dergleichen für die Shadowrun Kampagne meiner Runner und mir, ihrem Spielleiter. Meine Matrix-Persona ist \textit{MarauderPixie} und mein echter Name tut hier nichts zur Sache. Falls ihr unbedingt einen braucht, koennt ihr mich auch Mr. Johnson nennen.

Bis auf weiteres ist dies nur eine copy-pasted Version \textit{simple edition (SE)} von \textbf{SIXTH WORLD}, die sich zum Zeitpunkt des Kopierens (18. Jan. 2023) unter https://swse.neocities.org/ findet. Soweit ich sagen kann, basiert die SE auf der \textit{Digest Edition 1.2} mit einigen Reduktionen. Die SIXTH WORLD Vibes lehnen stark in Richtung Shadowrun 1-3, ich werde das Ganze stärker Richtung sechster Edition ausrichten. Mein Ziel ist es also, die Regeln im Laufe der Zeit anzupassen, zu kürzen oder zu erweitern und letzlich auch zu uebersetzen. Den Anfang machen dabei ein paar Bezeichnungsänderungen (XP zu Karma zB) und die Reduktionen der SE sozusagen rueckgaengig zu machen, d.h. z.B. Moves aus der Digest Edition 1.3 oder anderen Derivaten hinzuzufügen. Abgesehen davon, werden wir sehen, wohin es gehen wird. 

\section{Changelog}

\begin{easylist}
# rephrased the Ammo section in chapter 'Combat'
# (re-)added a few metatype moves
# use an array of stats instead of loosely assignable points
# changed 'mark XP' to 'receive Karma' (in chapters 1-3)
# changed 'Karma' to 'Edge' (in chapters 1-3), as is customary when doing this
# changed 'Hacker' to 'Decker'; that's just how we roll (talking about respecting the canon...)
# removed the option to \textit{awaken} from the \skill{Advance} Move, since awakening just like that isn't really canonical
# minor semantics
\end{easylist}


\section{Vorwort der SIXTH WORLD Simple Edition}

Sixth World is a ``hack'' of the game Dungeon World which attempts to capture the flavor of the world of the well-known RPG Shadowrun®.

The Sixth World is the dangerous and grim future of our own world, where magic has resurfaced, megacorporations rule the world, and humanity has perfected incredible new technological capabilities including advanced cybernetics and the worldwide virtual reality network called the Matrix.

This game assumes familiarity with Shadowrun, as well as with Dungeon World.

\subsection{Disclaimer}

Dungeon World is the property of Sage LaTorra and Adam Koebel, and is available under the Creative Commons Attribution 3.0 Unported License. See www.dungeon-world.com for details.

The Topps Company, Inc. has sole ownership of the names, logo, artwork, marks, photographs, sounds, audio, video and/or any proprietary material used in connection with the game Shadowrun. This is a fan-created adaptation, and no challenge is intended toward Topp’s ownership of the Shadowrun intellectual property.

\subsection{Simple Edition Notes}

This is simplified edition of Sixth World by Chris Clouser and Tanner Yea. The original used to be possible to find on http://waryjack.com/sw/ but now it seems to have disappeared. PDF is still available on dropbox.

This edition is very similar however – I see myself more as an editor and final beta-tester than new autor, and I aim to make the game (already excellent before I got to it) “production quality”.

– Wrb, wrb@autistici.org
    
    \chapter{Principles Of Play} \label{principles}

\epigraph{Mr. Johnson does not play dice with his Runners; He plays an ineffable game of His own devising, which might be compared, from the perspective of any of the other players (i.e. everybody), to being involved in an obscure and complex variant of poker in a pitch-dark room, with blank cards, for infinite stakes, with a Dealer who won't tell you the rules, and \textit{who smiles all the time}.}{\textit{adapted from Terry Pratchett}}

\textbf{Fiction First:} everything that happens in a session of Sixth World starts with the fiction, proceeds to rules (if necessary), and ends with the fiction.

\textbf{Moves are Not Powers:} most of the game's rules are encapsulated in small packages called moves. A move provides the rules to resolve particular situations that arise in the fiction (for instance, how to shoot someone, or seduce someone). Try not to think of moves as powers you must activate or ``use'', but as the rules that come into play when your character gets into a situation.

\textbf{Never Say Your Move:} because the game starts with and ends with the game fiction, you won’t say ``I use \skill{Rock \& Roll} on that guy!'' Instead, determine from what you are doing in the game world (running, shooting, jumping, dying, etc.) what move would apply. When the rolling is done, you conclude with some more fiction (or perhaps the GM does, depending on the outcome). This is the story, not whether you used \skill{Rock \& Roll} or \skill{Stay Frosty}.

\textbf{Fiction Forces:} if you do something in the game world that would trigger a move, then you \textbf{must} make that move. You can’t say ``I’m diving into the closet to avoid being spotted'' and then not make the \skill{Stay Frosty} move. Conversely, you can’t make a move unless the situation actually demands it. If you’re not fighting someone who’s fighting back, then you don’t get to make the \skill{Rock \& Roll} move. The game fiction dictates what happens.

\textbf{Speak to the Characters:} since the fiction anchors the game, remember that if you want to speak to or ask something of Valentin, the character being played by Keith, don’t say ``Hey Keith, do you have a spare frag grenade?'' Instead, speak to the character: ``Hey, Valentin, do you have a spare frag?'' (remember, though, you don’t have to act in first person! it’s okay to speak about your character, not as your character, if you prefer).


\section{Stats} \label{stats}

Most of the rules of Sixth World rely on the value of a player character’s Stats. You’ll hear more about these later on (especially when you get to the \nameref{Dossiers}), but every player character is described by 5 basic stats.

Each stat could fill in the blank in the following sentence: How \_\_\_\_\_\_\_\_\_\_ is my character?

\begin{easylist}
    # \textbf{Sharp:} \textit{alert, quick, perceptive, and instinctive}
    # \textbf{Hard:} \textit{ruthless, cold, and willing to do harm}
    # \textbf{Steady:} \textit{focused, cool, and mentally and physically tough}
    # \textbf{Skilled:} \textit{educated, trained, skillful, and intelligent}
    # \textbf{Smooth:} \textit{stylish, appealing, and charismatic}
\end{easylist}

And two pools of points that fluctuate in the course of play:

\begin{easylist}
# \textbf{Essence:} your life force and (meta)humanity, this also fuels the powers of magical archetypes (Adept, Mage, and Shaman)
# \textbf{Edge:} a pool of points used to boost you when you need it, or bail you out in a tight spot. Your Edge starts out at zero, but gain in through Experience, if you live long enough. You gain one Edge every third Advance, instead of new move or stat boost.
\end{easylist}

% \subsection{Sag's ihnen ins Gesicht}



\section{Rolling The Dice}

In this game, the dice rolling revolves around the concept of the Move. When you are instructed to roll dice for a move, your responsibility is simple: roll 2d6, and add the value of a stat (or sometimes some other value) to the result. When a roll is needed, it is usually phrased as \skill{roll+Stat}, where ``Stat'' is the value of your characters stat to add to the roll.

\textbox{Example}{If you are told to \skill{roll+Steady}, you would roll 2d6, sum the total, and add the value of your Steady stat to the result.}

The total of the roll indicates the outcome of the action taken by the character:

\begin{easylist}
    # On a 10+, you achieve a strong success: you've achieved your aim without complication, and to the fullest extent possible.
    # On 7-9, you have achieved a weak success: your achieve your aim, but with a cost. You will usually be presented with a list of complications to choose from, although sometimes instead the GM will tell you what complication occurs.
    # On a total of 6 or less, you have failed: you don’t get what you want. In fact, things are probably going to get worse.
\end{easylist}

Note that if a move just says ``roll'', then you don’t add anything — just roll 2d6. In addition to the common 2d6 roll, Sixth World uses the other common polyhedral dice: d4, d6, d8, d10, and d12. Twenty-sided dice are not used for mechanics, but can be used for some of the random generators at the end of this document.


\subsection{ROLL MODIFIERS}
While the basic move roll is 2d6+(something), there are a few modifiers and tricks that may apply to a roll. The rules will always indicate when to use one of these modifiers.

\textbf{boosted:} whenever you are boosted, your result is never lower than 7 (even if you roll 6 or less). So, when boosted, you cannot fail, though success may still come at a cost (not least of which is the fact that while boosted, you can’t receive Karma while boosted).

\textbf{glitched:} glitched rolls are the opposite of boosted rolls. Whenever you are glitched, your result is never higher than 9, even if you rolled a 10+. You can succeed while glitched, but it will always come with a cost.

\textbf{hold \textit{n}:} when you are told to Hold n, or that you gain n Hold, this means you have a small pool of points that can be spent at some future moment of your choosing. You will be told on what, specifically, you may spend the Hold. Note that if you can spend Hold on a dice roll, you can do so after you see the results of the roll!

\textbf{take +n forward/-n forward:} this means take a bonus (the +) or a penalty (the -) equal to n to your next Move.

\textbf{take +n ongoing/-n ongoing:} this means to take a bonus or penalty equal to n to all of your future rolls, until whatever circumstances caused the ongoing modifier have changed.

\textbf{b:} this means ``take the best of'' - you roll multiple dice, but keep only one of them to determine the final total. For instance, if you are instructed to roll 2d6b, you would roll 2d6, and keep the highest die. When written by itself (without a dice expression) it will be written as [b].

\textbf{w:} this means ``take the worst of'' - if you are instructed to roll 2d6w, then you would roll 2d6 and keep the lowest die. When written by itself (without a dice expression), it will be written as [w].


\section{Essence}

Every character in Sixth World has a stat called essence, representing their humanity, life force, and mystical connection with the world.

Characters start with 6 essence, although this may be reduced through the installation of cyberware. Essence can also be lost to some creatures and to certain injuries, depending on what optional rules you have in effect.

Essence is an important characteristic in three ways:

\begin{easylist}
    # It fuels the Adept’s powers, the Mage’s spells, and the Shaman’s spirits.
    # It acts as a limit on the amount of cyberware any character can carry.
    # It may be the thing that saves your life when the chips are down. See the \skill{Last Chance} move.
\end{easylist}


\section{Edge}

Each character has a pool of points called Edge. Edge is an in-game currency representing a number of real-world (or at least, game-world) concepts, from luck to experience to their ability to turn a bad situation into a survivable one.

\subsection{SPENDING Edge}
The main way to spend Edge is to gain bonuses to rolls. When a player wishes it, they can spend Edge as follows:

\begin{easylist}
    # To improve damage: for every point of Edge spent, they can add +1d6 damage to their most recent attack.
    # To boost a roll: a character can spend one point of Edge to be boosted on their next roll.    
\end{easylist}

Edge refreshes at a rate of one point per day (assuming a good night’s rest).

\subsection{EARNING Edge}
When you start play, your Edge value is zero. You gain one Edge every third Advance, instead of normal advancement.


\section{Karma}

Characters earn Karma (typically called ``Receiving Karma'') as they navigate the shadows, get in fights, and survive their adventures in the Sixth World. Characters can receive Karma in the following circumstances:

\begin{easylist}
# when they finish a run, or a significant portion of a major run
# when they resolve one of the debts or favors they have with another character
# when they are manipulated (see page 4) by another character
\end{easylist}

Once a character received 5 Karma, they may use the \skill{Advance} move to ``spend'' that Karma to improve their character. If you already have 5 Karma marked, you don’t receive any more Karma until you advance. You're not Buddha and can only learn so much at a time.


\section{Debts \& Favors}

Nobody goes it alone in the shadows for long. Sooner or later, you need to get help from somebody. Sometimes, you can buy that help with money. Other times, legal tender won’t cover it and that’s when debts and favors come into play.

The total number of Debts \& Favors you have with another character on your team equals your Bond with that character. For example, if you have 2 debts to another character, your Bond with them is 2. If, at the end of a session, you have resolved one of these bonds, you erase the debt or favor, and you and the other runner receive Karma.

\subsection{DEBT}
A debt is something you owe a fellow runner. Maybe they yanked your ass out of a bad situation down in Aztlan, or helped spring you from jail, or just lent you some of their own hard-won experience that saved your bacon.

\subsection{FAVOR}
A favor, conversely, is something owed to you by a fellow runner. Maybe you were the one doing the hot-LZ extraction in Aztlan, or you took the rap for them on a particular smash ‘n grab job.

Debts and favors are not necessarily reciprocal! A character might perceive a debt to another that is entirely self-imposed. Conversely, a character might feel like one of their teammates owes them something, while that teammate might be completely unaware of that feeling. So, when establishing debts and favors, don’t assume that a debt on one sheet has to correspond to a favor on another.

    \chapter{Moves} \label{moves}

\epigraph{\textit{Watch your back. Shoot straight. Conserve ammo. And never, ever, cut a deal with a dragon.}}{-- Street Proverb}

In Sixth World, the place where rules and fiction intersect are the character’s Moves. Moves are the mechanical structure used when the fictional actions of a character require some resolution, and where the outcome of such actions is sufficiently interesting - or in doubt - as to be worth taking a risk to achieve.

It is tempting to think of moves as a character’s ``powers'' or ``abilities'', but remember: you should not be looking for a move to make. Instead, you should describe fictional actions that fit the circumstances, and when those actions trigger a move, then you engage the game mechanics to determine the outcome.

There are four general categories of moves in Sixth World: Core, Secondary, Metatype, and Archetype.

Core moves are the most commonly used moves, and provide mechanics for frequent activities like fighting, hiding, looking around, and interacting.

Secondary moves are less frequently used, and are usually situational.

Archetype moves are moves unique to one of the character archetypes, and reflect their particular abilities.

Metatype moves are moves that reflect the differing traits of the five human metatypes in the game. Core, secondary, and metatype moves are detailed on the following pages. Archetype moves can be found in the dossier for each archetype.

\clearpage
\section{Core Moves}

\paragraph{CHECK THE SITUATION -} When you check out a charged situation, \roll{Sharp}. On a hit, you can ask the GM questions. Whenever you act on one of the GM’s answers, take +1 forward.

On a 10+, ask 3. On a 7–9, ask 1:

\begin{easylist}
    # Where’s my best escape route / way in / way past?
    # Which enemy is most vulnerable to me?
    # Which enemy is the biggest threat?
    # What should I be on the lookout for?
    # What’s my enemy’s true position?
    # Who’s really in control here?
    # Ask the GM a question of your own. If they will answer it, it stands; otherwise, retract it and ask another.
\end{easylist}


\paragraph{READ A PERSON -} When you read a person in a charged interaction, \roll{Sharp}. On a 10+, hold 3. On a 7–9, hold 1. While you’re interacting with them, spend your hold to ask their player questions, 1 for 1:

\begin{easylist}
    # Is your character telling the truth?
    # What’s your character really feeling?
    # What does your character intend to do?
    # What does your character wish I’d do?
    # How could I get your character to \_\_\_\_\_?
    # Ask their player a question of your own. If their player will answer it, it stands; otherwise, retract it and ask another.
\end{easylist}

The player must answer truthfully.


\paragraph{RECALL -} When you consult your knowledge of a specific topic or determine facts about your environment, \roll{Skilled}. On 10+, the GM will tell you give you a useful, specific detail about the situation. On 7-9, the GM will give you a general impression.

\paragraph{FUCK IT UP / MAKE IT RAIN -} When you aid or interfere with someone you have Bond with, roll+ your Bond with them. On 10+, they are boosted or glitched, your choice. On 7-9, they’re still boosted or glitched, but you are exposed to danger or retribution.

\paragraph{SEDUCE / MANIPULATE -} When you try to seduce or manipulate someone, tell them what you want and \roll{Smooth}.
For NPCs: on a hit, they ask you to promise something first, and do it if you promise. On a 10+, whether you keep your promise is up to you, later. On a 7–9, they need some concrete assurance right now.
For PCs: on a 10+, both. On a 7–9, choose 1:

\begin{easylist}
    # if they do it, they receive Karma
    # if they refuse, they have to Stay Frosty 
\end{easylist}

What they do then is up to them.


\paragraph{MAKE ‘EM SWEAT -} When you impose your will on someone through violence or threat thereof, \roll{Hard}. On a 7+, they choose one:

\begin{easylist}
    # do what you say
    # get the hell out
    # attack you
\end{easylist}

On a 10+, you also take +1 forward against them. On a miss, they do what they want (or, if it’s an NPC, the GM makes their move), and you take -1 forward against them.


\paragraph{ROCK \& ROLL -} When you attack an enemy, \roll{Hard}. Determine the outcome based on the range at which you attack:

\begin{easylist}
    # Melee Combat: on 10+, you hit and deal your damage. On 7-9, you hit and deal damage, but your target attacks you as well.
    # Ranged Combat: on 10+, you hit and deal your damage. On 7-9, you deal damage, but (choose 1):
    ## you must expose yourself to danger or attack
    ## you burn up ammunition; mark off 1 ammo
    ## you only graze the target (-2 damage)
\end{easylist}


\paragraph{STAY FROSTY -} When you try to stay frosty in the face of pain, danger, urgency, impatience, or emotion \roll{Steady}. On 10+, you succeed. On 7-9, you succeed, but the GM will present you with a worse outcome, hard bargain, or ugly choice.


\paragraph{I'VE PLANNED FOR THIS - } When you come upon a situation or obstacle where a certain object or item is needed to proceed, roll \roll{Sharp}. On a 10+, you've got it with you. On a 6-9, you have it, but (choose 1):
\begin{easylist}
    # you trigger an unauthorized access, a silent alarm, etc. by using it
    # you damage the obstacle somehow
    # there's somebody or something coming around the corner, behind that door, etc.
\end{easylist}
\textit{Anm.: im Grunde, glaub ich, sehr ähnlich zu \move{Stay Frosty}, aber halt wesentlich spezifischer. Not sure yet wie ich das finde; N.S. darf ruhig weniger catch-all-ig sein, aber am Ende so 5-6 sehr ähnliche moves zu haben, ist ja auch eher meh...}


\section{Secondary Moves}

\paragraph{ADVANCE}
When you have downtime and have marked 7 Karma, you can spend time reflecting on your experiences and honing your skills. When you Advance, choose one of the following:
\begin{easylist}
    # increase a stat (each stat may be advanced only once; check the small box in the stat area to indicate a stat that has already been advanced)
    # gain a new move from your dossier
    # gain a move from another Archetype’s dossier (up to three times)
\end{easylist}
You may only choose one benefit each time you advance. However, you can choose a benefit multiple times, subject to the limits specified above. Once you have advanced, clear your Karma track.

Your every third Advance, you get a point of Edge instead.

% \paragraph{AWAKEN}
% You become attuned to the deeper mysteries of the magical world. You can now spend Essence to power abilities and spells. This move, obviously, can only be taken once.

\paragraph{CITATION NEEDED}
When you research something, \roll{Skilled}. On 10+, you spend 1 day searching, and locate a useful detail about the topic of the research. On 7-9, you locate a useful detail, but (choose 1):

\begin{easylist}
    # you end up in a rabbit warren of information; spend 1 additional day digging through it
    # your search raises a flag in someone else’s systems (the GM determines whose)
    # the information is in hardcopy, and you need to go to it; spend 1 additional day on the search
\end{easylist}

\paragraph{FIRST AID}
When you try to keep a teammate from dying from their wounds, \roll{Skilled}. On 10+, you stabilize your teammate. On 7-9, you stabilize them, but (choose 1):
\begin{easylist}
    # you can’t move them to cover
    # you expose yourself to danger (take 2 damage)
    # their wounds force you to Stay Frosty
\end{easylist}
On a failure, your teammate cannot be saved.

\paragraph{GUT CHECK}
When you check off your 8th wound box, \roll{Steady}. On 10+, you stay on your feet, and if the damage you just received would take you beyond 8 boxes, ignore any excess. On 7-9, as above, but (choose 2):
\begin{easylist}
    # you take -2 ongoing to all moves
    # you’ll pass out in a few moments (you’ll have time for 1 or 2 moves, tops)
    # you’re making it worse; First Aid moves to help you take -1
\end{easylist}
On a failure, you collapse unconscious. If you were taken down by a weapon dealing stun damage, you are merely unconscious. Otherwise, you require first aid to stabilize you.

\paragraph{HIT THE BOOKS}
When you spend time training, practicing, or studying your abilities, you gain Prep. You gain 1 Prep for every 2 days spent in training or practice. When that training and preparation pays off, you can spend 1 Prep to get +1 to any roll. You can only spend 1 Prep per roll.

\paragraph{LAST CHANCE}
When you teeter on the brink of death and have no options left, permanently sacrifice at least one point of Edge, and roll + the amount sacrificed. On 10+, you’ll pull through somehow—you just won’t let go of life that easy. On 7-9, you will survive, but the GM will privately discuss with you what terrible bargain you agreed to in order to live. On a 6 or less, nothing can save you.

If you survive, your maximum Edge is reduced by 1 point (this can reduce your Edge to 0). Edge may be regained by Advancing as normal.

\paragraph{OVERWATCH}
When you’re providing cover for an ally and a threat appears, \roll{Sharp}. On 10+, your ally gets the drop on the threat. On 7-9, they’re alerted, and take +1 forward to their next move. On a miss, the threat gets the drop on your ally.

\paragraph{PULL STRINGS}
When you hit up a contact for info, items, or assistance, \roll{Smooth}. On 10+, the contact provides useful information (related to their own knowledge) or assistance. On 7-9, the contact provides information or assistance, but (choose 1):
\begin{easylist}
    # has to get back to you; wait 1 day
    # isn’t happy about it; take -1 forward to the next time you Pull Strings with this contact
    # requires a favor in return
\end{easylist}
If you fail, your contact doesn’t want to see you for a while, and will not return calls or meet with you for 1d6+1 days. Repeated failures of this move can permanently sever your relationship.

\paragraph{POP PILLS}
When you indulge in a drug, \roll{Steady}. On a 10+, you experience the effects as normal. On 7-9, you experience the effects but you got a weak batch, so the effects last half as long.

If you roll snake eyes when you pop pills, you become addicted to the drug. If you go 3 sessions without a hit, roll 2d6w. If you roll a 4 or higher, you are no longer addicted; otherwise, you’re still hooked. If you are an addict and roll snake eyes while popping pills, you overdose and take 8 Stun.


\paragraph{SUPPRESSIVE FIRE}
When you suppress an area to pin the enemy down down, \roll{Hard} and mark off 2 Ammo. On 10+, the targets are suppressed and cannot move or return fire. On 7-9, the only most of the targets are suppressed or they are only mostly suppressed.

\paragraph{TAKE A BULLET}
When you stand in defense of another, \roll{Steady}. On 10+, the attack hits you intead. On 7-9, the attack partly hits you (you take half damage, or half the attack’s effect, if non-damaging).



\section{Metatype Moves}
There are five primary metahuman types (or ``metatypes'') in the Sixth World: Human, Dwarf, Elf, Ork and Troll, each with their own unique moves. When you choose your metatype, you get a passive trait and choose one of it's moves.

\subsection*{Human}
\textit{homo sapiens sapiens} -- Humans can choose from the following moves:
\paragraph{Just Lucky:} you start with an extra point of Edge.
\paragraph{Privilege:} when interacting with humans, take +1 when you \roll{Smooth}.

\subsection*{Dwarf}
\textit{homo sapiens pumillionis} -- All dwarves have natural low-light vision and can choose from the following moves:
\paragraph{Never Sick:} you are immune to disease and poisons.
\paragraph{Tonight We Drink:} if you’re drinking with someone, you may manipulate someone using Steady instead of Smooth.

\subsection*{Elf}
\textit{homo sapiens nobilis} -- All elves have natural low-light vision and can choose from the following moves:
\paragraph{Ethereal:} take +1 forward to persuade or seduce someone.
\paragraph{Uncanny Grace:} once per fight, when you take damage, \roll{Sharp}. On 10+, reduce damage by half. On 7-9, reduce damage, but take -2 forward.

\subsection*{Ork}
\textit{homo sapiens robustus} -- All orks have natural low-light vision and can choose from the following moves:
\paragraph{Hard bastard:} take +1 forward to gut checks.
\paragraph{Fearless:} take +1 forward to stay frosty in the face of fear.
\paragraph{Streetfighter:} the first time you attack an enemy with a non-lethal weapon (fists, feet, batons, etc), you are boosted.

\subsection*{Troll}
\textit{homo sapiens ingentis} -- All trolls have natural thermographic vision and can choose from the following moves:
% Thermographic Vision: when you Check the Situation, you may ask one additional question from the list.
\paragraph{Dermal Bone Plating:} you have +1 armor. (idk, give this by default?)
\paragraph{You’ll Just Make It Angry:} you gain 1 additional wound box.
\paragraph{Juggernaut:} your fists should be licensed weapons. You deal 1d6 lethal damage in unarmed combat.


\section{Multiclassing}

You can choose moves freely from other archetypes, subject to the following two restrictions:

\begin{easylist}
    # You may choose no more than 3 moves from another archetype.
    # If your character is a non-magical archetype, they may not select moves that require Essence to be spent. They may select moves in which Essence expenditure is optional, however (although those usually don’t have much benefit without it).
\end{easylist}

    \chapter{Character Creation}

Creating a character is a multi-step process (don’t worry, though, it’s pretty easy). The overall process is described here; more detail is provided in each Archetype’s dossier. You’ll record the details you create on the dossier page or the supplemental “extra info” page.

\begin{enumerate}
    \item \textbf{Choose your archetype} \\
    There are 10 Archetypes to choose from: Adept, Face, Ex-Cop, Decker, Mage, Mercenary, Rigger, Shaman, Street Doc, and Street Samurai. You can learn more about them in the Dossiers that follow.
    
    \item \textbf{Choose your Metatype and Moves} \\
    There are 5 metatypes: Human, Dwarf, Elf, Ork, and Troll. Each metatype offers a choice of Metatype Moves.
    
    \item \textbf{Choose your Looks} \\
    Each character archetype will present options for look; you are free to make up your own as well.
    
    \item \textbf{Choose your Name and Street Name} \\
    Pick a real name and street name. You may use the lists provided in the GM Resources, or create your own.
    
    \item \textbf{Assign your Stats} \\
    You have an array of points to distribute among your stats: +2, +1, +1, 0, -1. Assign each of these freely to one of your \refname{stats}: \textit{Sharp, Hard, Steady, Smooth, Skilled}
    
    \item \textbf{Choose Equipment} \\
    Each archetype will present various weapon, spell, cyberware, and equipment options. Choose one item from each list (unless the list indicates that you may choose more than one item). Some choices, particularly cyberware, are optional.
    
    \item \textbf{Determine your Essence and Edge} \\
    Your starting Essence is equal to 6 – the Essence cost of any cyberware you have installed or choose to install. Your starting Edge is zero.
    
    \item \textbf{Choose Contacts} \\
    Everybody knows somebody. You will be presented with a list of potential contacts your character might know as a result of their experiences both before and after they became shadowrunners.
    
    \item \textbf{Establish Debts and Favors} \\
    It's dangerous in the shadows and nobody survives there for long completely on their own. Luckily, chummers help each other out - albeit not out of pure altruism, of course. A runner might live a life of freedom when compared to your average wage slave but that doesn't mean there ain't no strings attached. Take a look at the section on \refname{debts and favors} for details on how to establish them.
    
    \item \textbf{Starting Moves} \\
    Your character knows all the Core and Secondary Moves. You character also knows one or more of his or her Archetype moves. If you are given an option to choose additional moves, check off the box next to them on the character sheet.
    
    \item \textbf{Advancement} \\
    Each time you fail a roll - that is, you roll a 6 or less - you receive Karma. When you mark 5 Karma and you have downtime, you can make the Advance move.
\end{enumerate}



\clearpage
\section{Dossiers} \label{Dossiers}
\subsection{THE ADEPT}

\subsubsection{CREATING YOUR ADEPT}
\begin{enumerate}
    \item Follow through general steps 2 to 6. Some suggestions for Adept looks: \textit{wise eyes, wary eyes, glowing eyes; no hair, cropped hair, long braid; clean skin, tattooed skin, hard skin; perfect body, heavy body, lithe body}
    
    \item \textbf{Record your equipment} \\
    Basic Equipment: commlink \\
    Armor (choose 1): leather armor, arcane armor \\
    Weapons (choose 2): paired Ares Predators, katana, bo staff, paired combat knives, compound bow
    
    \item \textbf{Determine your Essence and Edge} \\
    Essence: 6 – total cost of all cyberware implants \\
    Edge: 0
    
    \item \textbf{Choose 2 Contacts} \\
    Temple master, gunsmith, underground fight club organizer, tea shop owner, yakuza soldier, talismonger
    
    \item \textbf{Establish Debts and Favors} \\
    Place one of your fellow runners’ names in at least one of the blanks below:
        \begin{easylist}
            ## If \_\_\_\_\_\_\_\_\_\_\_\_\_\_\_ hadn’t been there, I’d be dead right now.
            ## One day, I’ll make it up to \_\_\_\_\_\_\_\_\_\_\_\_\_\_\_ for letting that suspect walk.
            ## I let \_\_\_\_\_\_\_\_\_\_\_\_\_\_\_ skate on a serious charge once. 
            ## Letting \_\_\_\_\_\_\_\_\_\_\_\_\_\_\_ see that evidence earned me a formal reprimand.
        \end{easylist}
    
    \item \textbf{Starting Funds} \\
    You start play with 3d6 x 250\nuyen immediately available.
    
    \item \textbf{Starting Moves} \\
    You know all the Core and Secondary Moves. You also know the Enhanced Ability move, and one other Adept move.
\end{enumerate}


\subsubsection{ADEPT MOVES}
\paragraph{Enhanced Ability -} when you spend uninterrupted time (an hour or so) in quiet contemplation of your abilities, you gain +1 ongoing to one Stat of your choice, as long as you’re conscious or until you meditate again.

\paragraph{Gunfighter -} when you Rock \& Roll while wielding one or two handguns, you may spend 1 essence. In addition to the usual results of Rock \& Roll, choose 1:
    \begin{easylist}
        # you maneuver quickly and precisely, giving yourself the best shots possible while minimizing your opponents’ advantage; take +1 forward to Rock \& Roll
        # one of your targets is suppressed; take +1 forward to Stay Frosty
        # you grab an opponent and use them as a human shield; split any damage taken between you and the enemy
        # you physically strike an enemy within melee range with your weapon, dealing 1d6 stun
    \end{easylist}

\paragraph{Killing Hands -} when you deal damage while unarmed, you can chose to deal lethal damage instead of stun. In addition, you can spend 1 essence to roll damage twice and take the better value.

\paragraph{Danger Sense -} when you open your mind to the world of subtle mundane and magical information in your environment, spend 1 essence and roll+Sharp. On 10+, you cannot be surprised. On 7-9, take +1 to Stay Frosty.

\paragraph{The Sight -} when you take time to study an enemy, roll+Sharp. On 10+, take +1 forward or take +2 damage forward to your next attack. On 7-9, take +1 forward.

\paragraph{Astral Projection -} when you project your spirit into astral space, spend 1 Essence and roll+Steady. On 10+, you project successfully. On 7-9, you project, but your connection is tenuous; take -1 ongoing while in astral space.

\paragraph{Mystic Armor -} you gain +2 armor when naked or in normal clothes, or +1 armor when wearing mundane armor.

\paragraph{Traceless Walk -} your footsteps are silent and leave no trace, and you can walk on soft or brittle surface like snow, sand or broken glass without sinking. Whenever you’re trying to be sneaky and sound is of importance, you’re boosted.



\clearpage
\subsection{THE EX-COP}

\subsubsection{CREATING YOUR EX-COP}
\begin{enumerate}
    \item Follow through general steps 2 to 6. Some suggestions for Ex-Cop looks: \textit{cold eyes, tired eyes, wary eyes; close cropped hair, shaggy hair, bald; cheap suit, street clothes, hawaiian shirt; heavy body, fit body, injured body}
    
    \item \textbf{Choose your Equipment} \\
    Basic Equipment: commlink \\
    In addition, choose from the lists below: \\
        Armor: armor vest, form-fitting armor \\
        Service Pistol: Ruger Super Warhawk, Colt Manhunter \\
        Additional Weapon: HK 227, Remington 990 \\
        Installed Cyberware (optional): datajack (1 essence), cybereye with 2 enhancements (1 essence), skillwires (2 essence)
    
    \item \textbf{Determine your Essence and Edge} \\
    Essence: 6 – total cost of all cyberware implants \\
    Edge: 0
    
    \item \textbf{Choose 3 Contacts} \\
    Confidential informant (CI), precinct secretary, gang leader, prosecutor, journalist, former partner, defense attorney
    
    \item \textbf{Establish Debts and Favors} \\
    Place one of your fellow runners’ names in at least one of the blanks below:
        \begin{easylist}
            ## If \_\_\_\_\_\_\_\_\_\_\_\_\_\_\_ hadn’t been there, I’d be dead right now.
            ## One day, I’ll make it up to \_\_\_\_\_\_\_\_\_\_\_\_\_\_\_ for letting that suspect walk.
            ## I let \_\_\_\_\_\_\_\_\_\_\_\_\_\_\_ skate on a serious charge once.
            ## Letting \_\_\_\_\_\_\_\_\_\_\_\_\_\_\_ see that evidence earned me a formal reprimand.
        \end{easylist}
    
    \item \textbf{Starting Funds} \\
    You start play with 3d6 x 250¥ immediately available.
    
    \item \textbf{Starting Moves} \\
    You know all the Core and Secondary Moves.
    You know the Gumshoe move, and one other Cop move.
\end{enumerate}

\subsubsection{EX-COP MOVES}
\paragraph{Gumshoe -} when you examine the scene of an event, or interrogate someone about an event, roll+Sharp. On 10+, pick two of the following to learn (relevant to what you’re investigating). On 7-9, pick one:
    \begin{easylist}
        # \textbf{Scene:} when the events happened; whether magic was involved; how many individuals were involved; if this is the primary scene of the event
        # \textbf{Person:} if they’re connected to the event; whether they’re hiding something; what they stood to lose or gain; a useful personal detail (e.g, a tic, handedness, etc.)
    \end{easylist}

\paragraph{Work the System -} when you use your ex-LEO status to get help, roll+Smooth. On 10+, you have an old pal jam somebody up or cut them a break. On 7-9, you get the desired result, but (choose 1):
    \begin{easylist}
        # the person knows who helped or hindered them
        # your buddy got in trouble
        # your name got mentioned to the wrong ears        
    \end{easylist}

\paragraph{Takedown -} when you take control of a person physically, roll+Hard. On 10+, they are under your complete control. On 7-9, you gain control of them, but either you or your target must take 2 damage.

\paragraph{Interrogation -} when you attempt to make someone sweat, you may roll+Skilled instead of +Hard.

\paragraph{The Feds -} you have a connection in federal law enforcement. Roll+Smooth. On 10+, pick 2. On 7-9, pick 1.
    \begin{easylist}
        # You get a tip-off on a big operation so you can steer clear
        # You gain interesting and useful information about your current run
        # You get access to federal data on an individual
        # You are listed as a “consultant” on a case        
    \end{easylist}

\paragraph{Doorkicker -} when you lead the team in an assault on the enemy, roll+Steady. On 10+, designate up to 3 enemies who are surprised. On 7-9, designate up to 2 enemies.



\clearpage
\subsection{THE FACE}
\subsubsection{CREATING YOUR FACE}
\begin{enumerate}
    \item Follow through general steps 2 to 6. Some suggestions for Face looks: \textit{wise eyes, jeweled eyes, laughing eyes; normal skin, perfect skin, synthetic skin; great smile, smoky stare, rugged good looks, regal bearing; fit body, compact body, androgynous body}
    
    \item \textbf{Choose your Equipment} \\
    Basic Equipment: commlink, fashionable clothing \\
    In addition, choose from the lists below: \\
    Armor: armorweave clothing, form fitting armor, light armor jacket \\
    Weapon: Colt L36, Beretta 101T, stun baton, taser \\
    Cyberware: datajack (1 essence), cybereyes with 2 enhancements (1 essence), hold-out cybergun (2 essence), voice modulator (1 essence)
    
    \item \textbf{Determine your Essence and Edge} \\
    Essence: 6 – total cost of all cyberware implants \\
    Edge: 0
    
    \item \textbf{Choose 4 Contacts} \\
    Club owner, Yakuza boss, car dealer, journalist, senator’s aide, money launderer, mafia capo, arms dealer, wealthy socialite
    
    \item \textbf{Establish Debts and Favors} \\
    Place one of your fellow runners’ names in at least one of the blanks below:
        \begin{easylist}
            ## \_\_\_\_\_\_\_\_\_\_\_\_\_\_\_ always answers my calls.
            ## \_\_\_\_\_\_\_\_\_\_\_\_\_\_\_ knows I screwed over their friend, and has never said anything about it.
            ## \_\_\_\_\_\_\_\_\_\_\_\_\_\_\_ hung me out to dry.
            ## I helped \_\_\_\_\_\_\_\_\_\_\_\_\_\_\_ lay low after that nasty business with Renraku.
        \end{easylist}
    
    \item \textbf{Starting Funds} \\
    You start play with 3d6 x 350¥ immediately available.
    
    \item \textbf{Starting Moves} \\
    You know all the Core and Secondary Moves. You know the Razor Insight move, and one other Face move.
\end{enumerate}

\subsubsection{FACE MOVES}
\paragraph{Razor Insight -} when you have a casual conversation with someone, roll+Sharp. On 10+, you learn three of the following. On 7-9, you learn 2.
    \begin{easylist}
        # Something they love        
        # Something they fear        
        # Something they need        
        # Something they lost        
        # Something they took
    \end{easylist}
If you use this information when fast talking, manipulating, or making them sweat, you are boosted.

\paragraph{Fast Talk -} when you need to convince somebody of something fast, roll+Smooth. On 10+, your quick thinking gets you through. On 7-9, they’re convinced, but (choose 1)
    \begin{easylist}
        # they check up on your story later
        # they get in serious trouble for believing you
        # one of your contacts somehow ends up involved…in a bad way   
    \end{easylist}
        
\paragraph{Build a Legend -} when you create a false identity, spend 1 day working on it and roll+Skilled. On 10+, your legend is solid and will hold up to any scrutiny. On 7-9, it holds up for now, but (choose 1):
    \begin{easylist}
        # you've only got 1d4+Skilled days before its blown
        # you run into someone who knows you…as someone else.
        # you have to do something unpleasant to maintain your cover.
    \end{easylist}

\paragraph{I Know A Guy -} when you need an illegal good or service, roll+Smooth. On 10+, you know someone who can get it for you immediately, and discreetly. On 7-9, they can get it, but (choose 1):
    \begin{easylist}
        # it takes 1 additional day
        # it costs twice as much as predicted
        # your fence has to drop your name to get it
    \end{easylist}

\paragraph{Honeyed Words -} when you make someone sweat, you may roll+Smooth instead of Hard.

\paragraph{Irresistible -} even if you anger, insult, or otherwise tick off a contact, they just can’t stay mad at you. They only avoid you for half as long as normal.



\clearpage
\subsection{THE DECKER}

\subsubsection{CREATING YOUR DECKER}
\begin{enumerate}
    \item Follow through general steps 2 to 6. Some suggestions for Decker looks: \textit{strange eyes, glasses, unfocused eyes; no hair, unkempt hair, mohawk, ponytail; pale skin, bad skin, tattooed skin; thin body, heavy body, compact body, flabby body}
    
    \item \textbf{Choose your Equipment} \\
    Basic Equipment: commlink, Fuchi Cyber-4 or Fuchi Cyber-7 \\
    Installed Cyberware: datajack (1 essence) \\
    In addition, choose from the lists below: \\
    Armor (choose 1): trenchcoat, light armor jacket \\
    Weapon (choose 1): Fichetti Needler, Ares Lightfire 70, Combat Axe, Remington 990 \\
    Cyberware: cybereyes with 2 enhancements (1 essence), synaptic hardening (2 essence)
    
    \item \textbf{Determine your Essence and Edge} \\
    Essence: 6 – total cost of all cyberware implants \\
    Edge: 0
    
    \item \textbf{Choose 2 Contacts} \\
    Electronics dealer, military decker, gang member, former professor, matrix guru, white hat, script kiddie, poker dealer, money launderer
    
    \item \textbf{Establish Debts and Favors} \\
    Place one of your fellow runners’ names in at least one of the blanks below:
        \begin{easylist}
            ## I did a run with \_\_\_\_\_\_\_\_\_\_\_\_\_\_\_ that went bad…because of me.
            ## If \_\_\_\_\_\_\_\_\_\_\_\_\_\_\_ hadn’t unplugged me, that IC would have fried my brain.
            ## I scrubbed \_\_\_\_\_\_\_\_\_\_\_\_\_\_\_’s arrest record; they’re pure as driven snow. For now.
            ## I don’t work for free. But \_\_\_\_\_\_\_\_\_\_\_\_\_\_\_ can be very convincing.
        \end{easylist}
    
    \item \textbf{Starting Funds} \\
    You start play with 3d6 x 150¥ immediately available.
    
    \item \textbf{Starting Moves} \\
    You know all the Core and Secondary Moves. You know the Born Digital and Sling Code moves.
\end{enumerate}

\subsubsection{DECKER MOVES}
\paragraph{Born Digital -} while in the Matrix, when you:
    \begin{easylist}
        # Stay Frosty: add your deck’s Mask rating to the roll
        # Take damage: subtract your deck’s Hardening rating from the damage
        # Rock \& Roll: roll+Skilled instead of +Hard
    \end{easylist}

\paragraph{Sling Code -} when you hack a Matrix node or device, roll+Sharp. On 10+, choose 3. On 7-9, choose 2:
    \begin{easylist}
        # The node or device remains unaware of the intrusion
        # You leave no trace of your presence
        # You don’t trigger IC
        # You learn a useful detail about another node connected to this one
    \end{easylist}
Once in control of a node, you can issue commands appropriate to it.

\paragraph{Matrix Overwatch -} when you defend a device or node against a matrix attack, roll+Steady. On 10+, the attack is ineffective. On 7-9, halve the damage or duration of the attack’s effect.

\paragraph{IC Killer -} when you inflict damage to IC, inflict +1 damage.

\paragraph{Multitasker -} you can hack multiple systems or devices simultaneously. Roll+Steady. On 10+, you suffer no penalties to hack two systems. On 7-9, take -1 ongoing to the second system.

\paragraph{Tracer -} when you would deal damage to an enemy decker in Matrix combat, you can instead forgo damage to plant a tracer tag on their avatar. This tracer is active for 1+Training days.



\clearpage
\subsection{THE MAGE}

\subsubsection{CREATING YOUR MAGE}
\begin{enumerate}
    \item Follow through general steps 2 to 6. Some suggestions for Mage Looks: \textit{blank eyes, unnatural eyes, piercing eyes; Long hair, bald, wild hair; robes, street clothes, dress clothes; thin body, weak body, muscular body}
    
    \item \textbf{Choose your Equipment and Spells} \\
    Choose from the lists below: \\
    Armor: trenchcoat, light armor jacket, armor charm \\
    Weapon: Beretta 101T, Ruger Super Warhawk, Staff
    
    In addition, Choose 3 of the following 5 spell categories: \\
    Combat, Detection, Illusion, Health, Manipulation
    
    You know 2 spells in one of your chosen categories, 1 in the other caategories each.
    
    \item \textbf{Determine your Essence and Edge} \\
    Essence: 6 – total cost of all cyberware implants \\
    Edge: 0
    
    \item \textbf{Choose 2 Contacts} \\
    Wage Mage, Corporate Exec, Fetishmonger, Paranormal Animal Expert, Bartender, Street Cop, Professor of Magical Theory
    
    \item \textbf{Establish Debts and Favors} \\
    Place one of your fellow runners’ names in at least one of the blanks below:
        \begin{easylist}
            ## I’d still be a wage mage today if \_\_\_\_\_\_\_\_\_\_\_\_\_\_\_ hadn’t made that call.
            ## Those gangers would have waxed me if \_\_\_\_\_\_\_\_\_\_\_\_\_\_\_ hadn’t happened along.
            ## I helped get rid of a curse. You believe that? A curse.
            ## I sucked up a manabolt for \_\_\_\_\_\_\_\_\_.
        \end{easylist}
    
    \item \textbf{Starting Funds} \\
    You start play with 3d6 x 250¥ immediately available.
    
    \item \textbf{Starting Moves} \\
    You know all the Core and Secondary Moves. You know the Cast a Spell, Center, and Counterspell moves.
\end{enumerate}

\subsubsection{MAGE MOVES}
\paragraph{Cast a Spell -} When you cast a spell, spend the required essence and roll. The stat you add depends on the type of spell:
    \begin{easylist}
        # Combat: roll+Hard
        # Detection: roll+Sharp
        # Illusion: roll+Smooth
        # Health: roll+Skilled
        # Manipulation: roll+Steady
    \end{easylist}        
On 10+, the spell is cast. On 7-9, the spell is cast, but (choose 1):
    \begin{easylist}
        # it causes drain; take 1 stun
        # it causes astral feedback; take -1 to the next spell you cast
        # you must expose yourself to danger or an attack to cast the spell
    \end{easylist}

\paragraph{Center -} when you take a moment to concentrate and restore yourself, regain 1d6 essence.

\paragraph{Spell Defense -} when you defend an ally from a magic spell, spend 1 Essence and roll+Sharp. On 10+, choose 2. On 7-9, choose 1:
    \begin{easylist}
        # halve the spell’s damage
        # halve the spell’s duration
        # locate the spell’s caster
        # deal 1d6 damage to the caster
    \end{easylist}

\paragraph{Astral Trace -} when you observe a magical effect for which you cannot determine the source, roll+Sharp. On a 10+, the GM answers three of the following. On 7-9, two:
    \begin{easylist}
        # In what direction does the source of this magic lie?
        # Approximately how far away is the source?
        # What is the general nature of the source?
        # How powerful is the source?
    \end{easylist}

\paragraph{Hermetic Library -} you have permission to access an extensive library of hermetic lore. When you or a teammate uses the Citation Needed move to research magical history or theory, the move is boosted.

\paragraph{Initiate -} when you hit the books, you may also spend Prep on:
    \begin{easylist}
        # reducing a spell’s Essence cost by 1 (to a minimum of 0)
        # boosting a Cast a Spell move
        # regain 1 Essence
    \end{easylist}


\clearpage
\subsection{THE MERCENARY}

\subsubsection{CREATING A MERCENARY}
\begin{enumerate}
    \item Follow through general steps 2 to 6. Some suggestions for Mercenary looks: \textit{dead eyes, cold eyes, soft eyes; boonie hat, high ‘n tight, ponytail, fauxhawk; combat fatigues, street clothes, nice suit; scarred skin, tough skin, soft skin}
    
    \item \textbf{Choose your Equipment} \\
    Basic Equipment: commlink \\
    Installed Cyberware: bone lacing (2 essence) \\
    In addition, choose from the lists below: \\
    Armor: ballistic vest, armor jacket, combat armor \\
    Weapon (choose 3): Ares Predator, Browning Max Power, AK-97K, Ingram Smartgun, Colt M22A2, AK-97, tomahawk, combat knife
    
    \item \textbf{Determine your Essence and Edge} \\
    Essence: 6 – total cost of all cyberware implants \\ Edge: 0
    
    \item \textbf{Choose 2 Contacts} \\
    Former CO, Terrorist Cell Member, Arms Dealer, Veterans Clinic Doctor, Old War Buddy, Street Pharmacist, Therapist
    
    \item \textbf{Establish Debts and Favors} \\
    Place one of your fellow runners’ names in at least one of the blanks below:
        \begin{easylist}
            ## \_\_\_\_\_\_\_\_\_\_ dragged me out when shit went sideways.
            ## \_\_\_\_\_\_\_\_\_\_ backed my play even when nobody else would.
            ## It was not fun explaining to my CO what happened to those weapons \_\_\_\_\_\_\_\_\_ "borrowed".
            ## First time I saw \_\_\_\_\_\_\_\_, it was at the other end of my gun.
        \end{easylist}
    
    \item \textbf{Starting Funds} \\
    You start play with 3d6 x 150¥ immediately available.
    
    \item \textbf{Starting Moves} \\
    You know all the Core and Secondary Moves. You know the Go Tactical move and one other Mercenary move.
\end{enumerate}

\subsubsection{MERC MOVES}
\paragraph{Go Tactical -} when you Check the Situation during combat, roll+Hard instead of +Sharp. On a 10+, instead of asking the GM questions, you may instead choose to Hold 3. On a 7-9, you may choose to Hold 1. You can then spend that Hold 1-for-1 to grant a bonus to any ally at any point during the combat.

\paragraph{Deadeye -} when you attack a surprised or defenseless enemy in ranged combat, you can deal damage or, name your target and roll+Hard:
    \begin{easylist}
        # Head: on 10+, you deal your damage and they fall to the ground, stunned. 7-9: they fall to the ground, stunned.
        # Arms: on 10+, you deal your damage, and they drop whatever they’re holding. 7-9: they drop whatever they’re holding.
        # Legs: on 10+, you deal your damage, and they are slowed or immobilized. 7-9: they are slowed or immobilized.
    \end{easylist}

\paragraph{Veteran -} when you Stay Frosty, you take +1.

\paragraph{Contracts Available -} you have contacts with a mercenary force or guild. Roll+Smooth. On 10+, they can pass you a contract worth 10,000¥. On 7-9, they can pass you a contract worth 5,000¥.

\paragraph{Field Trial -} when you use your military connections to acquire military- only equipment, roll+Smooth. On 10+, you’re able to borrow the equipment for 5 days. On 7-9, you borrow it, but (choose 1):
    \begin{easylist}
        # There’s an unscheduled inventory inspection before you can return it
        # You need to pony up a sizeable ``security deposit''
        # You got a hangar queen. The equipment requires 1 day of maintenance, or it will fail at a most inopportune moment.
    \end{easylist}

\paragraph{Inspiring -} when you roll a 10+ when you Stay Frosty, one ally who saw you can take +1 forward to their next move.




\clearpage
\subsection{THE RIGGER}

\subsubsection{CREATING YOUR RIGGER}
\begin{enumerate}
    \item Follow through general steps 2 to 6. Some suggestions for Rigger looks: \textit{goggles, alert eyes, obvious cybereyes; kaiser helmet, cowboy hat, pirate bandana; biker clothes, flight suit, street clothes, punk clothes; heavy body, built body, lean body}
    
    \item \textbf{Choose you Equipment} \\
    Basic equipment: commlink, 1 drone, 1 vehicle \\
    Installed cyberware: control rig (2 essence) \\
    Choose from the lists below: \\
        Armor: ballistic vest, lined coat \\
        Weapon (choose 2): Enfield AS-7, Browning Max Power, Ares Predator, AK-97K, combat axe \\
        Cyberware: cyberarm with 1 enhancement (2 essence), cybereyes with 2 enhancements (1 essence)
    
    \item \textbf{Determine your Essence and Edge} \\
    Essence: 6 – total cost of all cyberware implants \\
    Edge: 0
    
    \item \textbf{Choose 2 Contacts} \\
    Chop shop worker, go ganger, fence, trucker, arms dealer, mechanic, bartender, cargo pilot, car thief
    
    \item \textbf{Establish Debts and Favors} \\
    Place one of your fellow runners’ names in at least one of the blanks below:
        \begin{easylist}
            ## \_\_\_\_\_\_\_\_\_\_\_\_ tipped me off to some sweet (and lucrative) courier runs.
            ## When I ended up in the slam for the Dynagene job, \_\_\_\_\_\_ bailed me out.
            ## I wrecked my favorite ride working with \_\_\_\_\_\_\_\_\_\_\_\_. Took months to fix it.
            ## \_\_\_\_\_\_\_\_\_ jammed me up for a goddamned percentage.
        \end{easylist}
    
    \item \textbf{Starting Funds} \\
    You start play with 3d6 x 400¥ immediately available.
    
    \item \textbf{Starting Moves} \\
    You know all the Core and Secondary Moves. You know the Jumped In move and one other Rigger move.
\end{enumerate}

\subsubsection{RIGGER MOVES}
\paragraph{Jumped In -} while jacked into a vehicle or drone you own, when you:
    \begin{easylist}
        # Rock \& Roll or Stay Frosty: roll+Skilled
        # Check the Situation: add the vehicle or drone’s Sensor rating to the roll
        # Fail a move involving the vehicle or drone, mark off 1 Fuel.
        # Take an action not related to controlling the vehicle or drone, take -2.
    \end{easylist}

\paragraph{Autonomous Mode -} when you put a drone in autonomous mode, indicate which mode setting you want, and roll+Skilled. On 10+, hold 2 to be spent on the drone’s moves. On 7-9, hold 1. Drone mode settings (and the rolls they use for moves) are:
    \begin{easylist}
        # Sentry: the drone can make the Rock \& Roll move; roll+Tactical
        # Recon: the drone can make the Check the Situation move; roll+- Sensor
        # Evasion: the drone can make the Stay Frosty move; roll+Power
    \end{easylist}

\paragraph{Split Personality -} when you launch a drone, roll+Steady. On 10+, you don’t take the normal -2 penalty to non-drone moves while controlling it. On 7-9, the penalty is reduced to -1.

\paragraph{Jury Rig -} when you have to make fast repairs to a vehicle or machine, roll+Sharp. On 10+, you get it running again and fast. On 7-9, you get it running, but (choose 1):
    \begin{easylist}
        # it will only run for 1d10 minutes
        # afterwards, it will be a total loss.
        # one of its qualities is reduced by 1, permanently
    \end{easylist}

\paragraph{Percussive Maintenance -} when you smack the hell out of a recalcitrant device, roll+Hard. On 10+, the device springs to life. On 7-9, the device works for only a moment, but you know what you need to do to fix it. Take +1 forward to Jury Rig.

\paragraph{Paint the Target -} when you point out a drone or vehicle’s weakness to your teammates, they take +1 forward to attacks against it.



\clearpage
\subsection{THE SHAMAN}

\subsubsection{CREATING A SHAMAN}
\begin{enumerate}
    \item Follow through general steps 2 to 6. Some suggestions for Shaman looks: \textit{heterochromic eyes, wise eyes, sunglasses; long hair, dreadlocks, shaved head; street clothes, anachronistic clothes, biker gear; wiry body, thin body, round body}
    
    \item \textbf{Choose your Totem} \\
    Choose a totem from the list, or make up one of your own.
    
    \item \textbf{Choose your Equipment} \\
    Choose from the lists below: \\
        Armor: Leather jacket, defensive charm, riot shield \\
        Weapon: Ruger Super Warhawk, Colt Manhunter, AK-97, combat axe, crossbow \\
        Spirits: choose 3 spirits from the gear section
    
    \item \textbf{Determine your Essence and Edge} \\
    Essence: 6 – total cost of all cyberware implants \\
    Edge: 0
    
    \item \textbf{Choose 2 Contacts} \\
    Wage mage, ork underground, gang thug, street cop, herbalist, university professor, diner owner, fetishmonger, art dealer, hedge wizard, houngan
    
    \item \textbf{Establish Debts and Favors} \\
    Place one of your fellow runners’ names in at least one of the blanks below:
        \begin{easylist}
            ## \_\_\_\_\_\_\_\_\_ had me in his sights, and let me live.
            ## \_\_\_\_\_\_\_\_\_ put their life on the line helping me battle a wild spirit.
            ## When \_\_\_\_\_\_\_\_ fell foul of that corp hit squad, I provided additional security.
            ## Getting the artifact \_\_\_\_\_\_\_\_\_ wanted wasn’t easy.
        \end{easylist}
    
    \item \textbf{Starting Funds} \\
    You start play with 3d6 x 150¥ immediately available.
    
    \item \textbf{Starting Moves} \\
    You know all the Core and Secondary Moves. You know the Conjure and Banish moves.
\end{enumerate}

\subsubsection{SHAMAN MOVES}
\paragraph{Conjure -} When you summon a spirit, spend at least 1 essence and roll. The stat you add to the roll depends on the spirit’s nature:
    \begin{easylist}
        # Destroyer: roll+Hard
        # Protector: roll+Steady
        # Watcher: roll+Sharp
        # Teacher: roll+Skilled
        # Seducer: roll+Smooth
    \end{easylist}

On 10+, the spirit is conjured and will perform a number of moves equal to the essence spent. On 7-9, the spirit is conjured, but (choose 1):
    \begin{easylist}
        # It can perform one fewer moves (you cannot choose this option if you spent only 1 essence)
        # It is draining; take 1 stun
        # You must expose yourself to danger or an attack
    \end{easylist}

When the spirit has used all of its moves, you regain the essence committed to the summoning. If the spirit is destroyed, you regain half the committed essence, round down.

On a failure, the spirit does not manifest, and the essence spent is lost. If you roll snake eyes, the spirit is summoned in an uncontrolled state, and the GM will control its actions until it is exhausted or banished.

\paragraph{Banish -} when you attempt to banish a spirit, roll+Hard. On 10+, you reduce the spirit’s available moves by 1. On 7-9, you reduce the spirit’s moves by 1, but it deals half its damage to you. If you reduce the spirit’s available moves to 0, it vanishes immediately.

\paragraph{Commune -} when you take a moment to mentally commune with your totem, you may gain its boons and flaws, or regain 1d6 essence.

\paragraph{Favored Spirit -} choose 1 spirit type (Watcher, Teacher, Protector, Destroyer, Seducer). This spirit type performs one free move.

\paragraph{Aura Mask -} you may conceal your magical nature. Roll+Skilled. On 10+, you appear to be a mundane individual to anyone or anything that examines you. On 7-9, you appear mundane, but must spend 1 Essence to do so.

\paragraph{Spirit Master -} you may conjure multiple spirits simultaneously, dividing the commited Essence among them.



\clearpage
\subsection{THE STREET DOC}

\subsubsection{CREATING YOUR STREET DOC}
\begin{enumerate}
    \item Follow through general steps 2 to 6. Some suggestions for Street Doc looks: \textit{clear eyes, old eyes, quick eyes; close cut hair, stylish hairdo, bandana; fit body, heavy body, compact body; business attire, street clothes, EMT jumpsuit}
    
    \item \textbf{Choose your Equipment} \\
    Basic equipment: commlink, MedKit with 6 supply \\
    In addition, choose from the lists below: \\
    Armor: ballistic vest, armor jacket \\
    Weapon: Narcoject rifle, Browning Max Power, HK227, stun baton, combat knife \\
    Cyberware: cyberarm with 2 enhancements (2 essence), skillwires (2 essence) \\
    
    \item \textbf{Determine your Essence and Edge} \\
    Essence: 6 – total cost of all cyberware implants \\
    Edge: 0
    
    \item \textbf{Choose 2 Contacts} \\
    ER doctor, morgue staffer, medical examiner, DocWagon driver, organlegger, black market organ dealer, blood bank worker, pharmacist
    
    \item \textbf{Establish Debts and Favors} \\
    Place one of your fellow runners’ names in at least one of the blanks below:
        \begin{easylist}
            ## \_\_\_\_\_\_\_\_\_\_\_ helped me get clean.
            ## \_\_\_\_\_\_\_\_\_\_\_ got their hands bloody helping me save a life.
            ## I arranged for \_\_\_\_\_\_ to receive a “mis-shipped” case of pharmaceuticals.
            ## I extracted information from a prisoner once for \_\_\_\_\_\_\_\_\_\_\_.
        \end{easylist}
    
    \item \textbf{Starting Funds} \\
    You start play with 3d6 x 400¥ immediately available.
    
    \item \textbf{Starting Moves} \\
    You know all the Core and Secondary Moves. You know the Combat Medic and Stay With Me moves.
\end{enumerate}

\subsubsection{STREET DOC MOVES}
\paragraph{Combat Medic -} when you provide medical aid to a person, roll+Skilled and mark off 1 Supply from your kit. On 10+, the patient heals 2d4b damage. On 7-9, the patient heals 1d4 damage.

\paragraph{Stay With Me -} when you attempt to stabilize a teammate who is bleeding out, roll+Steady and mark off 2 supply from your kit. On 10+, choose 3. On 7-9, choose 2:
    \begin{easylist}
        # they can be moved without a stretcher
        # it takes fewer supplies than expected - mark off only 1 supply
        # you do not expose yourself to danger to help them.
        # they will not have a chronic injury
    \end{easylist}
Your patient does not die if you fail this move, and you may take -1 and try again. A second failure, however, results in the death of the patient.

\paragraph{Grace Under Fire -} when you are working on a patient during a fight but not actively fighting, you have +1 armor.

\paragraph{We All Bleed Red -} when you take time to treat an injured enemy, mark off 1 supply and roll+Smooth. On 10+, they’re stable, and you can ask two questions which they will answer truthfully. On 7-9, you can ask only one question.

\paragraph{Pharmacy Is Open -} when you use a contact to obtain medical supplies (amounting to +1 supply), and roll+Smooth. On 10+, choose 2. On 7-9, choose 1:
    \begin{easylist}
        # you get +2 supply instead of +1
        # it takes 1 day to get the supplies instead of 2
        # nobody notices the supplies are missing
        # you receive an interesting piece of information as well
    \end{easylist}

\paragraph{You Got This -} whenever you walk someone through a medical procedure (such as first aid), roll+Smooth. On 10+, they are boosted. On 7-9, they take +1.



\clearpage
\subsection{THE STREET SAMURAI}

\subsubsection{CREATING A STREET SAMURAI}
\begin{enumerate}
    \item Follow through general steps 2 to 6. Some suggestions for Street Samurai looks: \textit{glowing eyes, silvered eyes, hard eyes; cropped hair, wild hair, topknot; tattooed skin, scarred skin, camo skin; bulky body, lithe body, skinny body}
    
    \item \textbf{Choose your Equipment} \\
    Basic Equipment: commlink, lined coat \\
    In addition, choose from the lists below: \\
    Armor: form-fitting armor, ballistic vest \\
    Weapon: choose four weapons from the list of melee and small arms \\
    Cyberware: choose up to 5 essence worth of cyberware
    
    \item \textbf{Determine your Essence and Edge} \\
    Essence: 6 – total cost of all cyberware implants \\
    Edge: 0
    
    \item \textbf{Choose 2 Contacts} \\
    Arms dealer, cybersurgeon, bartender, street clinic nurse, private investigator, dockworker, pilot, cab driver, retired runner, survival nut
    
    \item \textbf{CREATE YOUR CODE} \\
    The word “samurai” means something on these streets. Create the code of honor that you follow.
    
    \item \textbf{Establish Debts and Favors} \\
    Place one of your fellow runners’ names in at least one of the blanks below:
        \begin{easylist}
            ## \_\_\_\_\_\_\_\_\_\_ came back for me.
            ## Even with all this chrome, \_\_\_\_\_\_\_\_ still treats me like a real person.
            ## I got this scar taking a bullet for \_\_\_\_\_\_\_\_\_\_.
            ## \_\_\_\_\_\_\_\_\_\_’s “big score” ended with me in the lockup.
        \end{easylist}            
    
    \item \textbf{Starting Funds} \\
    You start play with 3d6 x 250¥ immediately available.
    
    \item \textbf{Starting Moves} \\
    You know all the Core and Secondary Moves. You know The Only Thing Faster is Light move and one other Street Samurai move.
\end{enumerate}

\subsubsection{STREET SAMURAI MOVES}
\paragraph{The Only Thing Faster is Light -} whenever you Rock \& Roll, on a 12+ you may deal your damage to a second target within range.

\paragraph{More Power -} when you attempt to bend, break through, or otherwise destroy something, roll+Hard. On 10+, you easily achieve your goal. On 7-9, you break it, but (choose 1):
    \begin{easylist}
        # It takes longer than expected
        # It makes a lot of noise
        # You take 1 stun in the process
    \end{easylist}

\paragraph{Pain Editor -} when you make a Gut Check, you are boosted. Additionally, when you reach 9 or more wounds, you may choose to accept a chronic injury rather than bleeding out. If you already have all of the chronic injuries, you cannot use this move.

\paragraph{Honorable -} when you uphold a tenet of your code, roll+Smooth. On a 10+, hold 2. On 7-9, hold 1. You may spend this hold to pull strings, manipulate, or make someone sweat.

\paragraph{CQC Expert -} when you Rock \& Roll using a melee weapon or while unarmed, deal +1d4 damage.

\paragraph{Perfect Instincts -} when you act on GM’s answers after Checking a situation, take +2 instead of +1.

\paragraph{Dodge This -} when you manage to get out of an enemy’s line of sight, roll+Steady. On 10+, you get the drop on that enemy when you reappear. On 7-9, you take +1 forward against that enemy when you reappear.


\clearpage
\section{Debts \& Favors}

In your life before and after becoming a shadowrunner, you’ve worked with a lot of people, and ended up owing, or being owed, by them. These relationships include at least one of your fellow shadowrunners, and are described by debts and favors. When you are instructed to create your debts and favors with fellow runners, you’ll see a list of sample statements to help you create them. You don’t have to use these; they’re simply suggestions.
    
To create a debt or favor, place the name of one of the other characters in the blank space in one of the statements presented. You can place the same name more than once (that is, in more than one sentence), but you must establish at least one debt or favor to start with. Collectively, debts and favors are known as Bond. Later, during play, you may end up resolving a bond with someone. If you do, both of you receive Karma.
    \chapter{Combat} \label{combat}

Es geht um nicht weniger als die Beteiligung der Runner am Unabhängigkeitsprozess von Seatlle. Nach den Erignissen in/um Detroit und Ares und inmitten des Tumults der Aufkündigung der BRA von den UCAS und den Blackouts an der Ostküste werden die Runner angeheuert, um Informationen im Rahmen eines Diplomatiegipfels zu beschaffen.

\section{Szene 0: Interludium}

Auf dem Weg heim, durch die Mall: BREAKING NEWS - Arthur Vogel's Statement.

\gesicht{Detroit Free Press: BREAKING NEWS -- Wir unterbrechen unser reguläres Programm, um zum aktuellen Stand der Lage in Detroit zu berichten. Der neue Vorstandsvorsitzende von Ares Macrotechnology, Arthur Vogel, gibt ein erstes Statement nach Detroit-Krise ab.}

Yu lässt sich die Muskeln straffen - 3 Tage knocked out - genug für ein Wissenstalent.

Tusk will 'ne Monofilament-Peitsche haben. Best Bet: Benjamin Flowers. ``Nicht, dass ich dir in deine Ethik quatschen will, aber dir sollte klar sein, dass solche Teile eher als endgültige Lösung gedacht sind - und nicht gerade diplomatisch?'' -- ``Ich denke, sowas lässt sich auftreiben, aber dafür bist du mir einen Gefallen schuldig.''

Außerdem ist da noch die Challenge, Wapeka "Skillful" Becerra zu besiegen, um die Initiation zu beginnen:

\textbox{Wapeka}{
    Ini: 16, 
    NK: 13, 
    VT: 11 (11), 
    Waffe: Stab (4B, 10 AT), 
    Sonst: Athletik 12, 
    HP: 12K/12B
}

Rude bekommt von Hez mitgeteilt, dass sein Block aufgekauft wurde - von Vivaldi Immobilien Ltd., if that helps anything.

\subsubsection{Wrapup}

\begin{easylist}
    # Tusk hat Wapeka besiegt und kann jetzt ihre Initiation beginnen.
    # Yu hat die OP gut überstanden: +3 \skill{Geschick}, not bad at all
    # Rude macht sich Gedanken zur Runner-WG, ahem, zu einem gemeinsamen Safehouse; an den Grenzen von Puyallup oder Redmond in angenehmere Gegenden
\end{easylist}




\section{Preludium: UC\texorpdfstring{\textsubscript{r}AS\textsuperscript{h}}{UCrASh}}

\gesicht{Knapp 3 Wochen sind vergangen, seit die Krise in Detroit für beendet erklärt wurde. Aber ruhig blieb es nicht, ganz im Gegenteil: keine drei Tage später haben die UCAS als Reaktion auf den Bruch mit Ares eine metaphorische Atombombe gezündet, indem sie unilateral den BRA - den \textit{Business Recognition Accords} - aufgekündigt haben. In den BRA ist festgesetzt, dass der Konzerngerichtshof als alleinige Instanz über die Exterritorialität von Konzernen entscheidet. Die UCAS haben den AAA-Megakonzernen also nicht weniger als ihre Geschäftsgrundlagen wie einen Teppich unter den Füßen weggezogen. Im dadurch entstandenen Chaos, auch oder sogar besonders in Seattle, wurden dadurch die Rufe nach der Klärung der Frage nach Seattles Unabhängigkeit lauter. Eine der vielen Nachrichtenmeldungen zu dem Thema lautete zB folgendermaßen: }


\subsection{Die Lage in Seattle}

Aus \textit{Free Seattle - Einleitung} - alles hier gehört direkt \textit{\textbf{ins Gesicht}}.
 
\subsubsection{Seattles Zukunft steht auf dem Spiel}

Vor etwas mehr als einem Jahr wurde die Gouverneurin von Seattle, Corinne Potter, mit einem Programm gewählt, das ihr die Stimmenmehrheit brachte, aber nur wenige Versprechungen enthielt. Brackhaven hatte ihr viele ungelöste Probleme hinterlassen, und Potter versprach, sich um jedes einzelne zu kümmern, auch wenn ihre Kampagne nur wenige Details dazu enthielt. Seit ihrer Wahl hat Potter Berater und Experten hinzugezogen, um zu versuchen, die richtige Lösung für jedes dieser Probleme zu finden. Mit ihren Entscheidungen hat sie ihre Wähler manchmal verstimmt, da diese vielleicht erwartet hatten, dass sie eher auf ihr Herz als auf ihren Verstand hört, aber jetzt kümmert sie sich um das vielleicht kontroverseste Thema ihrer Kampagne: Seattles Unabhängigkeitsbewegung. 
Während der Wahl gab es mehrere Forderungen nach einem freien und unabhängigen Seattle. Diese Rufe stießen fast gleichermaßen auf Zustimmung und Ablehnung. Seattle ist in dieser Frage eindeutig tief gespalten. Um ihr bei der Lösung dieser Krise zu helfen, hat Potter Vertreter mehrerer Länder und Megakonzerne eingeladen, die ein besonderes Interesse daran haben, ob Seattle unabhängig wird oder bei den UCAS verbleibt.
Es überrascht nicht, dass Potter mehrere Vertreter des Konzerngerichtshofs eingeladen hat, insbesondere Major Brenda Reed von Ares, Thomas Miranda von Horizon und Takuto Nakagawa von Renraku; dazu kommt noch die Pacific Prosperity Group, die Wuxing-Exec Dewei T’ao an den Verhandlungstisch geschickt hat. Darüber hinaus haben Seattles Nachbarn ein Mitspracherecht: Der Salish-Shidhe-Rat hat John Abernathy vom Salish-Stamm geschickt, die UCAS haben den frischgewählten Kongressabgeordneten Carl Derrick entsandt, um sicherzustellen, dass die Dinge so bleiben, wie sie sind, und Tír Tairngire wird durch Margaret Telestrian vertreten. Die vielleicht überraschendste Einladung zum Treffen ging an die Seedrachin, die auf der Konferenz noch nicht erschienen ist. Was die Anwesenheit eines der umstrittensten Drachen der Welt für diese Konferenz bedeutet, lassen wir mal dahingestellt. 
Im Laufe der nächsten Woche wird Seattles Zukunft zum Teil von diesen Parteien bestimmt werden. Nur die Zeit wird zeigen, ob die Stimme von Seattles Bevölkerung wichtiger sein wird als die Stimmen von Seattles Megakonzernen.

\vspace{1em}
\textbf{An Tusk}: Deine Connection April Summers schrieb in einem Kommentar dazu: 

\gesicht{Das ist vielleicht das klügste politische Spiel, das Potter spielen konnte. Sie kann öffentlich Unterstützung sammeln und bei einigen Mächten das Terrain sondieren, und wenn es schiefläuft, kann sie immer noch den anderen die Schuld zuschieben. Allerdings bedeutet das auch, dass sie das Rampenlicht meidet und nicht die volle Anerkennung für einen möglichen Erfolg erhält. Sie verspielt einen Teil der Publicity des Erfolgs, um die Kritik an einem möglichen Scheitern zu dämpfen. Das wiederum könnte letztendlich als rückgratlos angesehen werden.}


\subsection{Die Lage der Runner}

All das ist natürlich auch in eurem Umfeld ein großes Thema: Yu, du weißt zB von Mia, dass sie Feuer und Flamme für ein unabhängiges Seattle ist:

``Was haben die in DC denn jemals für uns getan? Ihre scheiß Armee sorgt nur für mehr Spannung zwischen dem Council und den anderen NANs als für irgendwelchen angeblichen Schutz! In keinem anderen Metroplex der Welt herrscht ein solches Machtgleichgewicht wie hier, keiner der Megas hier hat dsa Sagen - nicht Ares, keiner der Japanokons, \textbf{niemand} - nichtmal das goldene Würmchen hat hier viel zu melden. Und das liegt sicher nicht an DC, sondern allein an \textbf{uns}, an den Seattler Schatten. Wir sind \textbf{niemandes} Schoßhunde, hier suchen \textbf{wir} uns aus, für wen wir laufen.''. 

Rude, im Burning Hole werden Wetten darauf abgehalten, welche Repräsentanten am Leben bleiben und du hast mit Hez darauf gewettet, dass Seattle unabhängig wird - die Chancen stehen 2:1, auch wenn dort absolut niemand auch nur im Ansatz genug von der Sache versteht, um das einschätzen zu können. Du weißt auch, dass Hez es lieber wäre, wenn es nicht dazu kommt: ``DC ist zwar auch nur ein weiterer Sumpf, aber wenigstens ein weit entfernter. Und immer noch besser, als wenn die Kons hier komplett den Laden übernehmen. Ich mein, die Containment Zone in Chicago strahlt nach wie vor stärker als die Sonne über Fujiyama. Und guck dir doch an, wie es in Detroit aussieht: alles liegt in Schutt und Asche und was macht Ares? Sie verschwinden. Einfach so. Aus ihrer \textit{eigenen} Amerikanischer-als-Uncle-Sam-Vorzeigeenklave. Einfach, weil ein Wiederaufbau zu teuer ist. Und von den anderen fangen wir am besten gar nicht an. Du weißt genauso gut wie ich, wieviele Trogs es bei den Japanos auch nur in der kleinsten Tochter gibt - nicht einen einzigen.''




\section{Szene 1: Glanz, Gosse, Gloria}

Mia hat die Runner zu sich gerufen, um einen Job zu vermitteln; Gilroy 'Romeo' Steele ist auf die Runner aufmerksam geworden und sucht ein Team, dass für die Dauer des Gipfels verschiedene Aufträge für die Gipfelteilnehmer erledigen und ihm Bericht erstatten wird.


\subsection{Das Pan Pacific Hotel}

\gesicht{Das Hotel könnte beeindruckender nicht sein. Der Weg zum Eingang führt bereits an einem perfekt manikürten Design-Vorgarten vorbei. Spätestens, als ihr das Foyer betretet, ist klar: Ihr betretet das luxuriöseste Hotel, das ihr jemals gesehen habt: das Pan Pacific Seattle. Alles hier ist luxe, De-luxe, extra-luxe. Wenn es möglich wäre, dass die Einrichtung noch mehr Grandeur vermittelt, hätte sie eine eigene Postleitzahl.}

\skill{Wahrnehmung:}

\subparagraph{(1)} Der Stil ist neo-modernistisch gehalten: Smaragdapplikationen auf den Möbeln, echte Messingglocken an Telefonen, die Laufburschen tragen Fez, die Zimmermädchen Rüschen-Schürze. 
\subparagraph{(2)} In diesem perfekten Luxus erscheint es fast etwas seltsam, dass hier ein Hausmeister am Fahrstuhl mit einem Mop den Boden wischt. 
\subparagraph{(3)} du hast den Eindruck, dass er euch beobachtet
\subparagraph{(4)} er hat eine cyberbuchse hinter dem Ohr, ungewöhnlich für einfache Hausmeister (wenn yu das erfährt: erinnerung 2: iwas mit hiding in plain sight)


\subsection{Gilroy 'Romeo' Steele}

Das Team soll offiziell bestimmte Aufgaben für die Gipfelteilnehmer erledigen. Danach sollen die Runner Romeo über jeden einzelnen dieser Jobs berichten: was dabei jeweils passiert ist \textbf{und was das Team über den jeweiligen Job denkt}.

Er wird das Team mit einem Safehouse in \textit{Puyallup}, der elfischen Version der Barrens, ausstatten. Romeo möchte, dass die Runner während der Aufträge, die sie ausführen, auch sämtliche verwertbaren Informationen und Erkenntnisse über ihre Auftraggeber sammeln, die sie finden können.

Er bezahlt für alle Daten, die die Runner über die Teilnehmer ausfindig machen können, egal, wie sie sie bekommen. Sie können die Informationen von Connections, lokalen Quellen, Aufzeichnungen, Datenspeichern oder Zeugen erhalten. \textbf{siehe \textit{Legwork} zu den Personen als exzellente Quelle für Informationen, die ich irgendwo einbauen kann!} 

\gesicht{``Wie ihr an diese Infos kommt, ist mir egal. Ihr könnt eure Connections bemühen, lokalen Quellen ausfindig machen, Aufzeichnungen und Datenspeicher auftreiben oder Zeugen befragen, also lasst euch was einfallen. Und vor allem: seid aufmerksam und lasst euch nichts durch die Finger gehen! Ihr kennt ja das Sprichwort: Haltet euch den Rücken frei, spart Muni, zielt genau und lasst euch nicht mit Drachen ein.}

Erster Auftrag: einen Kontakt auf der Gala treffen (\probe{Einfluss}{gg Romeo} um zu erfahren, wen). Am besten früh erscheinen, keine Waffen.


\subsection{Puyallup Safehouse}

\subsubsection{Sag's ihnen ins Gesicht}

Romeo hat sein Wort gehalten und dafür gesorgt, dass ihr für die Dauer eures Aufenthaltes in einem Stadthaus in Puyallup wohnen könnt. Puyallup ist nicht gerade das beste Viertel, nur etwas besser als die Redmond Barrens. Es ist eine Gemeinschaft der Entrechteten, vor allem Elfen, die aber fast jeden akzeptieren, der ein Verstoßener ist. Es ist ein großartiger Ort, um sich zu verstecken, aber es ist auch gefährlich und unberechenbar. Wie das eben so ist, wenn man auf einer Vulkancaldera sitzt. 

Einige der Straßen sind mit recyceltem Kunststoff und gekühlter Lava gepflastert. Die Straßen sind an manchen Stellen eben, an anderen unwegsam, und gelegentlich haben sich Felsen und Krater gebildet. Nicht gut zum Fahren oder Gehen, aber besser als gar kein Pflaster.

\textbox{Wahrnehmung:}{
    \subparagraph{(1)} Ok, die Gegend ist ein Slum... Die Wände sind voller Risse, die Fenster regelmäßig zerbrochen. Alles ist voller Graffitis. 
    \subparagraph{(2)} Du kannst einige davon ausmachen; zB \textit{Soykaf ist Menschen!} oder \textit{Azzies raus, Ancients rein!} Tatsächlich auch verblasste Logos der Gang, aber auch die vieler anderer. 
    \subparagraph{(3)} die meisten Logos, und auch die neusten, sind eine Art stilisierte Krabbe, oder ein Hummer. Irgendein Krustentier zumindest. (roll Gang Knowledge: (3) Rock Lobsters)
}

Euer Safehouse scheint ein verlassenes Stadthaus zu sein. Die Fenster sind mit einer dicken Schmutzschicht bedeckt, und die Fassadenverkleidung ist stellenweise abgefallen. Aber die Türschlösser sehen stabil aus, und die Substanz des Hauses scheint gut erhalten zu sein. Das deutet darauf hin, dass das schmuddelige Aussehen beabsichtigt ist. 

Ihr findet die formelle Kleidung in einem der Schlafzimmerschränke. Sie steht in scharfem Kontrast zu der abblätternden Farbe und den fehlerhaften AR-Displays im restlichen Haus. Aber ihr habt schon an schlimmeren Orten gewohnt. Ihr habt noch ein paar Stunden zum Ausruhen und vorbereiten, bevor ihr zum Ballsaal des Pan Pacific Hotels fahrt.


\subsection{Die Gala}

Das Hotel ist bereits früh am Abend von riesigen Scheinwerfern beleuchtet, manche Gäste kommen offenbar per Hubschrauber - einer landet gerade im ``Vorgarten'' - andere in dicken Bonzenkarren. 

\subsubsection{Sag's ihnen ins Gesicht}

Als ihr den Ballsaal betretet, seid ihr sofort von der hier zur Schau gestellten Opulenz überwältigt. Die Leute tragen Kleidung, die mehr kostet als der Monatslohn eines Lohnsklaven. Die Nahrung ist echte Nahrung, kein Soja oder Mykoprotein. Umgeben von einer Menschentraube steht eine strahlende Elfe, die zu euch herüberkommt, als sie euch erblickt. Ihr Kleid funkelt und leuchtet, und winzige Sterne lassen eine Aura der Schönheit um sie herum entstehen. Ihr kurz geschnittenes Haar betont die Länge ihrer Ohren, und sie nähert sich euch mit dem anmutigen, selbstsicheren Schritt eines Raubtiers.

\subsubsection{Der Run}

Im Grunde bin ich hier ziemlich durch gerushed; die anderen hatten kaum Möglichkeitenm, was zu tun (mangels Beschreibungen auch) und ich hab völlig allerlei Dinge vergessen: Bezahlung? Was genau klauen? Wo abgeben? Siehe v.a. Wrapup Szene 2.

Telestrian will alles über die Pläne des SSC in den Verhandlungen wissen. Dafür zahlt sie 2.000\nuyen pro Kopf und lässt sich (ggf.) auf 2.500, d.h. 100 pro Nettoerfolg hochhandeln.




\section{Szene 2: Council Island}

Beschreibung von Council Island; Beschreibung der Situation vor Ort; welche Paydata liegt wie/wo vor; Karte und Icons vorbereiten.

\subsubsection{An den Docks}

\gesicht{Ihr fahrt zu den Docks von Tacoma und findet schnell Pier 25. Am Ende des Piers tanzt ein kleines U-Boot auf dem Wasser. Das U-Boot bietet gerade so Platz für drei Passagiere - erst recht, wenn einer davon ein bulliger Troll und eine weitere eine kräftige Orkin ist. Es ist also an der Zeit, zu kuscheln. Zum Glück seid ihr kein Team das auf schwere Hardware oder Panzerung setzt, denn dafür wäre hier eindeutig kein Platz gewesen.}


\subsection{Die Botschaft}

\textbf{Hauptziel} ist: ein Textdokument über die Pläne des SSC während der Verhandlungen (siehe \textit{Legwork - John Abernathy}). 

Bonus A: ein Ordner zur Telestrian Industries Corporation. 

Bonus B: ein weiterer Ordner über Dewei 'Dewey' T'ao.

\subsubsection{Und sonst so...}

\begin{easylist}
    # Wissensproben sagen was über:
    ## Die Ausrüstung der Wachen (wg. Protesten)
    ## Möglichkeiten, Kameras zu umgehen
    ## Druckgeflechte
    ## die Sicherheitsspinne
    ## den Aufbau der Botschaft: 
    ### $\rightarrow$ zwei Stockwerke mit identischem Layout
    ### $\rightarrow$ Botschafterbüro vmtl in prominenter (mittiger) Lage im oberen
    # Hauptziel liegt auf dem Schreibtisch; ja, einfach so
    # beide Boni fallen mit \probe{Wahrnehmung}{3} auf
    ## Bonus A in einer verschlossenen Schublade (Stufe 2: ``ein einfaches, mechanisches Schloss; von den meisten als kurios bezeichnet, von euch als 'leicht zu öffnen'.'')
    ### $\rightarrow$ zum Öffnen: Dietrich-Set und \probe{Mechanik}{2}
    ## Bonus B in einem Safe (Stufe 4) hinter einem Gemälde des ersten SSC Häuptlings Jon Moses
    ### $\rightarrow$ erst \skill{Wahrnehmung}, dann \probe{Mechanik}{4} + Würfel aus der vorherigen WN
\end{easylist}


\subsection{Wrapup}

Dinge, die in \textbf{Szene 1 und 2} passiert sind:

\begin{easylist}
    # es gab 20.000\nuyen Vorschuss von Romeo
    # die Runner haben für 1500\nuyen einen GMC Bulldog gemietet
    # Rude, Frosty und Tusk haben Actioneer(?) Anzüge bekommen; zumindest aber angemessen formelle Kleidung
    # die Bezahlung für den Botschafts-Run beträgt 2.200\nuyen pro Kopf
    # Beute vom Einbruch:
    ## Akte ``Unabhängigkeitsstrategie Seattle'' (\textbf{Hauptziel}, s. \textit{Legwork - John Abernathy})
    ## Dossier ``Telestrian Industries Corporation'' (\textbf{Bonus A}, s. \textit{Legwork - Margret Telestrian})
    ## eine Flasche Rum, wenn Yu das zugelassen hat
    ## eine Schachtel Zigarren
\end{easylist}


\section{Szene 3: Dirty Laundry}

Konfrontation mit den \textbf{Rock Lobsters}? Debriefing mit Romeo. Neuer Auftrag von Major Brenda Reed; Bunraku Salon schonmal vorbereiten.


\subsection{Gang }

Wir starten in den Ubooten von \textit{Council Island} nach \textit{Tacoma}. Auf der Fahrt, sowie der Fahrt zurück zum Safehouse haben die Runner ca. 2,5 Stunden, um ihr Material zu sichten. Bei ihrer Ankunft werden sie von den Rock Lobsters angegangen.

\gesicht{Es ist eindeutig, dass ihr immer weiter nach Puyallup hinein zurückkehrt: die Gebäude werden heruntergekommener, die Straßenlöcher häufiger und größer, der Geruch von Meer uns Salz weicht dem von Asche und Staub. Als ihr euren Wohnblock erreicht, fallen euch mehrere Grüppchen auf, die um brennende Fässer herumstehen.}

\textbox{Wahrnehmung:}{
    \subparagraph{(1)} Ein ganz normaler Spätsommerabend in den Slums. 
    \subparagraph{(2)} Seit einer Weile fährt auch ein weiteres Fahrzeug ein Stück weit hinter euch - und scheint euch zu verfolgen. 
    \subparagraph{(3)} Tatsächlich sind es mehrere Quads und Crossbikes, auf denen jeweils auch mehrere Leute sitzen. 
}

Vor dem safehouse steigen sie aus: 8 Leute, offensichtlich Go-Ganger. Auf manchen der Synthlederjacken ist hinten ein ähnliches Logo zu erkennen, das euch schon tagsüber als sehr präsent aufgefallen ist: ein stilisierter Hummer.


\subsection{Debriefing}

Romeo ist am nächsten morgen in der Küche und hat Kaffee gekocht - zumindest falls keine Wache bestimmt wurde.

1. Was denkt ihr über Margaret?

2. Wie ist der Run abgelaufen; was wollte Maggie? Gab es Probleme?

3. Was habt ihr über John Abernathy herausgefunden?

4. Und über andere Parteien bzw. deren Abgesandte?

5. Was haltet ihr von dem ganzen? Denkt ihr, der SSC wusste Bescheid? Dass Maggie weiß, was der SSC über sie und Telestrian weiß?

Es gibt insgesamt 9.000\nuyen: die mit Magaret ausgehandelten 2.200 pro Kopf + 2.400 für das Telestrian Industries Corp. Dossier.

``Bereitet euch schon mal auf den nächsten Run vor. Im Laufe des Tages wird es wieder was zu tun geben.''


\subsection{Der Run}

\gesicht{Am Nachmittag, gegen 15 Uhr, ertönt ein Klopfen an der Tür. \textbf{Reaktion abwarten!} Ihr macht eure Waffen bereit und schaut vorsichtig durch den Türspion. Auf der anderen Seite steht ein gelangweilt aussehender Teenager – ein Zwergenmädchen mit Pickeln im Gesicht und einem rosafarbenen Iro, das etwas kaut, von dem ihr hofft, dass es Kaugummi ist. Als ihr die Tür öffnet, gibt sie euch einen Datenchip, dreht sich um und geht weg, ohne ein Wort zu sagen. Auf dem Datenchip seht ihr eine Nachricht von einer Frau, die keinem von euch sofort bekannt vorkommt: eine Asiatin mittleren Alters. Ihr Gesicht ist streng und ernst, was durch ihren Dutt
noch betont wird, und man erkennt sogar auf der Trid-Projektion noch die harten, flachen Muskeln
einer Athletin. Sie sagt, ihr sollt sie ``Major'' nennen und sie will, dass ihr in einen Bunraku-Salon, dem \textbf{Cherry Patch} in \textit{Nord-Tacoma}, geht und euch dort um einen der Kunden kümmert. Die restlichen Details befinden sich auf dem Datenchip.}

Die Zielperson ist Carl Derrick, der Abgesandte der UCAS und die Mission besteht darin, einen Weg zu finden, ihn dazu zu bringen, eine der Prostituierten zu töten. Die Bezahlung beträgt 2.500\nuyen pro Person und bergeweise dreckige Wäsche, von der zumindest ausgegangen werden kann, dass sie in einem solchen ``Upper Class'' Bunraku-Salon zu finden ist. 



\subsection{Change of Plans?}

Die Runner stehen vor einem moralischen Dilemma. Well, und ein Spieler mimimi't mit dem Setting (\textit{nach} mehrfachem Fragen ob wetwork ok sei, ah well...).

\subsubsection{Status Report}

\begin{easylist}
    # Carl Derrick ist Abgeordneter der erzkonservativen Partei, äußerst patriotisch.
    # Infos über Reed sind den Runnern bekannt; Gerüchte über Kriegsverbrechen sind nicht öffentlich.
    # Die Runner haben einen Datenchip mit einem Trideo, in dem sie von einer deutlich erkennbaren Major Reed zum Mord/Blackmail aufgefordert / angeheuert werden
    # Sie können dem einfach nachgehen (Abentuer verläuft wie vorgesehen)
    ## Magische Beeinflusing von Derrick; leider nicht vorhanden
    ## Hacken der Persona-Chips (und des Hosts); leider kein Decker verfügbar
    ## 
    # Sie können Carl Derrick
\end{easylist}


\subsubsection{Vor dem Salon}

\gesicht{Der Salon befindet sich in einem beschäftigten Bezirk in Nord-Tacoma. Die Straßen sind voller Leute, die geschäftig ihren Vorhaben nachgehen, Werbung strahlt in knaliggem AR- und Neon-Bild und Ton auf die Straße, von den Ramenständen und ähnlichen hört man hektisches Treiben. In der Luft liegt ein Geruch von Abgasen, altem Fett und Asphalt, der vom letzten Regenschauer trocknet.}

\textbf{Bild zeigen!} 

\textbox{Wahrnehmung:}{
    \subparagraph{(1)} Der Bunraku-Salon selber ist, wenig überraschend, als solcher kaum zu erkennen. Er wirkt eher wie eine hochklassige Cocktailbar. Davor steht allerdings ein äußerst massiger, äußerst großer Mann mit Glatze, offensichtlich asiatischer Abstammung. (falls die Runner den Eingang zunächst beobachten, fällt hier auch auf, dass jeder, der rein will, einen kurzen Austausch mit dem Steher hat (\probe{}{2}: evtl. eine Losung für den Eintritt)
    \subparagraph{(2)} Dir bemerkst einen offensichtlich betrunkenen Gast, der aus dem Laden torkelt (\probe{Menschenkenntnis}{2}: ...und dir fällt auf: wer raus kommt, muss vorher auch reingekommen sein.)
    \subparagraph{(3)} An der Seite, in Griffreichweite des Türstehers, siehst du - etwas bedeckt - etwas, das ein Gehstock oder ein Regenschirm sein könnte. Du erkennst aber: es ist ein Katana, das in einer Scheide steckt. Wer nicht wirklich genau hinschaut, würde das nicht sehen.
}

Der Türsteher hält euch auf: ``Passwort?''; oder eben 5.000\nuyen.


\subsubsection{Im Salon}

Nachdem die Angelegenheit mit dem Zutritt auf die ein oder andere Art gelöst wurde, können die Runner den Salon betreten:

\gesicht{Ihr betretet zunächst eine Art Vorraum, der bereits relativ dunkel und eher modern als traditionell gehalten ist. Eine weitere Tür bringt euch in den eigentlichen Salon: von einer relativ niedrigen Decke vermittelt dämmeriges, blaues Licht ein Gefühl von Diskretion. Es gibt eine Full-Service Theke mit etwa einem Dutzend Hockern, an der zwei vereinzelte Gäste sitzen und im Hauptbereich sorgt eine Tänzerin für Unterhaltung, während die Anwesenden Kunden an Tischen auf ihre Suite warten - einige der anwesenden hier sind allerdings offensichtliche Yak-Enforcer. Im hinteren Bereich führen Treppen nach oben, offenbar zu besagten Suites und auf der anderen Seite der Treppe (nicht gegenüber, sondern quasi \textit{unter} der Treppe) befindet sich eine weitere Tür.}

% Wenn sich die Runner umschauen, werden sie Lucky Strike sehen?

Derrick selbst wird abgeklärt sein. Und ein (patriotisches) Arschloch. Er weiß von ihren allmorgendlichen Läufen und hat von ihrer Nahkämpferei gehört.


\subsubsection{Wrapup}

Yu hat konnte sich mit Derrick unterhalten, ihm das Nitro ins Getränk mischen und anschließend hinterher schleichen. Rude und Yu haben einen Enforcer ausgeschaltet und stehen nun vor der Tür von Derrick's Suite. Tusk sitzt draußen im Van und hat D's Gorilla-1 und den SUV im Blick. Blöderweise habe ich angekündigt, dass KE schon mit fetter Mannschaft anrollen wird, wenn sie einen entsprechenden Tip bekommen. Ergibt halt keinen Sinn, aber das ist nun, wie es ist...

\begin{easylist}
    # Derrick wird seinen Panic-Button drücken (wollen)
\end{easylist}





\clearpage
\section{Hauptdarsteller}

\subsection{Gilroy 'Romeo' Steele}

Zwerg, Information Broker, Spin-Doctor

\subsection{Margaret Telestrian}

vertritt das Tír Tairngire (``Tier Ta'en'gier''), eine der Vorsitzenden von Telestrian Industries.
Sie hat die hohen Wangenknochen und spitzen Ohren, die die meisten Elfen haben, und sie schneidet sich bewusst die Haare kurz, um diese Unterschiede hervorzuheben.
    \chapter{Magic} \label{magic}

In the Sixth World, the magic has returned to the world, and dormant powers have reawakened. Magic is fueled by Essence, one of the variable point pools each character has.



\section{The Awakening of Metahumanity} % oder so. ESSENCE und ESSENCE RECOVERY sind zumindest keine eigenen sections...

\subsection{Essence}
Three archetypes in the game - the Adept, the Mage, and the Shaman - are magically gifted, which means that they are able to spend their Essence to use their magical abilities.

\textbf{The Adept:} adepts turn their magical ability inward to improve themselves, sometimes to superhuman levels. An adept spends essence to temporarily modify his or her capabilities (for example the Enhanced Ability or Killing Hands moves).

\textbf{The Mage:} when a mage spends essence to power a spell, the player marks off the spent essence. That essence is not available for future spells until it is recovered. The amount of essence spent is a general indicator of the power, or Force, of the spell.

\textbf{The Shaman:} when a shaman spends Essence to summon a spirit or elemental, they are in effect committing or wagering some amount of essence to do so.

The Essence spent indicates the number of services the spirit will perform (mechanically, the number of moves the spirit may make before dissipating). A Shaman allocates this essence at the time of summoning. If the summoning fails, the wagered essence is lost. If the summoning succeeds, then the essence is “tied up” until the spirit is dispelled/destroyed.


\subsection{Essence Recovery}

All magic users may recover essence by resting. A substantial rest (usually a night’s sleep) will recover all Essence spent. Some archetypes have additional means of recovering essence, as described below:

\textbf{The Mage:} mages may use the Center move to recover some Essence without resting, simply by taking a moment to concentrate and recenter him- or herself.

\textbf{The Shaman:} because the essence used to summon a spirit is in effect a wager, when the spirit has performed its actions (or is dispelled by the shaman who summoned it), the essence “tied up” in the spirit immediately returns to the shaman. If the spirit is dispelled by another person, or destroyed, only half the wagered essence (round up) is recovered.



\section{Astral Space}

Much like the Matrix, Astral Space is a sort of alternate universe adjacent to our own. It is where spells, spirits, magical creatures, wards and more reside. When an individual perceives the Astral, they can see the entities existing in Astral Space. All three arcane archetypes can astrally perceive. In addition, they can perceive emotional auras of living beings, as well as background magical nature of the area. When an individual projects themselves into astral space, they transfer their consciousness from their physical body to the astral plane, and can fully interact with other Astral entities and traverse great distances. The Shaman and Mage can astrally project. The following effects occur while perceiving or projecting: 

\textbf{Perceiving:} while astrally perceiving, take -2 ongoing to any moves in the physical world.

\textbf{Projecting:} you cannot take action in the physical world (your body is unconscious and helpless).



\section{Astral Quests}

The Astral also serves as a huge deposit of magical information, though most of the deepest knowledge is hidden in the metaplanes. Metaplanes are the planes beyond the Astral, the real sources of all magic. Every metaplane has a citadel, a core of pure magical energy that can alter the magical world. Accessing it can let you destroy a spirit permanently, learn some information such as the true name of a spirit, or learn an individual’s true aura. Note, however, an astral quest may only have a single goal. Astral Quests are also dangerous in that you are stuck in a metaplane until you either complete your Quest or fail. You can’t give up, and you can never go back, only forward.


\subsection{Domains}
To go on an Astral Quest, you must visit various metalocations known as domains, similar to Nodes in the Matrix (in fact, mapping these domains is a useful tool to keep play on track and engaging). The number and nature of these domains depends on the quest you are undertaking, but each one presents a challenge the character must complete in order to move on to the next domain. This could be fierce combat, a riddle, a puzzle or any variety of things.

Minor quests usually have 3 or 4 domains, while major quests can have up to 10 or more, all of which lead, ultimately, to the Citadel, where the quester will find the object or information they seek. Moving from domain to domain is as simple as willing yourself there once the task in the current domain is completed.


\subsection{The Dweller}
The first domain you encounter is always the Domain of the Dweller, a mystical being who blocks the entrance to the metaplanes. The Dweller knows everything about the quester, and will always question the nature your quest before granting passage. The Dweller is an enigmatic trickster, but if you go on quests often, you’ll get to know this being quite well.






\section{Spells}

Like other equipment, spells (although they’re not exactly “equipment”) are described in terms of tags. Spells has the following special tag:
Essence: the minimum Essence expenditure required to cast the spell.
element: the spell has effects related to a particular element (e.g. fire, electricity, etc.)
sustainable: this spell may stay in effect as long as essence is committed to it. A caster cannot use the Centering move while sustaining a spell.
exhausting: this spell is quite difficult to cast; take 1 stun when casting it


\subsection{Combat}
Mana Bolt: deals 1d8 damage (bypassing armor) to creatures or spirits at short/medium ranges. Tags: range s/m/l, dmg 1d8, ignores armor, essence 2

Fire bolt: deals 1d6 damage and fire effects to creatures at short/medium range. Tags: range s/m, dmg 1d6, fire, essence 1

Lightning bolt: deals 1d8+1 damage and shock effects to creatures at short range. Tags: range s, dmg 1d8+1, shock, essence 2

Acid Stream: deals 1d8 damage and acid effects to targets and objects at short and medium range. Tags: range s/m, acid, dmg 1d8, essence 2

Fireball: deals 1d8+1 damage and fire effects to all creatures and objects in an area within short range. Tags: range s, fire, area, dmg 1d8+1, essence 3

Manaball: deals 1d8+1 damage (bypassing armor) to creatures and spirits within the target area. Tags: range s, area, dmg 1d8+1, ignores armor, essence 3

Knockout: deals 1d8 stun (bypassing armor) to creatures in touch range. Tags: range t, dmg 1d8 stun, ignores armor, essence 2


\subsection{Detection/Perception}
Analyze Device: take +1 to your next move involving the device being analyzed, or learn what the device does. Costs 1 essence. Tags: range touch, essence 1

Clairvoyance: when you Check the Situation, you can ask questions about a location you cannot see within the range of the spell. Tags: range c/s/m/l, essence 3

Combat Sense: while you sustain this spell, you cannot be surprised, and take +1 to your first Rock \& Roll or Stay Frosty move when combat starts. Tags: range t, subtle, essence 2

Mind Probe: when you touch the target, you get to know one thing as per Face’s Razor Insight move. Tags: range t, essence 2

Detect Life: when you look for living creatures in an area, take +2. Costs 2 essence. Tags: range c/s/m, essence 2


\subsection{Health}
Antidote: when you touch the target, you halt poison or other toxin effects in the target. Tags: range t, essence 2

Heal: when you touch the target, heal a number of wounds equal to 1 + the amount of essence spent on the spell. Tags:range t, exhausting, essence X

Increase Attribute: when you touch the target, choose 1 stat. Moves using that stat take +1 while the spell is sustained. Tags: range t, exhausting, essence 2


\subsection{Illusion}
Chaotic World: when you cast this spell, you can hold 1 to spend on your or your teammate’s moves. Tags: range c/s, 2 essence

Group Invisibility: while you sustain this spell, you conceal a number of creatures equal to the essence spent from being seen by creatures or metahumans. Costs 1 essence per target. Tags: range c, area, essence 1 per individual concealed

Silence: while you sustain this spell, all sound is silenced in the area you specify. Essence cost varies by the size of the area. Tags: range c, area, essence cost varies by size of area

Stink: while you sustain this spell, all creatures in the affected area have to either leave the area or use air filters or take 1 stun. Tags: range s/m, area, essence 2


\subsection{Manipulation}
Mana Barrier: while you sustain this spell, you create a barrier that blocks living creatures and spirits. Tags: range c, essence 2

Light: while you sustain this spell, an area you specify is illuminated by bright light. Tags: range s, area, essence 2

Shadow: while you sustain this spell, an area you specify is cloaked in arcane darkness. Tags: range s, essence 2

Fling: when you cast this spell on a target you are touching, you hurl the target out of melee range. Tags: range t/c, essence 1



\section{Spirits}

Spirits are the companions and tools of the Shaman, who summons them from the astral plane to perform services for him. Spirits have the following special tags: aspect: the spirit takes on the appearance of their domain, and is invisible in their domain unless it chooses to be seen. Elementals automatically gain this tag, otherwise it requires 1 spirit point.

desert: a spirit of the forbidding landscape of the deserts
earth: a spirit who dwells in the earth, caves, or landscape; earth spirits are widespread
elemental: these spirits represent the basic four elements, air, earth, fire, and water, and can be summoned anywhere.
engulf: the spirit may enclose a target in the ubstance of its domain, typically (but not always) dealing damage.
enthrall: use this stat for the Enthrall move
forest: a spirit of the forests, woods, or similar areas
generous: the spirit will perform one extra move; adding this tag costs 1 spirit point.
guard: use this stat for the Guard move
harm: use this stat for the Harm move
insubstantial: damage dealt and taken is halved
mentor: use this stat for the Mentor move
mountain: a spirit that dwell in foothills, crags, ridges, and other mountainous terrain
natural: natural spirits are spirits associated with particular domains (such as “city spirits” or “mountain spirits”).
plains: a spirit of the open plains, grasslands, fields, and farms
robust: the spirit is particularly resistant to damage; all damage rolls against it are [w]. Adding this tag costs 1 spirit point.
search: use this stat for the Search move
sky: a spirit of the open sky
storm: a spirit of storms and harsh weather
swamps: a spirits of the depths of the swamp, bayou, or wetlands
urban: a spirit dwelling in urban or developed lands, especially cities
water: a spirit of lake, river, or ocean
weakness (specify): the spirit has a weakness to a particular material or element which ignores insubstantiality, armor, and robustness. Adding this tag allows the free addition of another tag.
wild: this spirit has an extra spirit point, but the shaman must take -1 when he or she conjures it


\subsection{SPIRIT MOVES}

Spirits are independent entities, and have thier own moves. Their moves correspond to the harm, search, guard, enthrall, and mentor tags.

Harm: When a spirit attacks someone or something, roll+Harm. On 10+, the spirit deals its damage. On 7-9, the spirit deals damage, but also takes damage.

Search: When the spirit attempts to locate individuals or items within its domain, roll+Search. On 10+, the spirit locates the item and can tell the Shaman where it is. On 7-9, the spirit can tell the shaman whether the item or person is within its domain, but not it’s specific location. Note: the GM and player should determine the search range for elementals.

Guard: When a spirit stands in defense of its domain or inhabitants thereof, roll+Guard. On 10+, the spirit prevents damage or hostile effects from occurring. On 7-9, the spirit halves damage or the potency of a hostile effect.

Enthrall: When a spirit attempts to control someone’s actions or thoughts, roll+Enthrall. If the target is:

    An NPC: On a 10+, the spirit issues two instructions that the NPC must follow, or take 3 damage. On 7-9, the spirit may issue one instruction.

    A PC: On a 10+, both of the following apply. On 7-9, only 1 applies:

    If the character complies, they mark XP

    If the characer refuses, they must Stay Frosty

Mentor: When a spirit imparts knowledge or truth, roll+Mentor. On 10+, the GM provides, in secrete, a useful or interesting piece of information to the target. On 7-9, the GM provides an interesting piece of information.


\subsection{Spirit Examples}
There are 5 general spirit natures: Watchers simply observe and report. Teachers seek to instruct and guide others, but are reluctant to do harm. Protectors seek to defend their domain and its inhabitants, while Destroyers seek battle, blood, and vengeance. Finally, Seducer spirits desire control and devotion.

\subsubsection{Elementals}
Fire Elemental (destroyer, aspect, harm 2, search -1, guard 1, enthrall 1, mentor 0, dmg 1d10, armor 2, wounds 9)

Water Elemental (seducer, aspect, harm -1, search 2, guard 0, enthrall 3, mentor 1, dmg 1d4, armor 1, wounds 8)

Air Elemental (teacher, aspect, harm -2, search 2, guard 0, enthrall 1, mentor 2, dmg 1d4, armor 2, wounds 7)

Earth Elemental (protector, aspect, harm 1, search 2, guard 2, enthrall -1, mentor 0, dmg 1d8, armor 1, wounds 10)

\subsubsection{Nature Spirits}
Forest Protector (natural, forest, harm 1, search 1, guard 2, enthrall -1, mentor 0, dmg 1d8, aspect, armor 1, wounds 8)

Forest Watcher (natural, forest, search 3, guard 0, enthrall 1, mentor 1, aspect, armor 1, wounds 6, special: may not Harm)

Sky Watcher (natural, aspect, search 3, guard 0, enthrall 0, mentor 2, armor 1, wounds 6, special:may not Harm)

Urban Destroyer (natural, harm 2, search 0, guard 1, enthrall 1, mentor -1, dmg 1d10, armor 2, wounds 9)

Urban Seducer (natural, seducer, harm 0, search 2, guard 0, enthrall 2, mentor 1, dmg 1d4, armor 1, wounds 7)

Mountain Teacher (natural, aspect, harm 0, search 0, guard 2, enthrall 0, mentor 2, dmg 1d4, armor 1, wounds 8)

Swamp Destroyer (natural, aspect, harm 2, search 2, guard 0, enthrall 0, mentor -1, dmg 1d10, armor 2, wounds 9)

\subsection{Creating New Spirits}
The spirits above are just examples; the procedures that follow describe how to “build” a new spirit to suit your preferences.

Choose the spirit’s Type: elemental or nature.

Choose the spirit’s Domain, and record the base Armor and Wounds.

Choose the spirit’s Nature, and modify the basic spirit tags as needed.

Distribute 4 spirit points among spirit’s Moves, adjusting for the spirit’s purpose. No spirit move may have a modifier higher than +3.

Add additional tags if desired (see Other Spirit Tags).

\textbox{Example}{Pam is playing a Shaman named Chert and is developing the initial three spirits Chert can summon. Pam decides the first one will be a nature spirit of the forest, a protector of the dwindling unspoiled lands. With those decisions made, the spirit’s qualities so far are nature, forest, protector, armor 1, wounds 10, dmg 1d8, guard 1, enthrall -1. \\
Pam also wants the spirit to blend in with the forest and to be an excellent guardian of its inhabitants. She spends one spirit point (out of 4) to gain the aspect tag, and then spends the remaining three to boost the Guard move twice, and the Harm move once. The final spirit looks like this: nature, forest, protector, harm 2, guard 3, search 0, enthrall -1, mentor 0, armor 1, wounds 10.}


\subsection{Spirit Types}
Elemental: these spirits represent the basic four elements, air, earth, fire, and water, and can be summoned anywhere.

Nature: natural spirits are spirits associated with particular domains (such as “city spirits” or “mountain spirits”). Natural spirits may enter other domains freely, but they can only be summoned within their own, and if they cross domains, there’s always a chance they attract unwanted attention from other spirits who don’t like intruders.
BASIC SPIRIT TAGS

Domain represents the spirit’s preferred environment or the area in which it may be summoned. A natural spirit summoned in its domain always has the generous tag. The domain of an elemental is considered to be the same as its element (though they gain no benefit from being within their domain).

Urban: spirits that dwell in urban or developed lands, especially cities

Plains: spirits that dwell in open plains, grasslands, open fields, and farms

Forest: spirits that dwell in forests, woods, and similar areas

Mountain: spirits that dwell in foothills, crags, ridges, and other mountainous terrain

Earth: spirits that dwell underground or in caves; the domains of earth spirits are widespread.

Deserts: spirits that dwell in the sere, forbidding landscape of the deserts

Sky: spirits dwelling in the open skies.

Storm: spirits of storm and disruption

Swamps: spirits who dwell where earth and water are one

Water: spirits of the water, be it lakes, rivers, or the open sea

There are two things to be aware of regarding domains. First, domains are relatively confined—a mountain spirit’s domain is not all mountains, nor even all of a specific mountain. Rather, it is usually a region with a radius of around 250 meters, within a mountainous region. Overlap among domains is possible, and the byzantine negotiations that take place between spirits defy understanding even by the most gifted shamans.

Also remember that multiple domains may exist within a larger area that seems uniform. In other words, city spirits (for example) are the only kind of spirit you’ll run across in a city—a park within a city may be the home of a forest spirit, and you may find a river spirit fighting to protect it’s home from polluted runoff in some industrial area.

Armor represents the spirit’s innate magical resistance to damage; spirit armor cannot be ignored, nor reduced by weapons. All spirits have 1 armor.

Wounds simply represent the spirit’s innate health; all spirits, by default, have 8 wounds.


\subsection{Spirit Nature}
Every spirit has a nature, which indicates its sense of purpose and the activities to which it is drawn. A spirit’s nature also affects its basic tags and moves (see Spirit Moves, below) in various ways.

Watcher spirits observe, find, and note. They are incapable of dealing harm to anyone or anything. Watcher spirits have the following modifiers: Search +2, Wounds -2, may not Harm.

Teacher spirits wish to inform and instruct, and find it difficult to inflict damage upon those they could otherwise teach. Teacher spirits have the following modifiers: Mentor +2, Harm -2, dmg 1d4.

Protector spirits preserve, defend, and support their domain. They are unconcerned with influencing intruders, preferring to throw them out instead. Protector spirits have the following modifiers: Guard +1, Enthrall -1, Wounds +2, dmg 1d8.

Destroyer spirits are warrior spirits who revel in combat and bloodletting. They are fearsome enemies, though somewhat limited in imagination. Destroyer spirits have the following modifiers: Harm +2, Mentor -1, Search -2, Wounds +1, Armor +1, dmg 1d10.

Seducer spirits wish to influence, to inspire love, and to acquire servants, though they do not typically enjoy directly harming others. Seducer spirits have the following modifiers: Enthrall +2, Harm -1, Wounds -1, dmg 1d4.


\subsection{Other Spirit Tags}
Robust: the spirit is particularly resistant to damage; all damage rolls against it are [w]. Adding this tag costs 1 spirit point.

Aspect: the spirit takes on the appearance of their domain, and is invisible in their domain unless it chooses to be seen. All spirits have this tag.

Generous: the spirit will perform one extra move; adding this tag costs 1 spirit point.

Insubstantial: damage dealt and taken is halved Weakness (specify): the spirit has a weakness to a particular material or element which ignores insubstantiality, armor, and robustness. Adding this tag allows the free addition of another tag.

Engulf: the spirit may enclose a target in the substance of its domain, typically (but not always) dealing damage.

Wild: this spirit has an extra spirit point, but the shaman must take -2 whenever he or she conjures it.



\section{Totems}

Shaman characters must select a totem, representing their connection to one of the great spirits.
\paragraph{BEAR}
    \begin{easylist}
        # \textbf{Boon:} reduce essence cost to conjure protector spirits by 1 (to a minimum of 1)
        # \textbf{Flaw:} when injured, roll 1d6. On 1 or 2, the shaman goes berserk).
    \end{easylist}

\paragraph{CAT}
    \begin{easylist}
        # \textbf{Boon:} gain low-light vision; you cannot be surprised
        # \textbf{Flaw:} you cannot deal lethal damage to your enemy
    \end{easylist}

\paragraph{COYOTE}
    \begin{easylist}
        # \textbf{Boon:} take +1 to conjure Teacher spirits
        # \textbf{Flaw:} destroyer spirits summoned lose 1 spirit point
    \end{easylist}

\paragraph{DOG}
    \begin{easylist}
        # \textbf{Boon:} and take +1 to conjure protector spirits or city spirits
        # \textbf{Flaw:} your moves are glitched if you have left an ally behind or in danger
    \end{easylist}

\paragraph{GATOR}
    \begin{easylist}
        # \textbf{Boon:} take +1 to conjure water spirits.
        # \textbf{Flaw:} You are exceptionally greedy
    \end{easylist}

\paragraph{EAGLE}
    \begin{easylist}
        # \textbf{Boon:} take +1 to conjure watcher spirits or air elementals
        # \textbf{Flaw:} you have an allergy to something relatively common, and take -1 ongoing when exposed
    \end{easylist}

\paragraph{LION}
    \begin{easylist}
        # \textbf{Boon:} take +1 to conjure protector or plains spirits
        # \textbf{Flaw:} Take -1 on Gut Checks
    \end{easylist}

\paragraph{OWL}
    \begin{easylist}
        # \textbf{Boon:} gain low-light vision, take +1 to conjure teacher spirits
        # \textbf{Flaw:} Spells cost 1 more essence to cast in the day
    \end{easylist}

\paragraph{RACCOON}
    \begin{easylist}
        # \textbf{Boon:} and take +1 to conjure watcher spirits
        # \textbf{Flaw:} must Stay Frosty to avoid letting his curiosity get to him
    \end{easylist}

\paragraph{RAT}
    \begin{easylist}
        # \textbf{Boon:} take +1 to conjure city spirits
        # \textbf{Flaw:} when combat starts, you must Stay Frosty, or flee
    \end{easylist}

\paragraph{RAVEN}
    \begin{easylist}
        # \textbf{Boon:} take +1 to conjure watcher spirits
        # \textbf{Flaw:} you must take advantage of others’ misfortune when you can
    \end{easylist}

\paragraph{SHARK}
    \begin{easylist}
        # \textbf{Boon:} take +1 to conjure destroyer spirits
        # \textbf{Flaw:} when injured, roll 1d6: on <4, the shaman goes berserk
    \end{easylist}

\paragraph{SNAKE}
    \begin{easylist}
        # \textbf{Boon:} and take +1 to conjure seducer spirits
        # \textbf{Flaw:} take -1 ongoing to Rock \& Roll
    \end{easylist}

\paragraph{WOLF}
    \begin{easylist}
        # \textbf{Boon:} take +1 to conjure protector spirits
        # \textbf{Flaw:} you must Stay Frosty to retreat from combat
    \end{easylist}

    \chapter{Matrix} \label{matrix}
 
All die lieben Leute, die den Runnern weiter- und aus der Patsche helfen.
 
\section{Frostburn}

    \subsubsection{Brimstone}
    \paragraph{Elf, weiblich, Magierin -} Ehemaliger Mitrunner, Ex-Kollege von NeoNet. Oder so.
    
    \subsubsection{Harrison Kellerman}
    % siehe Pumat Sol, Doppelgänger und so, maybe?
    \paragraph{Normie, männlich, Magier -} Harrison ist der Inhaber des \textit{Io Pan!}; er hat sich als Taliskrämer im nördlichen Downtown selbstständig gemacht, nachdem er seinen Master of Magical Theory and Arts an der \textit{University of Seattle} abgeschlossen hat. Auch wenn er selbst ein eher geringes Interesse an Schattengeschäften und ein geradezu unschuldiges Gemüt hat, bringt es sein Job unweigerlich mit sich, dass er um die arkanen Vorgänge der Gegend üblicherweise äußerst gut informiert ist.
    
    \subsubsection{Jules 'Jules' Maguire}
    \paragraph{Ork, männlich, mundan -} Frostburns Bruder. Ein absoluter Gearhead; rauchgrauer und deutlich erkennbarer, voll-vercyberter Arm, ständig mit dem Kopf in irgendeiner Motorhaube oder via Riggerinterface im Fahrzeug selbst - Hauptsache es hat Räder oder Flügel. Er führt eine Werkstatt für die \textbf{Cascade Ork}, einem orkischen Indianerstamm im \textit{Salish-Shidhe Council}, der sich hauptsächlich durch Schmuggelgeschäfte entlang der Seattle-SSC-Grenze finanziert, an denen Jules in seiner Position natürlich maßgeblich beteiligt ist.
    
 
\section{Neuron}

    \subsubsection{Netcat}
    \paragraph{Elf, weiblich, Technomancerin -}%Ja, DIE Netcat
    Netcat wurde schon öfter als ``das schweizer Taschenmesser der Matrix'' bezeichnet - ein Titel, den sie zumindest nicht aktiv versucht, los zu werden. Ihre große Stärke ist Improvisation. Sie hat einen 8-jährigen Sohn, Jack, mit dem berufsjugendlichen Matrixvandalen (und einem der drei Jackpoint-Admins) \textbf{Slamm-O!} und ist selbst aktive Jackpoint-Userin. Sie verfasste u.a. Beiträge zum \textit{Digitalen Erwachen}, über \textit{Technocritter} und lud im (mittlerweile schon etwas veralteten) \textit{Corporation Guide} eine ausführliche Datei über \textbf{Mitsuhama} hoch.
    
    \subsubsection{Brent ‘Lone Wolf’ Conley}
    \paragraph{Normie, männlich, mundan -} Niemand kennt die Straßen von Downtown so gut wie Brent. Seit \textit{der Star} (d.h., Lone Star Security Services) 2072 den Kontrakt in Seattle an Knight Errant verloren hat, schlägt Brent sich als 'Lone Wolf' durch die Schatten; Ganger, Schieber, Peasants(?) (read: Knight Errant Coppers) - wer hat grad wo das Sagen und wem muss man einen Gefallen tun, um an ein Gespräch zu kommen? Brent weiß es.
    
    \subsubsection{Metatron}
    \paragraph{unbekannt -} Entwickler bei \textbf{Snowstorm}, einem großen Matrixspielestudio und Subsubsidiary von Horizon.

 
\section{Rude}
 
    \subsubsection{Momma Dot}
    \paragraph{Zwergin, weiblich, mundan -} Inhaberin von BBC Weapon Works - eines ‘Mom \& Pop’-style Waffenladens im Ares-Franchise. 
    
    \subsubsection{Jimmy Vitello}
    % Ex-Mafiaschläger der Finnigan Familie und Manager der Muckibude Max’s Ironworks
    \paragraph{Normie, männlich, mundan -} Jimmy ist das lebende Klischee eines Mafia-Schlägers-gone-Geschäftsmann: ein 1.70 Meter Muskelberg, schlecht sitzender, aber offensichtlich teurer Anzug, oben offenes Hemd, aus dem sich mächtige Brustbehaarung kräuselt, goldener Schneidezahn und goldenes Kettchen am kaum als solchen erkennbaren Hals.
    Jimmy hat sich als einfacher Fußsoldat der Familie auf der Straße verdient gemacht und leitet mittlerweile die Muckibude \textit{Max's Ironworks} für die Finnigans (die leitende Familie des O'Malley Syndikats - also der Cosa Nostra Seattles).
    % Lass Jimmy dir helfen, du weißt schon - badabing-badaboom -
    
    \subsubsection{Hez}
    \paragraph{Troll, männlich, mundan -} Leutnant der Skraacha-Gang, regelt die Nachbarschaftswache

 
\section{Tusk}

    \subsubsection{April Summers}
    \paragraph{Elf, weiblich, mundan -} Investigativjournalistin im Politikressort des Seattle Tribunes
    
    \subsubsection{Großmeister Arvid}
    \paragraph{Zwerg, männlich, Adept -} Großmeister eines Aikido-Dojos in Bellevue
    
    \subsubsection{Benjamin ‘Meatgrinder’ Flowers}
    \paragraph{Troll, männlich, mundan -} Profi-Brawler und F-Promi im Entertainmentsektor von Seattle


\section{Yu}

    \subsubsection{Billy Shen}
    % siehe https://shadowhelix.de/Octagon
    % Sie operieren vornehmlich im Stadtteil Little Asia in Tacoma, und haben dort unter den Einheimischen zu Beginn der 2070er effektiv eine militante Anti-Yakuza-Stimmung erzeugt.
    \paragraph{Elf, männlich, mundan -} ein Leutnant der Oktagon-Triade; Vollstrecker und Waffenhändler
    
    \subsubsection{Four-Finger-Wong}
    \paragraph{Normie, männlich, mundan -} ein Cyber-Chirurg und Straßendoc
    
    \subsubsection{Mia}
    \paragraph{Normie, weiblich, mundan -} Ehemalige Go-Gangerin, aber mittlerweile nur noch Teilzeit-Schieberin und Kellnerin im Peaceable Kingdom, wo sie einen Raum für ungestörte Treffen u.ä. anbieten kann. Hat aufgrund ihres Jobs und ihrer Vergangenheit Kontakte ins Drogenmilieu von Downtown und hat auch sonst einen guten Überblick darüber, mit wem sich worüber sprechen lässt.
    
    \subsubsection{Nahalie Brook}
    \paragraph{Normie, weiblich, mundan -} Eine talentierte Deckerin, die bei ReuterCore - einer Tochterfirma von S-K-Haas-Grotesk - für die Matrixsicherheit zuständig ist. Als solche hat sie wertvolle Kentnisse über entsprechende technische Schutzmaßnahmen und als eine Person, die konzernnäher nicht sein könnte, auch Einblicke in entsprechende Strukturen, Abläufe und Namen.
    
    
\section{Sonstige}

    \subsubsection{Yashnarz}
    % Schieber für die Los Ángeles Ardientes (Burning Angels)
    % https://shadowhelix.de/Los_%C3%81ngeles_Ardientes
    \paragraph{Ork, männlich, mundan -} Yashnarz' Tatoos zeigen deutlich seine Zugehörigkeit zu einer Straßengang, für die er sich offenbar als Waffenschieber zwischen dem \textit{Freistaat Los Angeles} und Seattle betätigt.
    \chapter{Gear} \label{Gear}

\epigraph{\textit{We're in the minority; Runners who are not jacked, rigged or wakened. We live by our guts and wits.}}{-- Jazzman Harker, Shadowrunner}

In this section you’ll find example equipment (weapons, cyberdecks, vehicles, etc.) available in the Sixth World. This isn’t an exhaustive list of what’s available; rather, they’re just samples of some classic items to help you get playing quickly.


\section{Tags} \label{tags}

Equipment — like many items in Sixth World — is described in terms of tags, which are short keywords that indicate various capabilities or qualities. Certain tags apply to multiple kinds of equipment (such as obvious, supply, or armor). Tags that only apply to specific kinds of equipment are described in the listing of that kind of item. The following tags apply to multiple types of equipment.

\fullbox{Tags}{
    \textbf{2-hand} & this item must be used with both hands \\
    \textbf{armor +n} & grants a +n bonus to existing armor \\
    \textbf{armor n} & grants n Armor (armor rating for vehicles or drones) \\
    \textbf{arcane} & can only be used by awakened characters \\
    \textbf{area / aoe} & affects multiple targets \\
    % \textbf{avail} & the availability of the item on the shadow markets \\    
    \textbf{+bonus} & grants a bonus to a particular move; e.g. +1 to \move{Null Sweat} \\
    \textbf{conceal} & this weapon or item is easily hidden and will not be spotted by enemies \\
    % \textbf{damage n} & the amount of damage a weapon or other item deals; abbreviated dmg \\
    % \textbf{heal n} & restores n wounds \\
    % \textbf{ignores armor} & bypasses the target’s armor \\
    \textbf{AP} & stands for Armor Penetration - bypasses/ignores the target’s armor \\
    \textbf{loud} & noisy and audible to anyone with functioning hearing; for weapons, it means the weapon cannot be suppressed \\
    \textbf{messy} & deals damage in a particularly gruesome way \\
    \textbf{obvious} & cannot be concealed, or is immediately visible to any observer \\
    % \textbf{range} & the range(s) at which the weapon or other attack is effective. Ranges are touch (t), close (c), short (s), medium (m), and long (l) \\
    % \textbf{shock} & the weapon deals electrical shock \\
    % \textbf{special (description)} & if the effect of the item requires explanation, use this tag \\
    \textbf{stun} & this weapon or attack deals non-lethal Stun damage \\
    \textbf{subtle} & not easily noticed (as opposed to conceal, which means it is unnoticeable) \\
    \textbf{Supply n} & the amount of supplies or uses you can get out of an item. Each use of the item consumes 1 supply (unless otherwise stated)
}



\section{Weapons}

The tags below apply to weapons. Feel free to customize the example equipment with these tags (subject to GM approval) to create your own gear, or recreate classic gear from Shadowrun.

\fullbox{Weapon Tags}{
    \textbf{FA} & this weapon can fire in Full Automatic mode \\
    \textbf{BF} & this weapon has a Burst Fire mode: mark off 1 additional Ammo to deal +1 damage \\
    \textbf{chem} & this weapon delivers a chemical agent of some kind to the target; depending on the delivery mechanism, armor may be ignored \\
    \textbf{forceful} & when this weapon deals damage, it also deals 1 stun \\
    % \textbf{fuzed} & this weapon cannot be used at less than the shortest range increment listed \\
    \textbf{reload} & after using this weapon, it takes more than a moment to reload it \\
    \textbf{SA} & this weapon fires Semi Automatically - one shot every time the trigger is pulled \\
    \textbf{stabilized} & this weapon cannot be fired except from a bipod, tripod, or supported position \\
    \textbf{silenced} & this weapon makes little to no noise when fired \\
    \textbf{thrown} & this item can be thrown \\
    \textbf{vented} & the weapon has recoil venting, granting +1 to Suppression Fire
}


\subsection{Weapon Conversions}

[\textit{Don't spend too much attention to this section for now, it's probably gonna change quite a lot anyway.}]

The (online-only) Simple Edition mentioned in the preamble provides quite a few items which will also be listed in this chapter. The Digest Edition 1.3 on the other hand provides some guidelines to convert items from Shadowrun 5 sourcebooks. Additionally, I aim to provide further guidelines to convert Shadowrun 6 gear.

\subsubsection{From Shadowrun 5}
Rather than reproduce a listing of shadowrun weapons here, or provide an overly generic "heavy pistols do this, and assault rifles do this", the following guidelines should help you convert weaponry from Shadowrun 4th or 5th edition core rulebooks. Keep in mind these are guidelines and not hard and fast rules; feel free to adjust weapons by hand to get them "just so".

\begin{easylist}
     # \textbf{Weapon Type:} Self-explanatory
     # \textbf{Weapon Range:} generally, melee weapons are range c, pistols and SMGs either range s or range s/m, and rifles and other longarms are tagged range s/m/l. Exceptions to this include  sniper rifles, which are optimal at long range (range l) only, and heavy weapons which generally are best tagged range m/l.
     # \textbf{Damage Type:} stun weapons should get the stun tag
     # \textbf{Damage Value:} the damage value of a weapon in Sixth World should be roughly one-half the damage value of the weapon as listed in the Shadowrun core books. Damage can either be fixed value or dice-based. For weapons that incorporate Strength into their damage ratings, you can add the character’s Hard rating (for instance, 2+Hard dmg).
     # \textbf{Armor Piercing:} divide the AP value in the Shadowrun core books by 2 to get the Sixth World equivalent AP value.
     # \textbf{Ammo:} divide the weapon capacity listed in the Shadowrun core books by 5 to arrive at the ammo value for the Sixth World equivalent weapon (note that some weapons may require some adjustment by hand on this point, and single-shot weapons should have ammo 1).
     # \textbf{Other Tags:} assign other tags as appropriate (such as firing modes, whether they require two hands, and so forth) to round out or customize the weapon.
     # \textbf{Cost:} dividing the cost by between 2 and 4 will generate an appropriate price for Sixth World use.
\end{easylist}

\textbox{Example}{The Ares Predator V does 8P damage according to the Shadowrun Fifth Edition core book. In Sixth World, the damage would be either 4, or a dice value approximating that (e.g., 1d8).}

\subsubsection{From Shadowrun 6}
\textit{yet about to come}


\subsection{Melee Weapons}

\wpnbox{Melee Weapons}{
    Staff        & stun  & 1d6+2 & --- & 100¥ \\
    Combat Axe   & messy & 1d6+2 & --- & 1.250¥ \\
    Combat Knife & ---   & 1d6   & --- & 300¥ \\
    Fists/Feet   & stun  & 1d4   & --- & --- \\
    Katana       & ---   & 2d6b  & --- & 1.000¥ \\
    Spiked Glove & ---   & 1d4+1 & --- & 50¥ \\
    Stun Baton   & stun, shock, ignores armor & 1d4 & --- & 750¥ \\
    Tomahawk     & messy, thrown & 1d6 & --- & 200¥ 
}


\subsection{Hold-Out Pistols}

A pocket pistol is any small, pocket-sized semi-automatic pistol (or less commonly referencing either derringers, or small revolvers), and is suitable for concealed carry in either a coat, jacket, or trouser pocket.

\wpnbox{Hold-Outs}{
    Streetline Special & SA, conceal    & 2d4b  & 3 & 250¥ \\
    Fichetti Needler   & conceal        & 2d4b  & 3 & 400¥ \\
    Walther PP         & SA/BF, conceal & 1d4+1 & 1 & 325¥
}


\subsection{Light Pistols}

\wpnbox{Light Pistols}{
    Colt L36 & SA, conceal & 1d6 & 3 & 500¥ \\
    Beretta 101T & SA/BF, subtle & 1d6 & 2 & 450¥ \\
    Ares Lightfire 70 & SA, conceal & 1d6 & 3 & 350¥
}


\subsection{Heavy Pistols}

\wpnbox{Heavy Pistols}{
    Ares Predator       & SA & 1d8+1  & 3 & 675¥ \\
    Colt Manhunter      & SA/BF & 1d8 & 3 & 560¥ \\
    Ruger Super Warhawk & SA, loud & 1d10 & 2 & 560¥ \\
    Browning Max Power  & SA & 2d8b & 3 & 675¥
}


\subsection{Submachine Guns}

\wpnbox{Submachine Guns}{
    HK227  & SA/BF, suppressed & 1d8 & 4 & 900¥ \\
    AK-97K & SA/FA & 1d8 & 3 & 1.000¥ \\
    Ingram Smartgun & BF/FA & 1d6+1 & 3 & 950¥
}


\subsection{Assault Rifles}

\wpnbox{Assault Rifles}{
    AK-97      & SA/FA, 2-hand, obvious & 1d10 & 3 & 800¥ \\
    Ares Alpha & SA/BF/FA, 2-hand, obvious & 2d8b & 4 & 1.150¥ \\
    Colt M22A2 & SA/BF, 2-hand, obvious & 1d10 & 3 & 850¥ \\
    FN-HAR     & SA/BF, 2-hand, obvious, loud & 2d8b & 3 & 1.050¥
}


\subsection{Shotguns}

\wpnbox{Shotguns}{
    Remington 990 & SA, obvious, loud, forceful & 1d10+1 & 2 & 750¥ \\
    Enfield AS7   & SA/BF, 2-hand, obvious, loud, forceful & 1d10 & 3 & 900¥
}

\subsection{Sniper Rifles}

\wpnbox{Sniper Rifles}{
    Ranger Arms    & SA, 2-hand & 1d10+1 & 3 & 1.150¥ \\
    Walther WA2100 & SA, 2-hand & 1d12   & 4 & 1.100¥
}


\subsection{Heavy Weapons}

\wpnbox{Heavy Weapons}{
    Ingram Valiant LMG & FA, 2-hand, loud, stabilize, obvious, loud, messy & 1d12 & 4 & 2.000¥ \\
    Stoner M202 HMG & BF/FA, 2-hand, loud, stabilize, obvious, loud, messy & 2d10b & 3 & 2.500¥
}


\subsection{Special Weapons}

\wpnbox{Speacial Weapons}{
    Compound Bow & 2-hand & 1d6+1 & 1 & 500¥ \\
    Narcoject Rifle & stun, suppressed, chem, slow & 1d8+1 & 1 & 700¥ \\
    Taser & stun, shock, slow & 1d8 & --- & 500¥ \\
    Crossbow & 2-hand, suppressed & 1d6 & 1 & 400¥
}


\subsection{Grenades}

\begin{strip}
    \textbox{Grenades}{
        \rowcolors{1}{gray!30!white}{}
        \begin{tabular}{m{0.15\textwidth} m{0.5\textwidth} m{0.1\textwidth} m{0.15\textwidth}}
            \textbf{\textsf{\textcolor{purplefont}{TYPE}}} & 
            \textbf{\textsf{\textcolor{purplefont}{TAGS}}} &
            \textbf{\textsf{\textcolor{purplefont}{DMG}}}  &
            \textbf{\textsf{\textcolor{purplefont}{PRICE}}}  \\
            EMP   & thrown, area, shock; disables electronics & --- & 95¥ \\
            Flash & thrown, area, stun, +1 to \move{Rock \& Roll}/\move{Null Sweat} & --- & 125¥ \\
            Frag  & thrown, area, forceful & 2d6b & 100¥ \\
            Incendiary & thrown, area, burn & 2d6b & 75¥ \\
            Smoke & thrown, area, +1 to \move{Null Sweat} & --- & 40¥ \\
            Stun  & thrown, area, stun & 2d6b & 100¥
        \end{tabular}
    }
\end{strip}


\subsection{Armor}

Armor provides protection against incoming attack, reducing the damage dealt by the armor value. Armor of the same type (e.g inherent) does not stack. Armor of differing types can stack. Armor has the following unique tags:
\textbox{Armor Types}{
    \rowcolors{1}{gray!30!white}{}
    \begin{tabular}{m{0.2\textwidth} m{0.7\textwidth}}
        \textbf{inherent} & this armor is either implanted, or occurs naturally. Cyberware armor is inherent armor. \\
        \textbf{worn} & this armor is worn on the body \\
        \textbf{mystic} & this armor is magical in nature
    \end{tabular}
}

\subsubsection{Sample Armor}
\armbox{Armor Examples}{
    Lined Coat     & 2 & obvious & worn & 600¥ \\
    Ballistic Vest & 2 & obvious & worn & 750¥ \\
    Armorweave Professional Wear & 1 & subtle & worn & 1.500¥ \\
    Leather Armor  & 1 & subtle  & worn & 250¥ \\
    Armor Charm    & +1 & conceal & mystic & 400¥ \\
    Light Armor Jacket & 1 & subtle & worn & 850¥ \\
    Combat Armor   & 3 & obvious & worn & 2.500¥  \\
    Form-fitting Armor & 1 & conceal & worn & 550¥ \\
    Riot Shield    & 2 & occupies one hand, obvious & worn & 700¥
}


\section{Cyberware}

Cyberware works slightly differently from other equipment. Instead of simply being something that has some tags or stats, each piece of cyberware provides new moves or modifies existing moves, based on the augmentation’s function.

Adding cyberware costs essence, which does have a significant effect on magic users, so magic archetypes who choose cyberware do so at the cost of their magical potency.

When you have downtime, you may elect to have cyberware installed. Installation and recovery from cybersurgery takes a number of days equal to 3x Essence cost of the implant.

\subsection{Containers}
The listed essence costs assume that the augmentation is directly implanted into the body. However, full-replacement cyberware (such as cybereyes, cyberears, and cyberarms) have the capacity to hold other implants without costing additional essence. Each of these items can hold additional augmentations equal to 1 + their Essence cost (for example, cybereyes cost 1 essence, and therefore can contain up to 2 essence worth of additional augmentations).


\subsection{Headware}
% FULL REPLACEMENTS

\subsubsection{Vision Enhancements}
\paragraph{Cybereyes -} Capacity for 2 essence worth of vision enhancement augmentations. Include low-light vision system. Cost: 1 essence, 4.000¥.

\begin{easylist}
    # \textbf{Low-light Vision:} you can see in the dark, as long as there’s at least some light, not complete darkness. Included in Cybereyes for no extra money or essence. Cost: 1 essence, 1,000¥.    
    # \textbf{Thermographic Vision:} when you Check the Situation, you may ask one additional question from the list. Cost: 1 essence, 1,500¥.    
    # \textbf{Recorder:} when you use this device, you gain 1 hold to spend on researching the individual, location, or event you recorded. Cost: 1 essence, 1,000¥.    
    # \textbf{Flare Compensator:} you do not suffer the effects of bright light (such as flash-bang grenades). Cost: 1 essence, 1,000¥.    
    # \textbf{Smartlink:} when you \move{Rock \& Roll}, you never graze the target. Additionally, on 10+, ignore 1 armor. Cost: 1 essence, 2,000¥.
\end{easylist}
    

\subsubsection{Auditory Enhancement}
\paragraph{Cyberears -} Capacity for 2 essence worth of hearing enhancement or auditory augmentations. Cost: 1 essence, 4.000¥.

\begin{easylist}
    # \textbf{Hearing Enhancement:} when you Check the Situation, you may ask one additional question. Cost: 1 essence, 4,000¥.
    # \textbf{Sound Damper:} you do not suffer the effects of loud noises. Cost: 1 essence, 2,000¥.
    # \textbf{Ultrasound system:} you can "see" in total darkness, or even while blind. Ultrasound is detectable if someone is listening for it, however. You can also determine the size of an enclosed space automatically. Cost: 1 essence, 10,000¥.    
    # \textbf{Balance Augmentation:} when performing an acrobatic or tricky maneuver, you are boosted. Cost: 1 essence, 6,000¥.
\end{easylist}


\subsubsection{Other}
\paragraph{Headware device -} you have a device built in to your head. Cost: 1 essence, device cost + 2,000¥.

\paragraph{Control Rig -} you can interface with vehicles and drones and control them directly. Control rigs include a datajack. Cost: 2 essence, 40,000¥.

\paragraph{Synaptic Hardening -} you gain +1 armor against Matrix attacks. Cost: 2 essence, 10,000¥.

\paragraph{Voice Modulator -} you can control your voice perfectly, imitating any sound you’ve heard or any voice you’ve heard. Cost: 1 essence, 6,500¥.


\subsection{Bodyware}

\paragraph{Bone Lacing -} when you make an unarmed attack, you deal lethal damage. Additionally, you take +1 to \move{Gut Check}. Cost: 2 essence, 15,000¥.

\paragraph{Cyberarm -} Capacity for 3 essence of additional implants. Deal +1 damage in melee. This replacement has the obvious tag by default. Increase the cost by 5,000¥ to remove the obvious tag. Cost: 2 essence, 15,000¥.

\paragraph{Cybergun -} you have a permanently implanted weapon. Choose a hold-out pistol or light pistol. This weapon gains the conceal and reload tags. Cost: 2 essence, 2,000¥ (hold out) or 3,900¥ (light pistol).

\paragraph{Datajack -} you are able to interface with a multitude of electronic devices. Datajacks can also be installed in any full-replacement item. Cost: 1 essence, 1,000¥.

\paragraph{Dermal Plating -} you gain +1 armor. This armor stacks with other armor, and has the obvious tag. Cost: 2 essence, 3,000¥.

\paragraph{Gyrostabilizer -} take +1 forward to Suppression Fire. Must be installed in a cyberarm. Cost: 3,000¥.

\paragraph{Hand Razors -} you have a permanently implanted weapon equivalent to a Combat Knife. This weapon can be extended or retracted at your discretion, and gains the conceal tag. Cost: 1 essence, 2,500¥.

\paragraph{Skillwires -} \label{skillwires} when you have an appropriate skillsoft, take +1 ongoing to Drop Science. Additionally, you may \roll{Skilled} to \move{Null Sweat} or Check the Situation. Cost: 2 essence, 10,000¥.

\paragraph{Wired Reflexes -} while active, when you fail a roll and would take damage or be attacked, \roll{Sharp}. On 10+, the damage or effect is halved. On 7-9, you take the damage, but boost your next move. Cost: 3 essence, 50,000¥.



\section{Cyberdecks}

Cyberdecks are the essential tool of the Decker. They are the Decker’s connection to the Matrix. Cyberdecks have the following special tags:

\textbox{Cyberdeck Stats}{
    \rowcolors{1}{gray!30!white}{}
    \begin{tabular}{m{0.2\textwidth} m{0.7\textwidth}}
        \textbf{System} & the power and system stability of the deck; this is the equivalent of the deck’s wounds. A deck whose System is reduced to zero is fried, and can’t be used until repaired  \\
        \textbf{Mask} & the stealthiness of a cyberdeck \\
        \textbf{Hardening} & the deck’s resistance to damage; this acts as armor protecting the decker \\
        \textbf{Storage} & the deck’s capacity for loaded programs
    \end{tabular}
}

\subsection{Converting Cyberdecks}

\textit{Some space for rules to come.}


\subsection{Example Decks}

\deckbox{Cyberdecks}{
    Allegiance Alpha & 5 & 1 & 1 & 5 & 25.000¥ \\
    Fuchi Cyber-4    & 6 & 2 & 1 & 6 & 50.000¥ \\
    Fuchi Cyber-7    & 6 & 1 & 2 & 6 & 75.000¥ \\
    Fairlight Excalibur & 8 & 2 & 1 & 8 & 100.000¥
}


\subsection{Programs}

Programs are the tools and weapons of the decker. Programs can modify a deck’s attributes, allow a decker to deal damage, or offer special moves to a decker. Programs must be loaded into deck storage to be running; each program has a size rating indicating how much storage the program occupies.

\subsubsection{Utilities}
\paragraph{Analyze:} when you examine a node, \roll{Skilled}. On 10+, hold 2 toward hacking the node. On 7-9, hold 1. Size 3, 750¥.

\paragraph{Decrypt:} take +1 forward to hacking Datastore nodes. Size 3, 750¥.

\paragraph{Interface:} take +1 forward to hacking Control nodes. Size 3, 750¥.

\paragraph{Interference:} slows hostile program alarm triggers. Size 2, 500¥.

\paragraph{Patch:} when you attempt to restore system stability to your deck, \roll{Skilled}. On 10+, restore 2 System to your deck. On 7-9, restore 1. Size 2, 500¥.

\paragraph{Reflect:} when you take damage in the matrix, \roll{Steady}. On 10+, redirect the damage to a matrix program or node of your choice. On 7-9, redirect half the damage. Size 3, 750¥.

\paragraph{Stealth:} your deck gains +1 Mask while this program is running. Size 2, 500¥.

\subsubsection{Combat}
\paragraph{Armor:} your deck gains +1 Hardening while this program is running. Size 2, 500¥.

\paragraph{Stunner:} deal 1d4 damage in matrix combat. Size 1, 250¥.

\paragraph{Hammer:} deal 1d6 damage in matrix combat. Size 2, 500¥.

\paragraph{Black Hammer:} deal 1d8 damage in matrix combat. Size 3, 750¥.

\paragraph{Static:} when you \move{Rock \& Roll} in the matrix, you may choose to forgo dealing damage, and instead hold 2 to grant to any ally’s roll. You can only spend 1 hold at a time. Size 3, 750¥.


\section{Vehicles}

Vehicles have the following special tags (or stats, rly, innit).

\textbox{Vehicle Stats}{
    \rowcolors{1}{gray!30!white}{}
    \begin{tabular}{m{0.2\textwidth} m{0.7\textwidth}}
        \textbf{Power} & the vehicle’s horsepower, speed, and acceleration \\
        \textbf{Armor} & the vehicle or drone’s armor rating \\
        \textbf{Frame} & the vehicle’s or drone’s resilience. This is the equivalent of a vehicle’s wounds. Remember that small arms deal half damage to vehicles \\
        \textbf{Sensor} & thethe quality of the vehicle’s sensors (used when Checking the Situation while driving or piloting the vehicle) \\
        \textbf{Seats} & the number of people who can normally occupy the vehicle, including the driver or pilot \\
        \textbf{Fuel} & fuel or battery capacity
    \end{tabular}
}


% \subsection{Bikes}

\vroombox{Bikes}{
    Dodge Scoot     & 1 & 1 & 0 & 4 & 0 & 4 & 1.800¥ \\
    Yamaha Rapier   & 1 & 2 & 0 & 4 & 1 & 4 & 9.500¥ \\
    Harley Scorpion & 2 & 2 & 1 & 7 & 1 & 2 & 17.500¥
}

% \subsection{Cars \& Trucks}

\vroombox{Cars \& Trucks}{
    C-N Jackrabbit   & 3 & 1 & 0 & 6 & 0 & 3 & 10.000¥ \\
    Ford Americar    & 4 & 1 & 0 & 8 & 1 & 3 & 16.000¥ \\
    Eurocar Westwind & 6 & 3 & 1 & 9 & 1 & 3 & 200.000¥ \\
    GMC Bulldog      & 8 & 2 & 1 & 9 & 1 & 3 & 45.000¥ \\
    Ares Roadmaster  & 6 & 3 & 2 & 11 & 1 & 2 & 52.000¥
}


\subsection{Drones}

Drones have most of the same qualities as vehicles, although they lack the seats tag, and replace it with the following:

Tactical: the quality of the drone’s tactical expert system, which comes into play when the drone is in autonomous mode. Abbreviated tac.
Armed drones also use the damage tag, indicating the damage of their built-in weapon systems.

% \subsubsection{Ground Drones}
\dronebox{Ground Drones}{
    Aztechnology Crawler & 1 & 5 & 0 & 2 & 0 & ---  & 3 & 4.000¥ \\
    GM-Nissan Doberman   & 1 & 7 & 1 & 1 & 1 & 1d6  & 3 & 5.000¥ \\
    Steel Lynx           & 1 & 9 & 2 & 1 & 2 & 2d6b & 2 & 9.500¥
}

% \subsubsection{Airborne Drones}
\dronebox{Airborne Drones}{
    Lockheed Optic-X & 1 & 2 & 0 & 2 & 1 & ---  & 2 & 12.500¥ \\
    MCT Roto-Drone   & 2 & 5 & 0 & 1 & 1 & 2d4b & 2 & 15.750¥ \\
    CD Dalmatian     & 1 & 8 & 1 & 0 & 2 & 1d8  & 3 & 22.000¥
}


\section{Other Equipment}

\subsection{Skillsofts}
Skillsofts are data chips that allow an individual to "slot" particular skills into their \refname{skillwire} system, gaining the benefit of the prerecorded knowledge. When you purchase a skillsoft, you must specify what skill area it covers. The following examples are not exhaustive, but should give an idea of what one Skillsoft covers. Skillsofts cannot be used without Skillwires.

\paragraph{Skillsoft -} choose a skill: Biotech, Electronics, Etiquette, Survival, Investigation, Mechanics, Academic Discipline, Pilot, Language; cost: 1,000¥

\textbf{Note:} You must specify a specific area for Etiquette, Pilot and Language and Academic Discipline. \textit{Language (Russian)}, for example, or \textit{Academia - anything you can have, say, a bachelors degree in}.


\subsection{Drugs}
Costs listed below are per dose.

\drugs{Drugs}{
    Bliss & take +1 to \move{Gut Check} & 2h & 15¥ \\
    Cram & take +1 to \move{Null Sweat} & 3h & 10¥ \\
    Deepweed & user can perceive astrally & 1h & 400¥ \\
    Jazz & take +2 to \move{Null Sweat} & 30min & 75¥ \\ 
    Kamikaze & take +1 to \move{Rock \& Roll} and \move{Gut Check} & 1h & 100¥ \\
    Long Haul & you can go without sleep for four days with no consequence & 4d & 50¥ \\
    Nitro & take +2 to \move{Rock \& Roll} and +1 to \move{Gut Check} & 30min & 75¥ \\
    Novacoke & take +1 to Make ‘em Sweat and Check the Situation & 2h & 10¥ \\
    Psyche & take +1 to Drop Science & 3h & 200¥ \\
    Zen & take +1 to \move{Null Sweat} & 30min & 5¥ \\
    BTLs & allow you to experience almost anything virtually & depends & 20-100¥
}


\subsection{Miscellaneous}

Here's some stuff that'll be helpful in one way or another during your runs.

\miscbox{Miscellaneous}{
    Medic Patch     & 1 & heal 2 & 500¥ \\ 
    Stimulant Patch & 1 & take +2 to next move, take 1 stun afterwards & 175¥ \\ 
    Antidote Patch  & 1 & halts poison damage & 200¥ \\ 
    Trauma Patch    & 1 & +1 to First Aid move & 300¥ \\ 
    Quik-H4x Kit    & 4 & bypasses low-grade security locks/electronic devices & 350¥ \\ 
    Spy Kit         & 4 & +1 to Citation Needed or Check the Situation (assuming bugs haven’t been found) & 4.000¥ \\ 
    Counter-surveillance Kit & 4 & +1 to \move{Check the Situation} to search for bugs & 3.000¥ \\ 
    Infiltrator’s Kit & 4 & +1 to \move{Null Sweat} to infiltrate or avoid detection & 1.000¥
}



\section{Magical Supplies}

\subsection{Foci}
A focus is a mundane item that has been imbued with an astral construct. When used by someone to which it is attuned, a focus helps them channel to astral power and greatly enhances their abilities.

\subsubsection{Attuning}
Before a focus can be used, the user must attune themselves to it. To do so, they must invest at least one point of essence into the focus. Essence committed in this fashion remains spent until the user de-attunes themselves from the focus, or the focus is destroyed, at which point the essence is recovered.

\subsubsection{Types of Foci}
\paragraph{Spell Focus -} enhances the casting of a specific spell. When attuned, the mage using the spell focus has \textsf{\textbf{hold}} equal to the Essence spent attuning the focus. Spend this hold toward casting that specific spell.

\paragraph{Spirit Focus -} enhances the summoning of a specific type of spirit. When attuned, the shaman has \textsf{\textbf{hold}} equal to the essence invested in the focus toward summoning that specific spirit type.

\paragraph{Weapon Focus -} primarily used by adepts. When attuned to a weapon focus, the adept using it has \textsf{\textbf{hold}} equal to the invested Essence to spend on the \move{Rock \& Roll} move or on dealing damage.

\subsubsection{Creating a Focus}
Although foci may be purchased from talismongers, street contacts and other sources, sometimes a magic user wishes to create one of their own. To do so, the user must spend two days researching and preparing the object, at the end of which they make the Imbue Focus move:

\paragraph{Imbue Focus -} When you imbue astral power into an object to create a focus, \roll{Skilled}. On 10+, the focus is created normally. On 7-9, the focus is weakly imbued, and requires one additional Essence point to attune (this essence does not count toward the Hold granted by the focus).


\subsection{Fetishes}

Fetishes are essentially one-shot magical supplies — small mundane objects imbued with the structure and energy to cast a spell or to summon a spirit, needing only to be triggered by the mage or shaman.

\subsubsection{Infusing}
To create a fetish, the mage or shaman decides what spell or spirit to place into the fetish, and then infuses the fetish with power, spending the Essence required for the spell or the essence they wish to provide to the spirit. Essence invested in a fetish in this manner remains spent until the fetish is used, at which point it immediately returns.

\subsubsection{Activating a Fetish}
Normally, to cast a spell or to summon a spirit, the mage or shaman must make the Cast a Spell or Conjure moves. With a fetish, this is no longer the case: instead, they can simply declare that they’re using it (making any other moves that the fiction would dictate; \move{Null Sweat} for instance). Once triggered, the stored spell or summoning is immediately carried out. The fetish is good for a single use, after which it crumbles to dust.

    
    % \chapter{Loose Ends}

Laufende (Meta-)Plots, Charakterprogression, noch ausstehende Dinge. Vor allem Dinge, die bereits passiert \textit{sind} oder \textit{definitiv} passieren \textit{werden}, bzw. die meine Spieler*innen direkt betreffen. Das Kapitel \textbf{R\&D} ist dagegen für eher generelle oder szenario-spezifische Sachen da.




\section{Loose indeed}


\subsection{Progress}

Was wollen meine Runner trainieren, lernen, worin wollen sie besser werden?

\begin{easylist}
    # Frosty hat \textit{Stille} und \textit{Gerät analysieren} gelernt - aber Leonie weiß noch nicht bescheid
    # Tusk kann mit dem \textit{Initiationstraining} beginnen und trainiert außerdem den Umgang mit \skill{Exotische Waffen: Monofilamentpeitsche}
    # Yu möchte \skill{Edge}... trainieren? 
    # Rude und Yu haben bereits, glaube ich, \skill{Wissen: Trollgesellschaft (Seattle)} (Rude) und \skill{Spezialisierung: Pistolen} (Yu) gelernt
\end{easylist}




\subsection{Frostburns Van}

Frosty hat ihren Van (einen Ford F-151; \textit{Toyota Gopher} in den Regeln) vor vier Wochen in Downtown geparkt, weil sie von einem Errant angehalten wurde und meinte, gesehen zu haben, dass dieser den Van verwanzt hat. Als sie Yu gebeten hat, den Van einzusammeln, war er nicht mehr da. Die Vermutung liegt nahe, dass er einfach abgeschleppt wurde.




\subsection{Yu's Wohnung}

Yu's Wohnung wurde verwanzt. Der Einbruch machte zwar einen durchaus professionellen Eindruck, allerdings wurde ziemlich ranzige Tech verwendet, was den Runnern wiederum äußerst widersprüchlich erscheint. Neuron konnte die Wanze bis nach \textit{Glow City} in den \textit{Redmond Barrens} zurückverfolgen.

\subsubsection{Behind the Scenes}

Der Hausmeisterei-Dienstleister von Yu's Wohngebäude ist die \textbf{Slupinski Gebäudewartung}(?), die als solche keine großen Probleme hatte, in die Wohnung einzudringen.




\subsection{Meta-Timeline}

Am 12.09. geht der Jackpoint down, dann ist bis zum 01.10. Stille. Detroits ``Befreiung'' findet in der Zeit statt. Huh. Tatsächlich vllt sogar schon am 10.09., zumindest ist Vogels Statement von dann.

Zurück zum 10.09.: Ares hat also einen neuen CEO und ist jetzt in Atlanta. Zwei Tage später, nach den harten Anschuldigungen Vogels, kündigen die UCAS den Business Recognition Accord auf.

Der Konzerngerichtshof braucht 16 Tage, um darauf zu reagieren - u.a. damit, dass Polizei- und Medizindienstleister regulär weitermachen können sollen (ob das bis dahin so war, steht nicht in \textit{Blackout}.

Der erste Blackout ist ca. am 1. November in Philadelphia, weitere einen Tag später. 30T3N spielt auch im November. 


\subsubsection{Der UCrASh}

Der Zusammenbruch der UCAS (bzw. der \textit{\textbf{UCrASh}} findet während der Blackouts statt:

\begin{easylist}
    # 01.11.: Colloton ruft Notstand aus, Wahlen werden verschoben
    # 14.11.: Quebec marschiert in die UCAS ein
    # 25.11.: Waffenstillstand vereinbart, nachdem UCAS militärische Erfolge verbuchen konnten
    # 12.12.: NAN-Streitkräfte greifen UCAS an; das SSC lässt verlauten, \textit{``dass sämtliche Versuche, die Joint Task Force Seattle des UCAS-Militärs durch Salish-Shidhe-Gebiet zu verlegen, als kriegerischer Akt angesehen und entsprechend beantwortet werden würden''}
    # 21.12.: Kentucky verlässt die UCAS und tritt den CAS bei
    # 26.12.: SSC lässt die JTFS auf bestimmten Routen abmarschieren, Marine wird eskortiert, etc; die UCAS haben keinerlei Militärpräsenz an der Westküste mehr
    # 28.12.: Seattle erklärt seine Unabhängigkeit
    # 02.01.: St. Louis ebenfalls
    # 25.01.: die UCAS unterzeichnen Wiedereintritt in die BRA
    # 27.01.: Vizepräsidentin Martin spricht sich für Unabhängigkeit Kanadas aus
\end{easylist}

% bzgl. der Alpha-Bilder:
%
% > Was ist mit den echten Kampfaufnahmen aus Detroit, die eindeutig diese Bugs zeigen?
% Treadle
% > Ein bisschen davon ist durchgesickert, aber die „Experten“ erklären sie schnell für falsch, nennen sie manipuliert und einen Versuch, Ares zu verleumden, indem man sich „die Tragödie in Chicago zunutze macht“ und mit einer falschen Story „von der Wahrheit ablenkt“. Das ärgert mich wirklich massiv. Nach all diesen Jahren hat niemand irgendetwas gelernt.
% Bull








\section{All tied up}

\subsection{Meta-Timeline}

Das III. Armeekorps, knapp 100.000 Leute, verschwinden spurlos am 10. August.

Riflemans Bericht über die Lage in Detroit ist von Ende August. Anfang bis Mitte September ist in Detroit Straßenkampf und Krieg, Deadlines Reports sind aus der Zeit (vllt nochmal nachschauen).

Zum Zeitpunkt von \textit{Midnight Run} ist es eigentlich Mitte August, um den 20en rum (\textbf{Anm. von später:} ich springe ein bisschen vor bis Anfang September). Bei uns könnte die UCAS Armee am nächsten Tag verschwinden, wenn die Runner in Vancouver sind. Die Bundesbehörden werden daraufhin auf höchste Alarmbereitschaft versetzt:

\textbox{Newsflash}{
Drittes Armeekorps der UCAS Army spurlos verschwunden. Regierung ratlos. Bundesbehörden, Militär und lokale Sicherheitsdienstleister sind in Alarmbereitschaft versetzt und beziehen defensive Positionen zur Sicherung ``strategisch wichtiger Punkte''. Präsidentin Colloton zu den immernoch anhaltenden Kämpfen in Detroit: ``Darum soll sich Ares kümmern.''
}


\subsubsection{Der Weg zu Free Seattle}

Am Ende von \textit{Deck Building} sollen die Kämpfe in Detroit enden und Vogel's Business-Bullshit-Rede vorkommen. Der Auslöser für die konkreten Unabhängigkeitsverhandlungen(?) Seattles aus den UCAS sollen die Kündigung der UCAS der BRA sein. Das heißt: die Runner müssen noch mehr vom Turmoil in den UCAS mitbekommen, vom Turmoil innerhalb Seattles deswegen (von wegen mehr Druck auf die Gouverneurin, ihre Wahlversprechen in der Angelegenheit zu verwirklichen), den Austritt aus den BRA und entsprechend das vermeintliche Chaos, das deswegen (auch in Seattle) entsteht. Puuuh...




\subsection{M-TOC Mark I}

Neurons neues Spielzeug.

\begin{easylist}
    # Stand: 30/30 \skill{Elektronik} 
    ## bei 6: Edge umverteilen als Nebenhandlung - alle
    ## bei 12: \textbf{Junk Wall} - +2 auf die Firewall einer verbundenen Cyberbuchse
    ## bei 18: Zugriff auf Ausrüstung, Kommunikation und Vitalmonitordaten (falls vorhanden)
    ## bei 24: Smartgun-Waffen +1 ATK
    ## bei 30: \textbf{Team Leader} - 3 frei auf- und verteilbare Würfel für \skill{Teamwork, Navigation, Wahrnehmung, Kampfmanöver}
\end{easylist}

Yu hat sein Kommlink gepaired, Rude sein Kommlink, seine Ingram-Smartgun und seine Cyberaugen.




\subsection{Deck Building}

Die Runner erhalten den Auftrag am zweiten September.

Yu ist hat sich am gleichen Tag, an dem sie den Auftrag erhalten für ``morgen oder übermorgen'' mit Four-Finger Wong verabredet, um sich eine \skill{Muskelstraffung (Stufe 3)} verpassen zu lassen (Kosten: 90.000\nuyen).

Frosty hat \textit{Stille} zur Hälfte gelernt, \textit{Gerät analysieren} noch gar nicht; sie braucht für beides noch jeweils einen 10-12 Stunden-Tag Ruhe.
    % \chapter{R \& D}


    % \chapter{Origins}

\epigraph{Shame on us, doomed from the start\\God have mercy on our dirty little hearts}{\textit{Nine Inch Nails - Zero Sum}}

\section{Frostburn}
% \paragraph{Orkin, 2.00m groß, 25 Jahre alt (geb. 2055);}
\textbox{Profil}{
    \begin{tabular}{l l l l}
        \textbf{Metatyp:} & Ork & \textbf{Alter:} & 25 Jahre \\
        \textbf{Größe:} & 2.00m & \textbf{Gewicht:} & 115kg
    \end{tabular}
}
Frostburn konnte, ebenso wie ihr großer Bruder Jules, nie besonders viel mit den schamanistischen Traditionen ihrer Leute im allgemeinen - den \textit{Cascade Ork} - und ihrer Familie im besonderen anfangen. Im Gegensatz zu Jules, der in seiner Besessenheit mit allem was Chrom, Öl und Metall ist, per se wenig mit spirituellen Dingen am Hut hat, liegen die Dinge bei Frosty anders: als eine erwachte Person der sechsten Welt wird von ihr erwartet, den Kontakt zu den Stammesgeistern zu pflegen. Sie dagegen suchte ihre ``Erleuchtung'' allerdings lieber in der Wissenschaft und Hermeneutik, studierte an der Universität Seattle und begann eine Karriere als Lohnmagierin im konzerneigenen Geheimdienst des ehemaligen Technologie-Giganten \textit{NeoNET} - wo sie auch an ihren Decknamen kam. Nachdem \textit{NeoNET} 2078 für die KFS-Krise verantwortlich gemacht und zerschlagen wurde, konnte Frostburn aufgrund ihrer Tätigkeiten jedoch nicht ohne weiteres bei der Konkurrenz anheuern (bei denen sie zwar auch auf Headhunting-Listen steht, allerdings im Sinne eines eher feindschaftlichen Personalmanagements) und eine Rückkehr zu den \textit{Cascade} stand ebenfalls außer Frage. Es lag also nahe, dass eine Person mit Frostburns Fähigkeiten und Interessen sich ihren Unterhalt in den Schatten verdient.



\section{Neuron}
% \paragraph{Elfin, 1.78m groß, 43 Jahre alt (geb. 2037);}
\textbox{Profil}{
    \begin{tabular}{l l l l}
        \textbf{Metatyp:} & Elf & \textbf{Alter:} & 43 Jahre \\
        \textbf{Größe:} & 1.78m & \textbf{Gewicht:} & 63kg
    \end{tabular}
}
Laut Kilian: Neuron hat das Studium \$aus\_Gründen nicht geschafft, was in Elfenkreisen schon irgendwie arg peinlich ist und sie geradezu unausweichlich zum schwarzen Schaf macht. Sie hat sich auf der Uni politisiert, was ihr starker Interesse und Engagement an allen Tätigkeiten bzgl. Humanis-Policlubs sowie (evtl.?) die Bekanntschaft mit Brent erklärt (?). Im Zuge der Aneignung (uff) ihrer krassen Skillz hat Neuron Netcat und/oder Metatron kennen gelernt (erstere evtl auch wg. der Politik? vllt auch beides) und nebenbei eine üble Abhängigkeit entwickelt (Anxiety? Depression? Schuldgefühle "versagt zu haben"? egal?).



\section{Rude}
% \paragraph{Troll, 2.40m groß, 29 Jahre alt (geb. 2051);}
\textbox{Profil}{
    \begin{tabular}{l l l l}
        \textbf{Metatyp:} & Troll & \textbf{Alter:} & 29 Jahre \\
        \textbf{Größe:} & 2.40m & \textbf{Gewicht:} & 310kg
    \end{tabular}
}
Rude wuchs im Seattler \textit{Ork-Untergrund} auf, und zwar lange bevor Proposition 23 angenommen und der Untergrund als offizieller Stadtteil anerkannt wurde. Entsprechend hat er kaum formelle Bildung erhalten, was aber nicht 'gar keine' heißt: Seine Rolemodels waren diejenigen unter den Skraacha, die (im Gegensatz zum Star oder den Knights) für Recht und Ordnung unter dem Sprawl gesorgt haben. Sein Handwerk dagegen hat er bei den zwielichtigeren Tätigkeiten der Gang und ihren ``Vertragspartnern'' erlernt - u.a. Teilen des O'Malley Syndicats, also der Casa Nostra Seattles. Was als scheinbar ehrliche und anständige Arbeit begann - Gebäude- und Personenschutz, Fahrdienste, Dealer aus der Gegend schmeißen, etc. - entpuppte sich mehr und mehr als schlicht die andere Seite der gleichen Medaille und kollidierte zunehmend mit Rudes ausgeprägtem Sinn für Gerechtigkeit. Nun schlägt er sich - auf sich allein gestellt oder mit einigen zuverlässigen Kollegen - durch die Schatten, während er dabei zusehen muss, wie sich der \textit{Ork-Untergrund} seiner Kindheit und Jugend durch den Erfolg von Prop 23 fortschreitend gentrifiziert und zu nichts weiter als einer weiteren Konzernenklave wird.



\section{Tusk}
% \paragraph{Orkin, 2.06m groß, 27 Jahre alt (geb. 2053);}
\textbox{Profil}{
    \begin{tabular}{l l l l}
        \textbf{Metatyp:} & Ork & \textbf{Alter:} & 27 Jahre \\
        \textbf{Größe:} & 2.06m & \textbf{Gewicht:} & 122kg
    \end{tabular}
}
Tusk ist eigentlich professionelle Kampfsportlerin. An sich ein toller Job, aber mitunter etwas schlecht bzw. unzuverlässig bezahlt. Außerdem kann die Nahkampfadeptin in einem so stark reglementierten Umfeld wie den traditionellen asiatischen Kampfkünsten nur schwerlich ihr ganzes Potential entfalten (oder ausleben). Die Schatten bieten die perfekte Ergänzung dazu, sowohl was die Herausforderung und den Nervenkitzel angeht, als auch den meißt durchaus soliden Verdienst. Das Problem ist nur: in der der Philosophie des Lethani, die Tusk's Großmeister Arvid lehrt, stellt die Anwendung von Gewalt zu anderen als rein defensiven Zwecken einen absoluten Bruch dar. Eine schwierige Vorraussetzung, um in den Schatten \textit{irgendetwas} zu erreichen...



\section{Yu}
% \paragraph{Elf, 1.80 groß, 28 Jahre alt (geb. 2052);}
\textbox{Profil}{
    \begin{tabular}{l l l l}
        \textbf{Metatyp:} & Elf & \textbf{Alter:} & 28 Jahre \\
        \textbf{Größe:} & 1.80m & \textbf{Gewicht:} & 77kg
    \end{tabular}
}

Yu wuchs inmitten des Schattens der Octagon Triade in Seattle auf und durch seine geradezu intuitive Eloquenz und Ausstrahlung sowie sein diebisches Talent, praktisch überall unter- oder einzutauchen (oder auch an Orte zu gelangen, an denen er eigentlich nichts zu suchen hat), war ihm ein rascher Aufstieg zum Unterhändler der Triade garantiert.

Womit sich Yu jedoch noch nie anfreunden konnte, ist der geradezu lächerlich kräftige Aberglaube, der sich innerhalb des Seattler Kapitels der Octagon nicht nur in den höheren Führungsebenen der Triade zeigt, sondern sogar von dort ausgeht. Es gibt Stimmen, die behaupten, dass darin der Grund liegt, weswegen die Octagon in Seattle im Gegensatz zu ihrem Hong Konger Pendant im Krieg gegen den Yellow Lotus zu unterliegen drohen und führte letzten Endes auch zum Bruch von Yu mit dieser Familie: bei einer Verhandlung mit den Yellow Lotus, die zum Beenden des langjährigen Kriegs zwischen den beiden Triaden beitragen sollte, hat einer der Fusssoldaten in Yu's Gruppe die Nerven verloren, als ein Yellow Lotus - vermutlich schlicht aus versehen - das Feng Shui des Treffpunktes störte. Die Situation eskalierte und endete in einem Blutbad. Yu konnte zwar schwer angeschlagen entkommen und sich von seinem Freund (und Ersatz-Großvater? Mentor? rationalen Bruder im Geiste?) Four-Finger Wong wieder aufpäppeln lassen, aber mit den Octagon hat er seitdem abgeschlossen. 




% Sidenote: der Gelbe Lotus operiert in den Barrens. Vielleicht geht da ja was?
%
% David Gao, der Shan Chu der Octagon in Seattle, ist ziemlich unfähig. Offenbar.
%
% irgendwas mit Octagon-Triade; Untergebene der \textbf{Red Dragon Association} \textit{(Hung Lung Mun)}, die wiederum praktisch vom großen Drachen Lung kontrolliert wird. Während die Octagon (und deren Meister RDA) in Hong Kong ihre Gegner vom \textbf{Gelben Lotus} 2068 praktisch völlig vernichtet haben, ist das in Seattle mitnichten der Fall - den Gelben Lotus geht's gut, die Octagon strugglen (wiki sagt: "die kleinste unter den drei bedeutsamen in Seattle).
    
    % \chapter{Testgelände}

\begin{tikzpicture}
    \draw [color=purplerule] (-0.2,-0.2) -- (0,0) ;
    \draw [color=purplerule] (-0.1,-0.2) -- (0.1,0) ;
    \draw [color=purplerule] (-0.0,-0.2) -- (0.2,0) ;
    \draw [color=purplerule] (0.1,-0.2) -- (0.3,0) ;
    \draw [color=purplerule] (0.2,-0.2) -- (0.4,0) ;
    \draw [color=purplerule] (0.3,-0.2) -- (0.5,0) ;
    \draw [color=purplerule] (0.4,-0.2) -- (0.6,0) ;
    \draw [color=purplerule] (0.5,-0.2) -- (0.7,0) ;
    \draw [color=purplerule] (0.6,-0.2) -- (0.8,0) ;
    \draw [color=purplerule] (0.7,-0.2) -- (0.9,0) ;
\end{tikzpicture}

Und jetzt als Schleife:

\begin{tikzpicture}
    \foreach \s in {-0.2,-0.1,...,0.7}
        \foreach \e in {0,0.1,...,0.9}
        {
            \draw [color=purplerule] (\s,-0.2) -- (\e,0);
        }
\end{tikzpicture}

\foreach \li/\re in {A/X,B/Y,C/Z} {bo-le: \li, to-ri: \re \par}

\subsection{tabellenmakro}

\karma{
allgemein & 3 / 5 \\
Frankreich & 2 / 7 \\
Autobahn & 5
}




Es braucht mitunter auch mal einen \textbullet zum sich austoben.


Leftmark:

\leftmark


Rightmark:

\rightmark

\paragraph{uff} schalalala lorem ipsum so viel text
lieber mal n \subparagraph{subparagraph} machen


\section{Fonts}

\textbf{Bold Text} \\

\textsf{Sans Text} \\

\textbf{\textsf{Bold Sans Text}} \\

\textsf{\textbf{Sans Bold Text}} \\

$\rightarrow$ der sans hat iwie keinen Bold und macht einfach Sabon draus :(

\begin{tcolorbox}[enhanced jigsaw,
                  colback   = graybox,
                  colframe  = grayframe, 
                  opacityback = 0.2,
                  arc = 10pt,
                  fuzzy halo = 1.5mm with grayframe,
                  fontupper = \sffamily\normalsize]
    % \textsf{\textbf{#1}} \par
    \paragraph{Lena Wozniak}
    
    \textsf{Konzernzugehörigkeit:} Ruhrmetall AG, Außendienst Verkauf, Sektor Europa Süd
    
    \textbf{Lizenzen:} Persönliche Bewaffnung (Schwere Pistolen), Cyberware
    
    \textsf{Reservierung:} Wagen 3, Sitzplatz 14 (1. Klasse) 
\end{tcolorbox}

\subsection{Beispiel:}

{\sffamily This is a sample text in \textbf{Sans Serif Font Typeface}}

%\begin{tikzpicture}[remember picture, overlay]
 %  \node ([draw, rotate=270, xshift=5cm, yshift=5cm]current page.center) {Shadowrun: Sechste Welt};
%\end{tikzpicture}

%\begin{tikzpicture} [remember picture, overlay]
%    \draw ([xshift=5cm,yshift=5cm]current page.center) circle [radius=1mm];
%\end{tikzpicture}




\section{Test}

% pro Senkung: huh.
% pro Steigerung: Frosty wurde fucking festgenommen.

\textbox{Mundanes}{
    \begin{tabular}{m{\textwidth}}
    140.000\nuyen insgesamt, ergo 35.000 pro Person (iirc) \\
    eine \textbf{Stufe 6 SIN} inkl. Lizenzen pro Person \\
    keine Änderung an Reputation oder Fahndungsstufe
    \end{tabular}
}

\karma{
    Allgemeine Schwierigkeit & 4 \\
    Kreativität und Roleplay; Beiträge zur Stimmung & 3 / 3 \\
    Erfüllte Teilziele & 1 / 2 \\
    Keine Unbeteiligten zu Schaden gekommen & 1 / 1 \\
    Runner sind nicht aufgefallen & 1 / 1 \\
    Austausch beobachtet oder aufgezeichnet & 0 / 1 
}
    % \chapter{Free Seattle}

\section{Preludium: UC\texorpdfstring{\textsubscript{r}AS\textsuperscript{h}}{Preludium: UCrASh}}

\gesicht{Knapp 3 Wochen sind vergangen, seit die Krise in Detroit für beendet erklärt wurde. Aber ruhig blieb es nicht, ganz im Gegenteil: keine drei Tage später haben die UCAS als Reaktion auf den Bruch mit Ares eine metaphorische Atombombe gezündet, indem sie unilateral den BRA - den \textit{Business Recognition Accords} - aufgekündigt haben. In den BRA ist festgesetzt, dass der Konzerngerichtshof als alleinige Instanz über die Exterritorialität von Konzernen entscheidet. Die UCAS haben den AAA-Megakonzernen also nicht weniger als ihre Geschäftsgrundlagen wie einen Teppich unter den Füßen weggezogen. Im dadurch entstandenen Chaos, auch oder sogar besonders in Seattle, wurden dadurch die Rufe nach der Klärung der Frage nach Seattles Unabhängigkeit lauter. Eine der vielen Nachrichtenmeldungen zu dem Thema lautete zB folgendermaßen:}


\subsubsection{Seattles Zukunft steht auf dem Spiel}

Vor etwas mehr als einem Jahr wurde die Gouverneurin von Seattle, Corinne Potter, mit einem Programm gewählt, das ihr die Stimmenmehrheit brachte, aber nur wenige Versprechungen enthielt. Brackhaven hatte ihr viele ungelöste Probleme hinterlassen, und Potter versprach, sich um jedes einzelne zu kümmern, auch wenn ihre Kampagne nur wenige Details dazu enthielt. Seit ihrer Wahl hat Potter Berater und Experten hinzugezogen, um zu versuchen, die richtige Lösung für jedes dieser Probleme zu finden. Mit ihren Entscheidungen hat sie ihre Wähler manchmal verstimmt, da diese vielleicht erwartet hatten, dass sie eher auf ihr Herz als auf ihren Verstand hört, aber jetzt kümmert sie sich um das vielleicht kontroverseste Thema ihrer Kampagne: Seattles Unabhängigkeitsbewegung. 

Während der Wahl gab es mehrere Forderungen nach einem freien und unabhängigen Seattle. Diese Rufe stießen fast gleichermaßen auf Zustimmung und Ablehnung. Seattle ist in dieser Frage eindeutig tief gespalten. Um ihr bei der Lösung dieser Krise zu helfen, hat Potter Vertreter mehrerer Länder und Megakonzerne eingeladen, die ein besonderes Interesse daran haben, ob Seattle unabhängig wird oder bei den UCAS verbleibt.

Es überrascht nicht, dass Potter mehrere Vertreter des Konzerngerichtshofs eingeladen hat, insbesondere \textbf{Major Brenda Reed} von Ares, \textbf{Thomas Miranda} von Horizon und \textbf{Takuto Nakagawa} von Renraku; dazu kommt noch die Pacific Prosperity Group, die Wuxing-Exec \textbf{Dewei T’ao} an den Verhandlungstisch geschickt hat. Darüber hinaus haben Seattles Nachbarn ein Mitspracherecht: Der Salish-Shidhe-Rat hat \textbf{John Abernathy} vom Salish-Stamm geschickt, die UCAS haben den frischgewählten Kongressabgeordneten \textbf{Carl Derrick} entsandt, um sicherzustellen, dass die Dinge so bleiben, wie sie sind, und Tír Tairngire\footnote{Ich sprech das in etwa \textit{Tier Ta(y)'en'gier} aus, mit stummen Ypsilon.} wird durch \textbf{Margaret Telestrian} vertreten. Die vielleicht überraschendste Einladung zum Treffen ging an \textbf{die Seedrachin}, die auf der Konferenz noch nicht erschienen ist. Was die Anwesenheit eines der umstrittensten Drachen der Welt für diese Konferenz bedeutet, lassen wir mal dahingestellt. 

Im Laufe der nächsten Woche wird Seattles Zukunft zum Teil von diesen Parteien bestimmt werden. Nur die Zeit wird zeigen, ob die Stimme von Seattles Bevölkerung wichtiger sein wird als die Stimmen von Seattles Megakonzernen.


\vspace{1em}
\textbf{An Tusk}: Deine Connection April Summers schrieb für ihre Zeitung in einem Kommentar dazu: 

\gesicht{Das ist vielleicht das klügste politische Spiel, das Potter spielen konnte. Sie kann öffentlich Unterstützung sammeln und bei einigen Mächten das Terrain sondieren, und wenn es schiefläuft, kann sie immer noch den anderen die Schuld zuschieben. Allerdings bedeutet das auch, dass sie das Rampenlicht meidet und nicht die volle Anerkennung für einen möglichen Erfolg erhält. Sie verspielt einen Teil der Publicity des Erfolgs, um die Kritik an einem möglichen Scheitern zu dämpfen. Das wiederum könnte letztendlich als rückgratlos angesehen werden.}

\subsection{Die Lage der Runner}

All das ist natürlich auch in eurem Umfeld ein großes Thema: Yu, du weißt zB von Mia, dass sie Feuer und Flamme für ein unabhängiges Seattle ist:

``Was haben die in DC denn jemals für uns getan? Ihre scheiß Armee sorgt nur für mehr Spannung zwischen dem Council und den anderen NANs als für irgendwelchen angeblichen Schutz! In keinem anderen Metroplex der Welt herrscht ein solches Machtgleichgewicht wie hier, keiner der Megas hier hat dsa Sagen - nicht Ares, keiner der Japanokons, \textbf{niemand} - nichtmal das goldene Würmchen hat hier viel zu melden. Und das liegt sicher nicht an DC, sondern allein an \textbf{uns}, an den Seattler Schatten. Wir sind \textbf{niemandes} Schoßhunde, hier suchen \textbf{wir} uns aus, für wen wir laufen.''. 

Rude, im Burning Hole werden Wetten darauf abgehalten, welche Repräsentanten am Leben bleiben und du hast mit Hez darauf gewettet, dass Seattle unabhängig wird - die Chancen stehen 2:1, auch wenn dort absolut niemand auch nur im Ansatz genug von der Sache versteht, um das einschätzen zu können. Du weißt auch, dass Hez es lieber wäre, wenn es nicht dazu kommt: ``DC ist zwar auch nur ein weiterer Sumpf, aber wenigstens ein weit entfernter. Und immer noch besser, als wenn die Kons hier komplett den Laden übernehmen. Ich mein, die Containment Zone in Chicago strahlt nach wie vor stärker als die Sonne über Fujiyama. Und guck dir doch an, wie es in Detroit aussieht: alles liegt in Schutt und Asche und was macht Ares? Sie verschwinden. Einfach so. Aus ihrer \textit{eigenen} Amerikanischer-als-Uncle-Sam-Vorzeigeenklave. Einfach, weil ein Wiederaufbau zu teuer ist. Und von den anderen fangen wir am besten gar nicht an. Du weißt genauso gut wie ich, wieviele Trogs es bei den Japanos auch nur in der kleinsten Tochter gibt - nicht einen einzigen.''


\section{Ludus Magnus}

Mia hat die Runner zu sich gerufen, um einen Job zu vermitteln; Gilroy 'Romeo' Steele ist auf die Runner aufmerksam geworden und sucht ein Team, dass für die Dauer des Gipfels verschiedene Aufträge für die Gipfelteilnehmer erledigen und ihm Bericht erstatten wird.

\gesicht{``Ich will, dass ihr während der Aufträge sämtliche verwertbaren Informationen und Erkenntnisse über ihre \textbf{Auftraggeber} sammelt, die ihr finden könnt - was bei den Runs jeweils passiert ist \textbf{und was ihr über den Job denkt}. Wie ihr an diese Infos kommt, ist mir egal. Ihr könnt eure Connections bemühen, lokale Quellen ausfindig machen, Aufzeichnungen und Datenspeicher auftreiben oder Zeugen befragen, also lasst euch was einfallen. Und vor allem: seid aufmerksam und lasst euch nichts durch die Finger gehen! Ihr kennt ja das Sprichwort: \textit{Haltet euch den Rücken frei, spart Muni, zielt genau und lasst euch nicht mit Drachen ein.}''}

\end{document}
